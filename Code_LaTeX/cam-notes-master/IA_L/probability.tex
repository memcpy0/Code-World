\documentclass[a4paper]{article}

\def\npart {IA}
\def\nterm {Lent}
\def\nyear {2015}
\def\nlecturer {R.\ Weber}
\def\ncourse {Probability}
\def\nofficial {http://www.statslab.cam.ac.uk/~rrw1/prob/index.html}

\RequirePackage{etex}
\makeatletter
\ifx \nauthor\undefined
  \def\nauthor{Dexter Chua}
\else
\fi

\author{Based on lectures by \nlecturer \\\small Notes taken by \nauthor}
\date{\nterm\ \nyear}

\usepackage{alltt}
\usepackage{amsfonts}
\usepackage{amsmath}
\usepackage{amssymb}
\usepackage{amsthm}
\usepackage{booktabs}
\usepackage{caption}
\usepackage{enumitem}
\usepackage{fancyhdr}
\usepackage{graphicx}
\usepackage{mathdots}
\usepackage{mathtools}
\usepackage{microtype}
\usepackage{multirow}
\usepackage{pdflscape}
\usepackage{pgfplots}
\usepackage{siunitx}
\usepackage{textcomp}
\usepackage{slashed}
\usepackage{tabularx}
\usepackage{tikz}
\usepackage{tkz-euclide}
\usepackage[normalem]{ulem}
\usepackage[all]{xy}
\usepackage{imakeidx}

\makeindex[intoc, title=Index]
\indexsetup{othercode={\lhead{\emph{Index}}}}

\ifx \nextra \undefined
  \usepackage[hidelinks,
    pdfauthor={Dexter Chua},
    pdfsubject={Cambridge Maths Notes: Part \npart\ - \ncourse},
    pdftitle={Part \npart\ - \ncourse},
  pdfkeywords={Cambridge Mathematics Maths Math \npart\ \nterm\ \nyear\ \ncourse}]{hyperref}
  \title{Part \npart\ --- \ncourse}
\else
  \usepackage[hidelinks,
    pdfauthor={Dexter Chua},
    pdfsubject={Cambridge Maths Notes: Part \npart\ - \ncourse\ (\nextra)},
    pdftitle={Part \npart\ - \ncourse\ (\nextra)},
  pdfkeywords={Cambridge Mathematics Maths Math \npart\ \nterm\ \nyear\ \ncourse\ \nextra}]{hyperref}

  \title{Part \npart\ --- \ncourse \\ {\Large \nextra}}
  \renewcommand\printindex{}
\fi

\pgfplotsset{compat=1.12}

\pagestyle{fancyplain}
\ifx \ncoursehead \undefined
\def\ncoursehead{\ncourse}
\fi

\lhead{\emph{\nouppercase{\leftmark}}}
\ifx \nextra \undefined
  \rhead{
    \ifnum\thepage=1
    \else
      \npart\ \ncoursehead
    \fi}
\else
  \rhead{
    \ifnum\thepage=1
    \else
      \npart\ \ncoursehead \ (\nextra)
    \fi}
\fi
\usetikzlibrary{arrows.meta}
\usetikzlibrary{decorations.markings}
\usetikzlibrary{decorations.pathmorphing}
\usetikzlibrary{positioning}
\usetikzlibrary{fadings}
\usetikzlibrary{intersections}
\usetikzlibrary{cd}

\newcommand*{\Cdot}{{\raisebox{-0.25ex}{\scalebox{1.5}{$\cdot$}}}}
\newcommand {\pd}[2][ ]{
  \ifx #1 { }
    \frac{\partial}{\partial #2}
  \else
    \frac{\partial^{#1}}{\partial #2^{#1}}
  \fi
}
\ifx \nhtml \undefined
\else
  \renewcommand\printindex{}
  \DisableLigatures[f]{family = *}
  \let\Contentsline\contentsline
  \renewcommand\contentsline[3]{\Contentsline{#1}{#2}{}}
  \renewcommand{\@dotsep}{10000}
  \newlength\currentparindent
  \setlength\currentparindent\parindent

  \newcommand\@minipagerestore{\setlength{\parindent}{\currentparindent}}
  \usepackage[active,tightpage,pdftex]{preview}
  \renewcommand{\PreviewBorder}{0.1cm}

  \newenvironment{stretchpage}%
  {\begin{preview}\begin{minipage}{\hsize}}%
    {\end{minipage}\end{preview}}
  \AtBeginDocument{\begin{stretchpage}}
  \AtEndDocument{\end{stretchpage}}

  \newcommand{\@@newpage}{\end{stretchpage}\begin{stretchpage}}

  \let\@real@section\section
  \renewcommand{\section}{\@@newpage\@real@section}
  \let\@real@subsection\subsection
  \renewcommand{\subsection}{\@ifstar{\@real@subsection*}{\@@newpage\@real@subsection}}
\fi
\ifx \ntrim \undefined
\else
  \usepackage{geometry}
  \geometry{
    papersize={379pt, 699pt},
    textwidth=345pt,
    textheight=596pt,
    left=17pt,
    top=54pt,
    right=17pt
  }
\fi

\ifx \nisofficial \undefined
\let\@real@maketitle\maketitle
\renewcommand{\maketitle}{\@real@maketitle\begin{center}\begin{minipage}[c]{0.9\textwidth}\centering\footnotesize These notes are not endorsed by the lecturers, and I have modified them (often significantly) after lectures. They are nowhere near accurate representations of what was actually lectured, and in particular, all errors are almost surely mine.\end{minipage}\end{center}}
\else
\fi

% Theorems
\theoremstyle{definition}
\newtheorem*{aim}{Aim}
\newtheorem*{axiom}{Axiom}
\newtheorem*{claim}{Claim}
\newtheorem*{cor}{Corollary}
\newtheorem*{conjecture}{Conjecture}
\newtheorem*{defi}{Definition}
\newtheorem*{eg}{Example}
\newtheorem*{ex}{Exercise}
\newtheorem*{fact}{Fact}
\newtheorem*{law}{Law}
\newtheorem*{lemma}{Lemma}
\newtheorem*{notation}{Notation}
\newtheorem*{prop}{Proposition}
\newtheorem*{question}{Question}
\newtheorem*{problem}{Problem}
\newtheorem*{rrule}{Rule}
\newtheorem*{thm}{Theorem}
\newtheorem*{assumption}{Assumption}

\newtheorem*{remark}{Remark}
\newtheorem*{warning}{Warning}
\newtheorem*{exercise}{Exercise}

\newtheorem{nthm}{Theorem}[section]
\newtheorem{nlemma}[nthm]{Lemma}
\newtheorem{nprop}[nthm]{Proposition}
\newtheorem{ncor}[nthm]{Corollary}


\renewcommand{\labelitemi}{--}
\renewcommand{\labelitemii}{$\circ$}
\renewcommand{\labelenumi}{(\roman{*})}

\let\stdsection\section
\renewcommand\section{\newpage\stdsection}

% Strike through
\def\st{\bgroup \ULdepth=-.55ex \ULset}


%%%%%%%%%%%%%%%%%%%%%%%%%
%%%%% Maths Symbols %%%%%
%%%%%%%%%%%%%%%%%%%%%%%%%

\let\U\relax
\let\C\relax
\let\G\relax

% Matrix groups
\newcommand{\GL}{\mathrm{GL}}
\newcommand{\Or}{\mathrm{O}}
\newcommand{\PGL}{\mathrm{PGL}}
\newcommand{\PSL}{\mathrm{PSL}}
\newcommand{\PSO}{\mathrm{PSO}}
\newcommand{\PSU}{\mathrm{PSU}}
\newcommand{\SL}{\mathrm{SL}}
\newcommand{\SO}{\mathrm{SO}}
\newcommand{\Spin}{\mathrm{Spin}}
\newcommand{\Sp}{\mathrm{Sp}}
\newcommand{\SU}{\mathrm{SU}}
\newcommand{\U}{\mathrm{U}}
\newcommand{\Mat}{\mathrm{Mat}}

% Matrix algebras
\newcommand{\gl}{\mathfrak{gl}}
\newcommand{\ort}{\mathfrak{o}}
\newcommand{\so}{\mathfrak{so}}
\newcommand{\su}{\mathfrak{su}}
\newcommand{\uu}{\mathfrak{u}}
\renewcommand{\sl}{\mathfrak{sl}}

% Special sets
\newcommand{\C}{\mathbb{C}}
\newcommand{\CP}{\mathbb{CP}}
\newcommand{\GG}{\mathbb{G}}
\newcommand{\N}{\mathbb{N}}
\newcommand{\Q}{\mathbb{Q}}
\newcommand{\R}{\mathbb{R}}
\newcommand{\RP}{\mathbb{RP}}
\newcommand{\T}{\mathbb{T}}
\newcommand{\Z}{\mathbb{Z}}
\renewcommand{\H}{\mathbb{H}}

% Brackets
\newcommand{\abs}[1]{\left\lvert #1\right\rvert}
\newcommand{\bket}[1]{\left\lvert #1\right\rangle}
\newcommand{\brak}[1]{\left\langle #1 \right\rvert}
\newcommand{\braket}[2]{\left\langle #1\middle\vert #2 \right\rangle}
\newcommand{\bra}{\langle}
\newcommand{\ket}{\rangle}
\newcommand{\norm}[1]{\left\lVert #1\right\rVert}
\newcommand{\normalorder}[1]{\mathop{:}\nolimits\!#1\!\mathop{:}\nolimits}
\newcommand{\tv}[1]{|#1|}
\renewcommand{\vec}[1]{\boldsymbol{\mathbf{#1}}}

% not-math
\newcommand{\bolds}[1]{{\bfseries #1}}
\newcommand{\cat}[1]{\mathsf{#1}}
\newcommand{\ph}{\,\cdot\,}
\newcommand{\term}[1]{\emph{#1}\index{#1}}
\newcommand{\phantomeq}{\hphantom{{}={}}}
% Probability
\DeclareMathOperator{\Bernoulli}{Bernoulli}
\DeclareMathOperator{\betaD}{beta}
\DeclareMathOperator{\bias}{bias}
\DeclareMathOperator{\binomial}{binomial}
\DeclareMathOperator{\corr}{corr}
\DeclareMathOperator{\cov}{cov}
\DeclareMathOperator{\gammaD}{gamma}
\DeclareMathOperator{\mse}{mse}
\DeclareMathOperator{\multinomial}{multinomial}
\DeclareMathOperator{\Poisson}{Poisson}
\DeclareMathOperator{\var}{var}
\newcommand{\E}{\mathbb{E}}
\newcommand{\Prob}{\mathbb{P}}

% Algebra
\DeclareMathOperator{\adj}{adj}
\DeclareMathOperator{\Ann}{Ann}
\DeclareMathOperator{\Aut}{Aut}
\DeclareMathOperator{\Char}{char}
\DeclareMathOperator{\disc}{disc}
\DeclareMathOperator{\dom}{dom}
\DeclareMathOperator{\fix}{fix}
\DeclareMathOperator{\Hom}{Hom}
\DeclareMathOperator{\id}{id}
\DeclareMathOperator{\image}{image}
\DeclareMathOperator{\im}{im}
\DeclareMathOperator{\re}{re}
\DeclareMathOperator{\tr}{tr}
\DeclareMathOperator{\Tr}{Tr}
\newcommand{\Bilin}{\mathrm{Bilin}}
\newcommand{\Frob}{\mathrm{Frob}}

% Others
\newcommand\ad{\mathrm{ad}}
\newcommand\Art{\mathrm{Art}}
\newcommand{\B}{\mathcal{B}}
\newcommand{\cU}{\mathcal{U}}
\newcommand{\Der}{\mathrm{Der}}
\newcommand{\D}{\mathrm{D}}
\newcommand{\dR}{\mathrm{dR}}
\newcommand{\exterior}{\mathchoice{{\textstyle\bigwedge}}{{\bigwedge}}{{\textstyle\wedge}}{{\scriptstyle\wedge}}}
\newcommand{\F}{\mathbb{F}}
\newcommand{\G}{\mathcal{G}}
\newcommand{\Gr}{\mathrm{Gr}}
\newcommand{\haut}{\mathrm{ht}}
\newcommand{\Hol}{\mathrm{Hol}}
\newcommand{\hol}{\mathfrak{hol}}
\newcommand{\Id}{\mathrm{Id}}
\newcommand{\lie}[1]{\mathfrak{#1}}
\newcommand{\op}{\mathrm{op}}
\newcommand{\Oc}{\mathcal{O}}
\newcommand{\pr}{\mathrm{pr}}
\newcommand{\Ps}{\mathcal{P}}
\newcommand{\pt}{\mathrm{pt}}
\newcommand{\qeq}{\mathrel{``{=}"}}
\newcommand{\Rs}{\mathcal{R}}
\newcommand{\Vect}{\mathrm{Vect}}
\newcommand{\wsto}{\stackrel{\mathrm{w}^*}{\to}}
\newcommand{\wt}{\mathrm{wt}}
\newcommand{\wto}{\stackrel{\mathrm{w}}{\to}}
\renewcommand{\d}{\mathrm{d}}
\renewcommand{\P}{\mathbb{P}}
%\renewcommand{\F}{\mathcal{F}}


\let\Im\relax
\let\Re\relax

\DeclareMathOperator{\area}{area}
\DeclareMathOperator{\card}{card}
\DeclareMathOperator{\ccl}{ccl}
\DeclareMathOperator{\ch}{ch}
\DeclareMathOperator{\cl}{cl}
\DeclareMathOperator{\cls}{\overline{\mathrm{span}}}
\DeclareMathOperator{\coker}{coker}
\DeclareMathOperator{\conv}{conv}
\DeclareMathOperator{\cosec}{cosec}
\DeclareMathOperator{\cosech}{cosech}
\DeclareMathOperator{\covol}{covol}
\DeclareMathOperator{\diag}{diag}
\DeclareMathOperator{\diam}{diam}
\DeclareMathOperator{\Diff}{Diff}
\DeclareMathOperator{\End}{End}
\DeclareMathOperator{\energy}{energy}
\DeclareMathOperator{\erfc}{erfc}
\DeclareMathOperator{\erf}{erf}
\DeclareMathOperator*{\esssup}{ess\,sup}
\DeclareMathOperator{\ev}{ev}
\DeclareMathOperator{\Ext}{Ext}
\DeclareMathOperator{\fst}{fst}
\DeclareMathOperator{\Fit}{Fit}
\DeclareMathOperator{\Frac}{Frac}
\DeclareMathOperator{\Gal}{Gal}
\DeclareMathOperator{\gr}{gr}
\DeclareMathOperator{\hcf}{hcf}
\DeclareMathOperator{\Im}{Im}
\DeclareMathOperator{\Ind}{Ind}
\DeclareMathOperator{\Int}{Int}
\DeclareMathOperator{\Isom}{Isom}
\DeclareMathOperator{\lcm}{lcm}
\DeclareMathOperator{\length}{length}
\DeclareMathOperator{\Lie}{Lie}
\DeclareMathOperator{\like}{like}
\DeclareMathOperator{\Lk}{Lk}
\DeclareMathOperator{\Maps}{Maps}
\DeclareMathOperator{\orb}{orb}
\DeclareMathOperator{\ord}{ord}
\DeclareMathOperator{\otp}{otp}
\DeclareMathOperator{\poly}{poly}
\DeclareMathOperator{\rank}{rank}
\DeclareMathOperator{\rel}{rel}
\DeclareMathOperator{\Rad}{Rad}
\DeclareMathOperator{\Re}{Re}
\DeclareMathOperator*{\res}{res}
\DeclareMathOperator{\Res}{Res}
\DeclareMathOperator{\Ric}{Ric}
\DeclareMathOperator{\rk}{rk}
\DeclareMathOperator{\Rees}{Rees}
\DeclareMathOperator{\Root}{Root}
\DeclareMathOperator{\sech}{sech}
\DeclareMathOperator{\sgn}{sgn}
\DeclareMathOperator{\snd}{snd}
\DeclareMathOperator{\Spec}{Spec}
\DeclareMathOperator{\spn}{span}
\DeclareMathOperator{\stab}{stab}
\DeclareMathOperator{\St}{St}
\DeclareMathOperator{\supp}{supp}
\DeclareMathOperator{\Syl}{Syl}
\DeclareMathOperator{\Sym}{Sym}
\DeclareMathOperator{\vol}{vol}

\pgfarrowsdeclarecombine{twolatex'}{twolatex'}{latex'}{latex'}{latex'}{latex'}
\tikzset{->/.style = {decoration={markings,
                                  mark=at position 1 with {\arrow[scale=2]{latex'}}},
                      postaction={decorate}}}
\tikzset{<-/.style = {decoration={markings,
                                  mark=at position 0 with {\arrowreversed[scale=2]{latex'}}},
                      postaction={decorate}}}
\tikzset{<->/.style = {decoration={markings,
                                   mark=at position 0 with {\arrowreversed[scale=2]{latex'}},
                                   mark=at position 1 with {\arrow[scale=2]{latex'}}},
                       postaction={decorate}}}
\tikzset{->-/.style = {decoration={markings,
                                   mark=at position #1 with {\arrow[scale=2]{latex'}}},
                       postaction={decorate}}}
\tikzset{-<-/.style = {decoration={markings,
                                   mark=at position #1 with {\arrowreversed[scale=2]{latex'}}},
                       postaction={decorate}}}
\tikzset{->>/.style = {decoration={markings,
                                  mark=at position 1 with {\arrow[scale=2]{latex'}}},
                      postaction={decorate}}}
\tikzset{<<-/.style = {decoration={markings,
                                  mark=at position 0 with {\arrowreversed[scale=2]{twolatex'}}},
                      postaction={decorate}}}
\tikzset{<<->>/.style = {decoration={markings,
                                   mark=at position 0 with {\arrowreversed[scale=2]{twolatex'}},
                                   mark=at position 1 with {\arrow[scale=2]{twolatex'}}},
                       postaction={decorate}}}
\tikzset{->>-/.style = {decoration={markings,
                                   mark=at position #1 with {\arrow[scale=2]{twolatex'}}},
                       postaction={decorate}}}
\tikzset{-<<-/.style = {decoration={markings,
                                   mark=at position #1 with {\arrowreversed[scale=2]{twolatex'}}},
                       postaction={decorate}}}

\tikzset{circ/.style = {fill, circle, inner sep = 0, minimum size = 3}}
\tikzset{scirc/.style = {fill, circle, inner sep = 0, minimum size = 1.5}}
\tikzset{mstate/.style={circle, draw, blue, text=black, minimum width=0.7cm}}

\tikzset{eqpic/.style={baseline={([yshift=-.5ex]current bounding box.center)}}}
\tikzset{commutative diagrams/.cd,cdmap/.style={/tikz/column 1/.append style={anchor=base east},/tikz/column 2/.append style={anchor=base west},row sep=tiny}}

\definecolor{mblue}{rgb}{0.2, 0.3, 0.8}
\definecolor{morange}{rgb}{1, 0.5, 0}
\definecolor{mgreen}{rgb}{0.1, 0.4, 0.2}
\definecolor{mred}{rgb}{0.5, 0, 0}

\def\drawcirculararc(#1,#2)(#3,#4)(#5,#6){%
    \pgfmathsetmacro\cA{(#1*#1+#2*#2-#3*#3-#4*#4)/2}%
    \pgfmathsetmacro\cB{(#1*#1+#2*#2-#5*#5-#6*#6)/2}%
    \pgfmathsetmacro\cy{(\cB*(#1-#3)-\cA*(#1-#5))/%
                        ((#2-#6)*(#1-#3)-(#2-#4)*(#1-#5))}%
    \pgfmathsetmacro\cx{(\cA-\cy*(#2-#4))/(#1-#3)}%
    \pgfmathsetmacro\cr{sqrt((#1-\cx)*(#1-\cx)+(#2-\cy)*(#2-\cy))}%
    \pgfmathsetmacro\cA{atan2(#2-\cy,#1-\cx)}%
    \pgfmathsetmacro\cB{atan2(#6-\cy,#5-\cx)}%
    \pgfmathparse{\cB<\cA}%
    \ifnum\pgfmathresult=1
        \pgfmathsetmacro\cB{\cB+360}%
    \fi
    \draw (#1,#2) arc (\cA:\cB:\cr);%
}
\newcommand\getCoord[3]{\newdimen{#1}\newdimen{#2}\pgfextractx{#1}{\pgfpointanchor{#3}{center}}\pgfextracty{#2}{\pgfpointanchor{#3}{center}}}

\newcommand\qedshift{\vspace{-17pt}}
\newcommand\fakeqed{\pushQED{\qed}\qedhere}

\def\Xint#1{\mathchoice
   {\XXint\displaystyle\textstyle{#1}}%
   {\XXint\textstyle\scriptstyle{#1}}%
   {\XXint\scriptstyle\scriptscriptstyle{#1}}%
   {\XXint\scriptscriptstyle\scriptscriptstyle{#1}}%
   \!\int}
\def\XXint#1#2#3{{\setbox0=\hbox{$#1{#2#3}{\int}$}
     \vcenter{\hbox{$#2#3$}}\kern-.5\wd0}}
\def\ddashint{\Xint=}
\def\dashint{\Xint-}

\newcommand\separator{{\centering\rule{2cm}{0.2pt}\vspace{2pt}\par}}

\newenvironment{own}{\color{gray!70!black}}{}

\newcommand\makecenter[1]{\raisebox{-0.5\height}{#1}}

\mathchardef\mdash="2D

\newenvironment{significant}{\begin{center}\begin{minipage}{0.9\textwidth}\centering\em}{\end{minipage}\end{center}}
\DeclareRobustCommand{\rvdots}{%
  \vbox{
    \baselineskip4\p@\lineskiplimit\z@
    \kern-\p@
    \hbox{.}\hbox{.}\hbox{.}
  }}
\DeclareRobustCommand\tph[3]{{\texorpdfstring{#1}{#2}}}
\makeatother


\begin{document}
\maketitle
{\small
  \noindent\textbf{Basic concepts}\\
  Classical probability, equally likely outcomes. Combinatorial analysis, permutations and combinations. Stirling's formula (asymptotics for $\log n!$ proved).\hspace*{\fill} [3]

  \vspace{10pt}
  \noindent\textbf{Axiomatic approach}\\
  Axioms (countable case). Probability spaces. Inclusion-exclusion formula. Continuity and subadditivity of probability measures. Independence. Binomial, Poisson and geometric distributions. Relation between Poisson and binomial distributions. Conditional probability, Bayes's formula. Examples, including Simpson's paradox.\hspace*{\fill} [5]

  \vspace{10pt}
  \noindent\textbf{Discrete random variables}\\
  Expectation. Functions of a random variable, indicator function, variance, standard deviation. Covariance, independence of random variables. Generating functions: sums of independent random variables, random sum formula, moments.

  \vspace{5pt}
  \noindent Conditional expectation. Random walks: gambler's ruin, recurrence relations. Difference equations and their solution. Mean time to absorption. Branching processes: generating functions and extinction probability. Combinatorial applications of generating functions.\hspace*{\fill} [7]

  \vspace{10pt}
  \noindent\textbf{Continuous random variables}\\
  Distributions and density functions. Expectations; expectation of a function of a random variable. Uniform, normal and exponential random variables. Memoryless property of exponential distribution. Joint distributions: transformation of random variables (including Jacobians), examples. Simulation: generating continuous random variables, independent normal random variables. Geometrical probability: Bertrand's paradox, Buffon's needle. Correlation coefficient, bivariate normal random variables.\hspace*{\fill} [6]

  \vspace{10pt}
  \noindent\textbf{Inequalities and limits}\\
  Markov's inequality, Chebyshev's inequality. Weak law of large numbers. Convexity: Jensen's inequality for general random variables, AM/GM inequality.

  \vspace{5pt}
  \noindent Moment generating functions and statement (no proof) of continuity theorem. Statement of central limit theorem and sketch of proof. Examples, including sampling.\hspace*{\fill} [3]}

\tableofcontents
\setcounter{section}{-1}
\section{Introduction}
In every day life, we often encounter the use of the term probability, and they are used in many different ways. For example, we can hear people say:
\begin{enumerate}
  \item The probability that a fair coin will land heads is $1/2$.
  \item The probability that a selection of 6 members wins the National Lottery Lotto jackpot is 1 in $\binom{19}{6} = 13 983 816$ or $7.15112\times 10^{-8}$.
  \item The probability that a drawing pin will land 'point up' is 0.62.
  \item The probability that a large earthquake will occur on the San Andreas Fault in the next 30 years is about $21\%$
  \item The probability that humanity will be extinct by 2100 is about $50\%$
\end{enumerate}
The first two cases are things derived from logic. For example, we know that the coin either lands heads or tails. By definition, a fair coin is equally likely to land heads or tail. So the probability of either must be $1/2$.

The third is something probably derived from experiments. Perhaps we did 1000 experiments and 620 of the pins point up. The fourth and fifth examples belong to yet another category that talks about are beliefs and predictions.

We call the first kind ``classical probability'', the second kind ``frequentist probability'' and the last ``subjective probability''. In this course, we only consider classical probability.

\section{Classical probability}
We start with a rather informal introduction to probability. Afterwards, in Chapter~\ref{sec:axioms}, we will have a formal axiomatic definition of probability and formally study their properties.

\subsection{Classical probability}
\begin{defi}[Classical probability]
  \emph{Classical probability} applies in a situation when there are a finite number of equally likely outcome.
\end{defi}

A classical example is the problem of points.

\begin{eg}
  $A$ and $B$ play a game in which they keep throwing coins. If a head lands, then $A$ gets a point. Otherwise, $B$ gets a point. The first person to get 10 points wins a prize.

  Now suppose $A$ has got 8 points and $B$ has got 7, but the game has to end because an earthquake struck. How should they divide the prize? We answer this by finding the probability of $A$ winning. Someone must have won by the end of 19 rounds, i.e.\ after 4 more rounds. If $A$ wins at least 2 of them, then $A$ wins. Otherwise, $B$ wins.

  The number of ways this can happen is $\binom{4}{2} + \binom{4}{3} + \binom{4}{4} = 11$, while there are $16$ possible outcomes in total. So $A$ should get $11/16$ of the prize.
\end{eg}

In general, consider an experiment that has a random outcome.

\begin{defi}[Sample space]
  The set of all possible outcomes is the \emph{sample space}, $\Omega$. We can lists the outcomes as $\omega_1, \omega_2, \cdots \in \Omega$. Each $\omega \in \Omega$ is an \emph{outcome}.
\end{defi}

\begin{defi}[Event]
  A subset of $\Omega$ is called an \emph{event}.
\end{defi}

\begin{eg}
  When rolling a dice, the sample space is $\{1, 2, 3, 4, 5, 6\}$, and each item is an outcome. ``Getting an odd number'' and ``getting 3'' are two possible events.
\end{eg}

In probability, we will be dealing with sets a lot, so it would be helpful to come up with some notation.
\begin{defi}[Set notations]
  Given any two events $A, B\subseteq \Omega$,
  \begin{itemize}
    \item The \emph{complement} of $A$ is $A^C = A' = \bar A = \Omega\setminus A$.
    \item ``$A$ or $B$'' is the set $A\cup B$.
    \item ``$A$ and $B$'' is the set $A\cap B$.
    \item $A$ and $B$ are \emph{mutually exclusive} or \emph{disjoint} if $A\cap B = \emptyset$.
    \item If $A\subseteq B$, then $A$ occurring implies $B$ occurring.
  \end{itemize}
\end{defi}

\begin{defi}[Probability]
  Suppose $\Omega = \{\omega_1,\omega_2, \cdots, \omega_N\}$. Let $A\subseteq \Omega$ be an event. Then the \emph{probability} of $A$ is
  \[
    \P(A) = \frac{\text{Number of outcomes in } A}{\text{Number of outcomes in }\Omega} = \frac{|A|}{N}.
  \]
\end{defi}
Here we are assuming that each outcome is equally likely to happen, which is the case in (fair) dice rolls and coin flips.

\begin{eg}
  Suppose $r$ digits are drawn at random from a table of random digits from 0 to 9. What is the probability that
  \begin{enumerate}
    \item No digit exceeds $k$;
    \item The largest digit drawn is $k$?
  \end{enumerate}

  The sample space is $\Omega = \{(a_1, a_2, \cdots, a_r): 0 \leq a_i \leq 9\}$. Then $|\Omega| = 10^r$.

  Let $A_k = [\text{no digit exceeds }k] = \{(a_1, \cdots, a_r): 0 \leq a_i \leq k\}$. Then $|A_k| = (k + 1)^r$. So
  \[
    P(A_k) = \frac{(k + 1)^r}{10^r}.
  \]
  Now let $B_k = [\text{largest digit drawn is }k]$. We can find this by finding all outcomes in which no digits exceed $k$, and subtract it by the number of outcomes in which no digit exceeds $k - 1$. So $|B_k| = |A_k| - |A_{k - 1}|$ and
  \[
    P(B_k) = \frac{(k + 1)^r - k^r}{10^r}.
  \]
\end{eg}
\subsection{Counting}
To find probabilities, we often need to \emph{count} things. For example, in our example above, we had to count the number of elements in $B_k$.
\begin{eg}
  A menu has 6 starters, 7 mains and 6 desserts. How many possible meals combinations are there? Clearly $6 \times 7 \times 6 = 252$.
\end{eg}
Here we are using the fundamental rule of counting:
\begin{thm}[Fundamental rule of counting]
  Suppose we have to make $r$ multiple choices in sequence. There are $m_1$ possibilities for the first choice, $m_2$ possibilities for the second etc. Then the total number of choices is $m_1\times m_2\times \cdots m_r$.
\end{thm}

\begin{eg}
  How many ways can $1, 2, \cdots, n$ be ordered? The first choice has $n$ possibilities, the second has $n - 1$ possibilities etc. So there are $n\times (n - 1)\times\cdots \times 1 = n!$.
\end{eg}

\subsubsection*{Sampling with or without replacement}
Suppose we have to pick $n$ items from a total of $x$ items. We can model this as follows: Let $N = \{1, 2, \cdots, n\}$ be the list. Let $X = \{1, 2, \cdots, x\}$ be the items. Then each way of picking the items is a function $f: N\to X$ with $f(i) =$ item at the $i$th position.

\begin{defi}[Sampling with replacement]
  When we \emph{sample with replacement}, after choosing at item, it is put back and can be chosen again. Then \emph{any} sampling function $f$ satisfies sampling with replacement.
\end{defi}

\begin{defi}[Sampling without replacement]
  When we \emph{sample without replacement}, after choosing an item, we kill it with fire and cannot choose it again. Then $f$ must be an injective function, and clearly we must have $x \geq n$.
\end{defi}

We can also have sampling with replacement, but we require each item to be chosen at least once. In this case, $f$ must be surjective.

\begin{eg}
  Suppose $N = \{a, b, c\}$ and $X = \{p, q, r, s\}$. How many injective functions are there $N\to X$?

  When we choose $f(a)$, we have 4 options. When we choose $f(b)$, we have 3 left. When we choose $f(c)$, we have 2 choices left. So there are $24$ possible choices.
\end{eg}

\begin{eg}
  I have $n$ keys in my pocket. We select one at random once and try to unlock. What is the possibility that I succeed at the $r$th trial?

  Suppose we do it with replacement. We have to fail the first $r - 1$ trials and succeed in the $r$th. So the probability is
  \[
    \frac{(n - 1)(n - 1) \cdots (n - 1)(1)}{n^r} = \frac{(n - 1)^{r - 1}}{n^r}.
  \]
  Now suppose we are smarter and try without replacement. Then the probability is
  \[
    \frac{(n - 1)(n - 2)\cdots (n - r + 1)(1)}{n(n - 1) \cdots (n - r + 1)} = \frac{1}{n}.
  \]
\end{eg}
\begin{eg}[Birthday problem]
  How many people are needed in a room for there to be a probability that two people have the same birthday to be at least a half?

  Suppose $f(r)$ is the probability that, in a room of $r$ people, there is a birthday match.

  We solve this by finding the probability of no match, $1 - f(r)$. The total number of possibilities of birthday combinations is $365^r$. For nobody to have the same birthday, the first person can have any birthday. The second has 364 else to choose, etc. So
  \[
    \P(\text{no match}) = \frac{365\cdot 364\cdot 363 \cdots (366 - r)}{365\cdot 365\cdot 365 \cdots 365}.
  \]
  If we calculate this with a computer, we find that $f(22) = 0.475695$ and $f(23) = 0.507297$.

  While this might sound odd since 23 is small, this is because we are thinking about the wrong thing. The probability of match is related more to the number of \emph{pairs} of people, not the number of people. With 23 people, we have $23\times 22/2 = 253$ pairs, which is quite large compared to 365.
\end{eg}
\subsubsection*{Sampling with or without regard to ordering}
There are cases where we don't care about, say list positions. For example, if we pick two representatives from a class, the order of picking them doesn't matter.

In terms of the function $f: N\to X$, after mapping to $f(1), f(2), \cdots, f(n)$, we can
\begin{itemize}
  \item Leave the list alone
  \item Sort the list ascending. i.e.\ we might get $(2, 5, 4)$ and $(4, 2, 5)$. If we don't care about list positions, these are just equivalent to $(2, 4, 5)$.
  \item Re-number each item by the number of the draw on which it was first seen. For example, we can rename $(2, 5, 2)$ and $(5, 4, 5)$ both as $(1, 2, 1)$. This happens if the labelling of items doesn't matter.
  \item Both of above. So we can rename $(2, 5, 2)$ and $(8, 5, 5)$ both as $(1, 1, 2)$.
\end{itemize}

\subsubsection*{Total number of cases}
Combining these four possibilities with whether we have replacement, no replacement, or ``everything has to be chosen at least once'', we have 12 possible cases of counting. The most important ones are:
\begin{itemize}
  \item Replacement + with ordering: the number of ways is $x^n$.
  \item Without replacement + with ordering: the number of ways is $x_{(n)} = x^{\underline{n}} = x(x - 1)\cdots (x - n + 1)$.
  \item Without replacement + without order: we only care which items get selected. The number of ways is $\binom{x}{n} = C^x_n = x_{(n)}/n!$.
  \item Replacement + without ordering: we only care how many times the item got chosen. This is equivalent to partitioning $n$ into $n_1 + n_2 + \cdots + n_k$. Say $n = 6$ and $k = 3$. We can write a particular partition as
    \[
      **\mid *\mid ***
    \]
  So we have $n + k - 1$ symbols and $k - 1$ of them are bars. So the number of ways is $\binom{n + k - 1}{k - 1}$.
\end{itemize}

\subsubsection*{Multinomial coefficient}
Suppose that we have to pick $n$ items, and each item can either be an apple or an orange. The number of ways of picking such that $k$ apples are chosen is, by definition, $\binom{n}{k}$.

In general, suppose we have to fill successive positions in a list of length $n$, with replacement, from a set of $k$ items. The number of ways of doing so such that item $i$ is picked $n_i$ times is defined to be the \emph{multinomial coefficient} $\binom{n}{n_1, n_2, \cdots, n_k}$.

\begin{defi}[Multinomial coefficient]
  A \emph{multinomial coefficient} is
  \[
    \binom{n}{n_1, n_2, \cdots, n_k} = \binom{n}{n_1}\binom{n - n_1}{n_2}\cdots \binom{n - n_1\cdots - n_{k - 1}}{n_k} = \frac{n!}{n_1!n_2!\cdots n_k!}.
  \]
  It is the number of ways to distribute $n$ items into $k$ positions, in which the $i$th position has $n_i$ items.
\end{defi}

\begin{eg}
  We know that
  \[
    (x + y)^n = x^n + \binom{n}{1}x^{n - 1}y + \cdots + y^n.
  \]
  If we have a trinomial, then
  \[
    (x + y + z)^n = \sum_{n_1, n_2, n_3} \binom{n}{n_1, n_2, n_3} x^{n_1}y^{n_2}z^{n_3}.
  \]
\end{eg}

\begin{eg}
  How many ways can we deal 52 cards to 4 player, each with a hand of 13? The total number of ways is
  \[
    \binom{52}{13, 13, 13, 13} = \frac{52!}{(13!)^4} = 53644737765488792839237440000 = 5.36\times 10^{28}.
  \]
\end{eg}


While computers are still capable of calculating that, what if we tried more \st{power} cards? Suppose each person has $n$ cards. Then the number of ways is
\[
  \frac{(4n)!}{(n!)^4},
\]
which is \emph{huge}. We can use Stirling's Formula to approximate it:
\subsection{Stirling's formula}
Before we state and prove Stirling's formula, we prove a weaker (but examinable) version:
\begin{prop}
  $\log n!\sim n\log n$
\end{prop}
\begin{proof}
  Note that
  \[
    \log n! = \sum_{k=1}^n \log k.
  \]
  Now we claim that
  \[
    \int_1^n \log x\;\d x \leq \sum_1^n \log k\leq \int_1^{n + 1}\log x\;\d x.
  \]
  This is true by considering the diagram:
  \begin{center}
    \begin{tikzpicture}[xscale=0.7, yscale=1.3]
     \draw [->] (0, 0) -- (11, 0) node [right] {$x$};
     \draw [->] (0, 0) -- (0, 2.5) node [above] {$y$};
     \draw [domain=2:10] plot (\x, {ln (\x - 1)}) node [anchor = south west] {$\ln x$};
     \draw [domain=1:10] plot (\x, {ln \x }) node [anchor = north west] {$\ln (x - 1)$};
     \draw [fill = black, fill opacity=0.2] (2, 0)-- (2, 0.693147180559945) -- (3, 0.693147180559945) -- (3, 0);
     \draw [fill = black, fill opacity=0.2] (3, 0)-- (3, 1.09861228866811) -- (4, 1.09861228866811) -- (4, 0);
     \draw [fill = black, fill opacity=0.2] (4, 0)-- (4, 1.38629436111989) -- (5, 1.38629436111989) -- (5, 0);
     \draw [fill = black, fill opacity=0.2] (5, 0)-- (5, 1.6094379124341) -- (6, 1.6094379124341) -- (6, 0);
     \draw [fill = black, fill opacity=0.2] (6, 0)-- (6, 1.79175946922806) -- (7, 1.79175946922806) -- (7, 0);
     \draw [fill = black, fill opacity=0.2] (7, 0)-- (7, 1.94591014905531) -- (8, 1.94591014905531) -- (8, 0);
     \draw [fill = black, fill opacity=0.2] (8, 0)-- (8, 2.07944154167984) -- (9, 2.07944154167984) -- (9, 0);
     \draw [fill = black, fill opacity=0.2] (9, 0)-- (9, 2.19722457733622) -- (10, 2.19722457733622) -- (10, 0);
    \end{tikzpicture}
  \end{center}
  We actually evaluate the integral to obtain
  \[
    n\log n - n + 1 \leq \log n! \leq (n+ 1)\log (n + 1) - n;
  \]
  Divide both sides by $n\log n$ and let $n\to \infty$. Both sides tend to $1$. So
  \[
    \frac{\log n!}{n\log n} \to 1.\qedhere
  \]
\end{proof}

Now we prove Stirling's Formula:
\begin{thm}[Stirling's formula]
  As $n\to \infty$,
  \[
    \log\left(\frac{n! e^n}{n^{n + \frac{1}{2}}}\right) = \log \sqrt{2\pi} + O\left(\frac{1}{n}\right)
  \]
\end{thm}
\begin{cor}
  \[
    n!\sim \sqrt{2\pi}n^{n + \frac{1}{2}} e^{-n}
  \]
\end{cor}
\begin{proof}(non-examinable)
  Define
  \[
    d_n = \log \left(\frac{n!e^n}{n^{n + 1/2}}\right) = \log n! - (n + 1/2)\log n + n
  \]
  Then
  \[
    d_n - d_{n + 1} = (n + 1/2)\log\left(\frac{n + 1}{n}\right) - 1.
  \]
  Write $t = 1/(2n + 1)$. Then
  \[
    d_n - d_{n + 1} = \frac{1}{2t}\log\left(\frac{1 + t}{1 - t}\right) - 1.
  \]
  We can simplifying by noting that
  \begin{align*}
    \log (1 + t) - t &= -\frac{1}{2}t^2 + \frac{1}{3}t^3 - \frac{1}{4}t^4 + \cdots\\
    \log (1 - t) + t &= -\frac{1}{2}t^2 - \frac{1}{3}t^3 - \frac{1}{4}t^4 - \cdots
  \end{align*}
  Then if we subtract the equations and divide by $2t$, we obtain
  \begin{align*}
    d_n - d_{n + 1} &= \frac{1}{3}t^2 + \frac{1}{5}t^4 + \frac{1}{7}t^6 + \cdots\\
    &< \frac{1}{3}t^2 + \frac{1}{3}t^4 + \frac{1}{3}t^6 + \cdots\\
    &= \frac{1}{3}\frac{t^2}{1 - t^2}\\
    &= \frac{1}{3}\frac{1}{(2n + 1)^2 - 1}\\
    &= \frac{1}{12}\left(\frac{1}{n} - \frac{1}{n + 1}\right)
  \end{align*}
  By summing these bounds, we know that
  \[
    d_1 - d_n < \frac{1}{12}\left(1 - \frac{1}{n}\right)
  \]
  Then we know that $d_n$ is bounded below by $d_1 +$ something, and is decreasing since $d_n - d_{n + 1}$ is positive. So it converges to a limit $A$. We know $A$ is a lower bound for $d_n$ since $(d_n)$ is decreasing.

  Suppose $m > n$. Then $d_n - d_m < \left(\frac{1}{n} - \frac{1}{m}\right)\frac{1}{12}$. So taking the limit as $m\to \infty$, we obtain an upper bound for $d_n$: $d_n < A + 1/(12n)$. Hence we know that
  \[
    A < d_n < A + \frac{1}{12n}.
  \]
  However, all these results are useless if we don't know what $A$ is. To find $A$, we have a small detour to prove a formula:

  Take $I_n = \int_0^{\pi/2} \sin^n\theta\;\d \theta$. This is decreasing for increasing $n$ as $\sin^n\theta$ gets smaller. We also know that
  \begin{align*}
    I_n &= \int_0^{\pi/2}\sin^n \theta\;\d \theta\\
    &= \left[-\cos\theta\sin^{n - 1}\theta \right]_0^{\pi/2} + \int_0^{\pi/2} (n - 1)\cos^2\theta \sin^{n - 2}\theta \;\d\theta\\
    &= 0 + \int_0^{\pi/2}(n - 1)(1 - \sin^2 \theta)\sin^{n - 2}\theta \;\d \theta\\
    &= (n - 1)(I_{n - 2} - I_n)
  \end{align*}
  So
  \[
    I_n = \frac{n - 1}{n}I_{n - 2}.
  \]
  We can directly evaluate the integral to obtain $I_0 = \pi/2$, $I_1 = 1$. Then
  \begin{align*}
    I_{2n} &= \frac{1}{2}\cdot\frac{3}{4}\cdots \frac{2n - 1}{2n} \pi/2 = \frac{(2n)!}{(2^nn!)^2}\frac{\pi}{2}\\
    I_{2n + 1} &= \frac{2}{3}\cdot\frac{4}{5}\cdots\frac{2n}{2n + 1} = \frac{(2^nn!)^2}{(2n + 1)!}
  \end{align*}
  So using the fact that $I_n$ is decreasing, we know that
  \[
    1 \leq \frac{I_{2n}}{I_{2n + 1}} \leq \frac{I_{2n - 1}}{I_{2n + 1}} = 1 + \frac{1}{2n} \to 1.
  \]
  Using the approximation $n!\sim n^{n + 1/2}e^{-n + A}$, where $A$ is the limit we want to find, we can approximate
  \[
    \frac{I_{2n}}{I_{2n + 1}} = \pi(2n + 1)\left[\frac{( (2n)!)^2}{2^{4n + 1}(n!)^4}\right] \sim \pi(2n + 1)\frac{1}{ne^{2A}}\to \frac{2\pi}{e^{2A}}.
  \]
  Since the last expression is equal to 1, we know that $A = \log\sqrt{2\pi}$. Hooray for magic!
\end{proof}

This approximation can be improved:
\begin{prop}[non-examinable]
  We use the $1/12n$ term from the proof above to get a better approximation:
  \[
    \sqrt{2\pi}n^{n + 1/2}e^{-n + \frac{1}{12n + 1}} \leq n! \leq \sqrt{2\pi} n^{n + 1/2} e^{-n + \frac{1}{12n}}.
  \]
\end{prop}

\begin{eg}
  Suppose we toss a coin $2n$ times. What is the probability of equal number of heads and tails? The probability is
  \[
    \frac{\binom{2n}{n}}{2^{2n}} = \frac{(2n)!}{(n!)^2 2^{2n}} \sim \frac{1}{\sqrt{n\pi}}
  \]
\end{eg}

\begin{eg}
  Suppose we draw 26 cards from 52. What is the probability of getting 13 reds and 13 blacks? The probability is
  \[
    \frac{\binom{26}{13}\binom{26}{13}}{\binom{52}{26}} = 0.2181.
  \]
\end{eg}
\section{Axioms of probability}
\label{sec:axioms}
\subsection{Axioms and definitions}
So far, we have semi-formally defined some probabilistic notions. However, what we had above was rather restrictive. We were only allowed to have a finite number of possible outcomes, and all outcomes occur with the same probability. However, most things in the real world do not fit these descriptions. For example, we cannot use this to model a coin that gives heads with probability $\pi^{-1}$.

In general, ``probability'' can be defined as follows:
\begin{defi}[Probability space]
  A \emph{probability space} is a triple $(\Omega, \mathcal{F}, \P)$. $\Omega$ is a set called the \emph{sample space}, $\mathcal{F}$ is a collection of subsets of $\Omega$, and $\P: \mathcal{F}\to [0, 1]$ is the \emph{probability measure}.

  $\mathcal{F}$ has to satisfy the following axioms:
  \begin{enumerate}
    \item $\emptyset, \Omega\in \mathcal{F}$.
    \item $A\in \mathcal{F} \Rightarrow A^C\in \mathcal{F}$.
    \item $A_1, A_2, \cdots \in \mathcal{F} \Rightarrow \bigcup_{i = 1}^\infty A_i \in \mathcal{F}$.
  \end{enumerate}
  And $\P$ has to satisfy the following \emph{Kolmogorov axioms}:
  \begin{enumerate}
    \item $0 \leq \P(A) \leq 1 $ for all $A\in \mathcal{F}$
    \item $\P(\Omega) = 1$
    \item For any countable collection of events $A_1, A_2, \cdots$ which are disjoint, i.e.\ $A_i\cap A_j = \emptyset$ for all $i, j$, we have
      \[
        \P\left(\bigcup_i A_i\right) = \sum_i \P(A_i).
      \]
  \end{enumerate}
  Items in $\Omega$ are known as the \emph{outcomes}, items in $\mathcal{F}$ are known as the \emph{events}, and $\P(A)$ is the \emph{probability} of the event $A$.
\end{defi}

If $\Omega$ is finite (or countable), we usually take $\mathcal{F}$ to be all the subsets of $\Omega$, i.e.\ the power set of $\Omega$. However, if $\Omega$ is, say, $\R$, we have to be a bit more careful and only include nice subsets, or else we cannot have a well-defined $\P$.

Often it is not helpful to specify the full function $\P$. Instead, in discrete cases, we just specify the probabilities of each outcome, and use the third axiom to obtain the full $\P$.

\begin{defi}[Probability distribution]
  Let $\Omega = \{\omega_1, \omega_2, \cdots\}$. Choose numbers $p_1, p_2, \cdots $ such that $\sum_{i = 1}^\infty p_i= 1$. Let $p(\omega_i) = p_i$. Then define
  \[
    \P(A) = \sum_{\omega_i\in A} p(\omega_i).
  \]
  This $\P(A)$ satisfies the above axioms, and $p_1, p_2, \cdots$ is the \emph{probability distribution}
\end{defi}

Using the axioms, we can quickly prove a few rather obvious results.
\begin{thm}\leavevmode
  \begin{enumerate}
    \item $\P(\emptyset) = 0$
    \item $\P(A^C) = 1 - \P(A)$
    \item $A\subseteq B \Rightarrow \P(A) \leq \P(B)$
    \item $\P (A\cup B) = \P(A) + \P(B) - \P(A\cap B)$.
  \end{enumerate}
\end{thm}

\begin{proof}\leavevmode
  \begin{enumerate}
    \item $\Omega$ and $\emptyset$ are disjoint. So $\P(\Omega) + \P(\emptyset) = \P(\Omega \cup \emptyset) = \P(\Omega)$. So $\P(\emptyset) = 0$.
    \item $\P(A) + \P(A^C) = \P(\Omega) = 1$ since $A$ and $A^C$ are disjoint.
    \item Write $B = A\cup (B\cap A^C)$. Then $\\P(B) = \P(A) + \P(B\cap A^C) \geq \P(A)$.
    \item $\P(A\cup B) = \P(A) + \P(B\cap A^C)$. We also know that $\P(B) = \P(A\cap B) + \P(B\cap A^C)$. Then the result follows.\qedhere
  \end{enumerate}
\end{proof}

From above, we know that $\P(A\cup B) \leq \P(A) + \P(B)$. So we say that $\P$ is a \emph{subadditive} function. Also, $\P(A\cap B) + \P(A \cup B) \leq \P(A) + \P(B)$ (in fact both sides are equal!). We say $\P$ is \emph{submodular}.

The next theorem is better expressed in terms of \emph{limits}.
\begin{defi}[Limit of events]
  A sequence of events $A_1, A_2, \cdots$ is \emph{increasing} if $A_1 \subseteq A_2 \cdots$. Then we define the \emph{limit} as
  \[
    \lim_{n\to \infty} A_n = \bigcup_{1}^\infty A_n.
  \]
  Similarly, if they are \emph{decreasing}, i.e.\ $A_1\supseteq A_2\cdots$, then
  \[
    \lim_{n\to \infty} A_n = \bigcap_{1}^\infty A_n.
  \]
\end{defi}

\begin{thm}
  If $A_1, A_2, \cdots$ is increasing or decreasing, then
  \[
    \lim_{n\to \infty} \P(A_n) = \P\left(\lim_{n\to \infty} A_n\right).
  \]
\end{thm}

\begin{proof}
  Take $B_1 = A_1$, $B_2 = A_2\setminus A_1$. In general,
  \[
    B_n = A_n\setminus\bigcup_1^{n - 1}A_i.
  \]
  Then
  \[
    \bigcup_1^n B_i = \bigcup_1^n A_i,\quad \bigcup_1^\infty B_i = \bigcup _1^\infty A_i.
  \]
  Then
  \begin{align*}
    \P(\lim A_n) &= \P\left(\bigcup_1^\infty A_i\right)\\
  &= \P\left(\bigcup_1^\infty B_i\right)\\
  &=\sum_1^\infty \P(B_i)\text{ (Axiom III)}\\
  &= \lim_{n \to \infty}\sum_{i = 1}^n \P(B_i)\\
  &= \lim_{n \to \infty} \P\left(\bigcup_1^n A_i\right)\\
  &= \lim_{n \to \infty} \P(A_n).
  \end{align*}
  and the decreasing case is proven similarly (or we can simply apply the above to $A_i^C$).
\end{proof}

\subsection{Inequalities and formulae}
\begin{thm}[Boole's inequality]
  For any $A_1, A_2, \cdots$,
  \[
    \P\left(\bigcup_{i = 1}^\infty A_i\right) \leq \sum_{i = 1}^\infty \P(A_i).
  \]
\end{thm}
This is also known as the ``union bound''.

\begin{proof}
  Our third axiom states a similar formula that only holds for disjoint sets. So we need a (not so) clever trick to make them disjoint. We define
  \begin{align*}
    B_1 &= A_1\\
    B_2 &= A_2\setminus A_1\\
    B_i &= A_i\setminus \bigcup_{k = 1}^{i - 1}A_k.
  \end{align*}
  So we know that
  \[
    \bigcup B_i = \bigcup A_i.
  \]
  But the $B_i$ are disjoint. So our Axiom (iii) gives
  \[
    \P\left(\bigcup_i A_i\right) = \P\left(\bigcup _i B_i\right) = \sum_i \P\left(B_i\right) \leq \sum_i \P\left(A_i\right).
  \]
  Where the last inequality follows from (iii) of the theorem above.
\end{proof}

\begin{eg}
  Suppose we have countably infinite number of biased coins. Let $A_k = [k$th toss head$]$ and $\P(A_k) = p_k$. Suppose $\sum_1^\infty p_k < \infty$. What is the probability that there are infinitely many heads?

  The event ``there is at least one more head after the $i$th coin toss'' is $\bigcup_{k = i}^\infty A_k$. There are infinitely many heads if and only if there are unboundedly many coin tosses, i.e.\ no matter how high $i$ is, there is still at least more more head after the $i$th toss.

  So the probability required is
  \[
    \P\left(\bigcap_{i = 1}^\infty\bigcup _{k = i}^\infty A_k\right) = \lim_{i \to \infty} \P\left(\bigcup_{k = i}^\infty A_k\right) \leq \lim_{i\to \infty}\sum_{k = i}^\infty p_k = 0
  \]
  Therefore $\P($infinite number of heads$) = 0$.
\end{eg}

\begin{eg}[Erd\"os 1947]
  Is it possible to colour a complete $n$-graph (i.e.\ a graph of $n$ vertices with edges between every pair of vertices) red and black such that there is no $k$-vertex complete subgraph with monochrome edges?

  Erd\"os said this is possible if
  \[
    \binom{n}{k} 2^{1 - \binom{k}{2}} < 1.
  \]
  We colour edges randomly, and let $A_i=[i$th subgraph has monochrome edges$]$. Then the probability that at least one subgraph has monochrome edges is
  \[
    \P\left(\bigcup A_i\right) \leq \sum \P(A_i) = \binom{n}{k} 2\cdot 2^{-\binom{k}{2}}.
  \]
  The last expression is obtained since there are $\binom{n}{k}$ ways to choose a subgraph; a monochrome subgraph can be either red or black, thus the multiple of 2; and the probability of getting all red (or black) is $2^{-\binom{k}{2}}$.

  If this probability is less than 1, then there must be a way to colour them in which it is impossible to find a monochrome subgraph, or else the probability is 1. So if $\binom{n}{k} 2^{1 - \binom{k}{2}} < 1$, the colouring is possible.
\end{eg}

\begin{thm}[Inclusion-exclusion formula]
  \begin{align*}
    \P\left(\bigcup_i^n A_i\right) &= \sum_1^n \P(A_i) - \sum_{i_1 < i_2} \P(A_{i_1}\cap A_{j_2}) + \sum_{i_1 < i_2 < i_3}\P(A_{i_1}\cap A_{i_2} \cap A_{i_3}) - \cdots\\
    &+ (-1)^{n - 1} \P(A_1\cap \cdots \cap A_n).
  \end{align*}
\end{thm}

\begin{proof}
  Perform induction on $n$. $n = 2$ is proven above.

  Then
  \[
    \P(A_1\cup A_2\cup \cdots A_n) = \P(A_1) + \P(A_2\cup\cdots\cup A_n) - \P\left(\bigcup_{i = 2}^n (A_1\cap A_i)\right).
  \]
  Then we can apply the induction hypothesis for $n - 1$, and expand the mess. The details are very similar to that in IA Numbers and Sets.
\end{proof}

\begin{eg}
  Let $1, 2, \cdots, n$ be randomly permuted to $\pi(1), \pi(2), \cdots, \pi(n)$. If
  $i \not= \pi(i)$ for all $i$, we say we have a \emph{derangement}.

  Let $A_i = [i \not= \pi(i)]$.

  Then
  \begin{align*}
    \P\left(\bigcup _{i = 1}^n A_i\right) &= \sum_{k} \P(A_k) - \sum_{k_1 < k_2} \P(A_{k_1} \cap A_{k_2}) + \cdots\\
    &= n\cdot \frac{1}{n} - \binom{n}{2}\frac{1}{n}\frac{1}{n - 1} + \binom{n}{3}\frac{1}{n}\frac{1}{n - 1}\frac{1}{n - 2} + \cdots\\
    &= 1 - \frac{1}{2!} + \frac{1}{3!} - \cdots + (-1)^{n - 1}\frac{1}{n!}\\
    &\to e^{-1}
  \end{align*}
  So the probability of derangement is $1 - \P(\bigcup A_k) \approx 1 - e^{-1}\approx 0.632$.
\end{eg}

Recall that, from inclusion exclusion,
\[
  \P(A\cup B\cup C) = \P(A) + \P(B) + \P(C) - \P(AB) - \P(BC) - \P(AC) + \P(ABC),
\]
where $\P(AB)$ is a shorthand for $\P(A\cap B)$. If we only take the first three terms, then we get Boole's inequality
\[
  \P(A\cup B\cup C) \leq \P(A) + \P(B) + \P(C).
\]
In general
\begin{thm}[Bonferroni's inequalities]
  For any events $A_1, A_2, \cdots, A_n$ and $1 \leq r\leq n$, if $r$ is odd, then
  \begin{align*}
    \P\left(\bigcup_{1}^n A_i\right) &\leq \sum_{i_1} \P(A_{i_1}) - \sum_{i_1 < i_2} \P(A_{i_1}A_{i_2}) + \sum_{i_1 < i_2 < i_3} \P(A_{i_1}A_{i_2}A_{i_3}) + \cdots\\
    &+ \sum_{i_1 < i_2 < \cdots < i_r} \P(A_{i_1}A_{i_2}A_{i_3}\cdots A_{i_r}).\\
    \intertext{If $r$ is even, then}
    \P\left(\bigcup_{1}^n A_i\right) &\geq \sum_{i_1} \P(A_{i_1}) - \sum_{i_1 < i_2} \P(A_{i_1}A_{i_2}) + \sum_{i_1 < i_2 < i_3} \P(A_{i_1}A_{i_2}A_{i_3}) + \cdots\\
    &- \sum_{i_1 < i_2 < \cdots < i_r} \P(A_{i_1}A_{i_2}A_{i_3}\cdots A_{i_r}).
  \end{align*}
\end{thm}

\begin{proof}
  Easy induction on $n$.
\end{proof}

\begin{eg}
  Let $\Omega = \{1, 2, \cdots, m\}$ and $1 \leq j, k \leq m$. Write $A_k = \{1, 2, \cdots, k\}$. Then
  \[
    A_k \cap A_j = \{1, 2, \cdots, \min(j, k)\} = A_{\min(j, k)}
  \]
  and
  \[
    A_k \cup A_j = \{1, 2, \cdots, \max(j, k)\} = A_{\max(j, k)}.
  \]
  We also have $\P(A_k) = k/m$.

  Now let $1 \leq x_1, \cdots, x_n \leq m$ be some numbers. Then Bonferroni's inequality says
  \[
    \P\left(\bigcup A_{x_{i}}\right) \geq \sum \P(A_{x_i}) - \sum_{i < j} \P(A_{x_i}\cap A_{x_j}).
  \]
  So
  \[
    \max\{x_1, x_2, \cdots, x_n\} \geq \sum x_i - \sum_{i_1 < i_2} \min\{x_1, x_2\}.
  \]
\end{eg}

\subsection{Independence}
\begin{defi}[Independent events]
  Two events $A$ and $B$ are \emph{independent} if
  \[
    \P(A\cap B) = \P(A)\P(B).
  \]
  Otherwise, they are said to be \emph{dependent}.
\end{defi}
Two events are independent if they are not related to each other. For example, if you roll two dice separately, the outcomes will be independent.

\begin{prop}
  If $A$ and $B$ are independent, then $A$ and $B^C$ are independent.
\end{prop}

\begin{proof}
  \begin{align*}
    \P(A\cap B^C) &= \P(A) - \P(A\cap B)\\
    &= \P(A) - \P(A)\P(B)\\
    &= \P(A)(1 - \P(B))\\
    &= \P(A)\P(B^C)
  \end{align*}
\end{proof}

This definition applies to two events. What does it mean to say that three or more events are independent?
\begin{eg}
  Roll two fair dice. Let $A_1$ and $A_2$ be the event that the first and second die is odd respectively. Let $A_3 = [$sum is odd$]$. The event probabilities are as follows:
  \begin{center}
    \begin{tabular}{cc}
      \toprule
      Event & Probability\\
      \midrule
      $A_1$ & $1/2$\\
      $A_2$ & $1/2$\\
      $A_3$ & $1/2$\\
      $A_1\cap A_2$ & $1/4$\\
      $A_1\cap A_3$ & $1/4$\\
      $A_2\cap A_3$ & $1/4$\\
      $A_1\cap A_2\cap A_3$ & $0$\\
      \bottomrule
    \end{tabular}
  \end{center}
  We see that $A_1$ and $A_2$ are independent, $A_1$ and $A_3$ are independent, and $A_2$ and $A_3$ are independent. However, the collection of all three are \emph{not} independent, since if $A_1$ and $A_2$ are true, then $A_3$ cannot possibly be true.
\end{eg}

From the example above, we see that just because a set of events is pairwise independent does not mean they are independent all together. We define:
\begin{defi}[Independence of multiple events]
  Events $A_1, A_2, \cdots$ are said to be \emph{mutually independent} if
  \[
    \P(A_{i_1}\cap A_{i_2} \cap \cdots \cap A_{i_r}) = \P(A_{i_1})\P(A_{i_2})\cdots \P(A_{i_r})
  \]
  for any $i_1, i_2, \cdots i_r$ and $r \geq 2$.
\end{defi}
\begin{eg}
  Let $A_{ij}$ be the event that $i$ and $j$ roll the same. We roll 4 dice. Then
  \[
    \P(A_{12}\cap A_{13}) = \frac{1}{6}\cdot \frac{1}{6} = \frac{1}{36} = \P(A_{12})\P(A_{13}).
  \]
  But
  \[
    \P(A_{12}\cap A_{13}\cap A_{23}) = \frac{1}{36} \not= \P(A_{12})\P(A_{13})\P(A_{23}).
  \]
  So they are not mutually independent.
\end{eg}

We can also apply this concept to experiments. Suppose we model two independent experiments with $\Omega_1 = \{\alpha_1, \alpha_2, \cdots\}$ and $\Omega_2 = \{\beta_1, \beta_2, \cdots\}$ with probabilities $\P(\alpha_i) = p_i$ and $\P(\beta_i) = q_i$. Further suppose that these two experiments are independent, i.e.
\[
  \P((\alpha_i, \beta_j)) = p_iq_j
\]
for all $i, j$. Then we can have a new sample space $\Omega = \Omega_1\times \Omega_2$.

Now suppose $A\subseteq \Omega_1$ and $B\subseteq \Omega_2$ are results (i.e.\ events) of the two experiments. We can view them as subspaces of $\Omega$ by rewriting them as $A\times \Omega_2$ and $\Omega_1\times B$. Then the probability
\[
  \P(A\cap B) = \sum_{\alpha_i\in A, \beta_i\in B} p_iq_i = \sum_{\alpha_i\in A} p_i\sum_{\beta_i\in B}q_i = \P(A)\P(B).
\]
So we say the two experiments are ``independent'' even though the term usually refers to different events in the same experiment. We can generalize this to $n$ independent experiments, or even countably many infinite experiments.


\subsection{Important discrete distributions}
We're now going to quickly go through a few important discrete probability distributions. By \emph{discrete} we mean the sample space is countable. The sample space is $\Omega = \{\omega_1, \omega_2, \cdots\}$ and $p_i = \P(\{\omega_i\})$.

\begin{defi}[Bernoulli distribution]
  Suppose we toss a coin. $\Omega=\{H, T\}$ and $p\in [0, 1]$. The \emph{Bernoulli distribution}, denoted $B(1, p)$ has
  \[
    \P(H) = p;\quad \P(T) = 1- p.
  \]
\end{defi}

\begin{defi}[Binomial distribution]
  Suppose we toss a coin $n$ times, each with probability $p$ of getting heads. Then
  \[
    \P(HHTT\cdots T) = pp(1 - p)\cdots (1 - p).
  \]
  So
  \[
    \P(\text{two heads}) = \binom{n}{2}p^2(1 - p)^{n -2}.
  \]
  In general,
  \[
    \P(k\text{ heads}) = \binom{n}{k}p^k(1 - p)^{n - k}.
  \]
  We call this the \emph{binomial distribution} and write it as $B(n, p)$.
\end{defi}

\begin{defi}[Geometric distribution]
  Suppose we toss a coin with probability $p$ of getting heads. The probability of having a head after $k$ consecutive tails is
  \[
    p_k = (1- p)^k p
  \]
  This is \emph{geometric distribution}. We say it is \emph{memoryless} because how many tails we've got in the past does not give us any information to how long I'll have to wait until I get a head.
\end{defi}

\begin{defi}[Hypergeometric distribution]
  Suppose we have an urn with $n_1$ red balls and $n_2$ black balls. We choose $n$ balls. The probability that there are $k$ red balls is
  \[
    \P(k\text{ red}) = \frac{\binom{n_1}{k}\binom{n_2}{n - k}}{\binom{n_1 + n_2}{n}}.
  \]
\end{defi}

\begin{defi}[Poisson distribution]
  The \emph{Poisson distribution} denoted $P(\lambda)$ is
  \[
    p_k = \frac{\lambda^k}{k!}e^{-\lambda}
  \]
  for $k\in \N$.
\end{defi}
What is this weird distribution? It is a distribution used to model rare events. Suppose that an event happens at a rate of $\lambda$. We can think of this as there being a lot of trials, say $n$ of them, and each has a probability $\lambda/n$ of succeeding. As we take the limit $n\to \infty$, we obtain the Poisson distribution.

\begin{thm}[Poisson approximation to binomial]
  Suppose $n\to \infty$ and $p\to 0$ such that $np = \lambda$. Then
  \[
    q_k = \binom{n}{k}p^k(1 -p)^{n - k} \to \frac{\lambda^k}{k!}e^{-\lambda}.
  \]
\end{thm}

\begin{proof}
\begin{align*}
  q_k &= \binom{n}{k}p^k(1 - p)^{n - k}\\
  &= \frac{1}{k!} \frac{n(n - 1)\cdots(n - k + 1)}{n^k}(np)^k \left(1 - \frac{np}{n}\right)^{n - k}\\
  &\to \frac{1}{k!}\lambda^ke^{-\lambda}
\end{align*}
since $(1 - a/n)^n \to e^{-a}$.
\end{proof}
\subsection{Conditional probability}
\begin{defi}[Conditional probability]
  Suppose $B$ is an event with $\P(B) > 0$. For any event $A\subseteq \Omega$, the \emph{conditional probability of $A$ given $B$} is
  \[
    \P(A\mid B) = \frac{\P(A\cap B)}{\P(B)}.
  \]
  We interpret as the probability of $A$ happening given that $B$ has happened.
\end{defi}
Note that if $A$ and $B$ are independent, then
\[
  \P(A\mid B) = \frac{\P(A\cap B)}{\P(B)} = \frac{\P(A)\P(B)}{\P(B)} = \P(A).
\]
\begin{eg}
  In a game of poker, let $A_i = [$player $i$ gets royal flush$]$. Then
  \[
    \P(A_1) = 1.539\times 10^{-6}.
  \]
  and
  \[
    \P(A_2\mid A_1) = 1.969\times 10^{-6}.
  \]
  It is significantly bigger, albeit still incredibly tiny. So we say ``good hands attract''.

  If $\P(A\mid B) > \P(A)$, then we say that $B$ attracts $A$. Since
  \[
    \frac{\P(A\cap B)}{\P(B)} > \P(A) \Leftrightarrow \frac{\P(A\cap B)}{\P(A)} > \P(B),
  \]
  $A$ attracts $B$ if and only if $B$ attracts $A$. We can also say $A$ repels $B$ if $A$ attracts $B^C$.
\end{eg}
\begin{thm}\leavevmode
  \begin{enumerate}
    \item $\P(A\cap B) = \P(A\mid B)\P(B)$.
    \item $\P(A\cap B\cap C) = \P(A\mid B\cap C) \P(B\mid C) \P(C)$.
    \item $\P(A\mid B\cap C) = \frac{\P(A\cap B\mid C)}{\P(B\mid C)}$.
    \item The function $\P(\ph \mid B)$ restricted to subsets of $B$ is a probability function (or measure).
  \end{enumerate}
\end{thm}
\begin{proof}
  Proofs of (i), (ii) and (iii) are trivial. So we only prove (iv). To prove this, we have to check the axioms.

  \begin{enumerate}
    \item Let $A\subseteq B$. Then $\P(A\mid B) = \frac{\P(A\cap B)}{\P(B)} \leq 1$.
    \item $\P(B\mid B) = \frac{\P(B)}{\P(B)} = 1$.
    \item Let $A_i$ be disjoint events that are subsets of $B$. Then
      \begin{align*}
        \P\left(\left.\bigcup_i A_i\right\vert B\right) &= \frac{\P(\bigcup_i A_i\cap B)}{\P(B)}\\
        &= \frac{\P\left(\bigcup_i A_i\right)}{\P(B)}\\
        &= \sum \frac{\P(A_i)}{\P(B)}\\
        &=\sum \frac{\P(A_i\cap B)}{\P(B)}\\
        &= \sum \P(A_i \mid B).\qedhere
      \end{align*}%\qedhere
  \end{enumerate}
\end{proof}
\begin{defi}[Partition]
  A \emph{partition of the sample space} is a collection of disjoint events $\{B_i\}_{i = 0}^\infty$ such that $\bigcup_i B_i = \Omega$.
\end{defi}
For example, ``odd'' and ``even'' partition the sample space into two events.

The following result should be clear:
\begin{prop}
  If $B_i$ is a partition of the sample space, and $A$ is any event, then
  \[
    \P(A) = \sum_{i = 1}^\infty \P(A\cap B_i) = \sum_{i = 1}^\infty \P(A\mid B_i) \P(B_i).
  \]
\end{prop}

\begin{eg}
  A fair coin is tossed repeatedly. The gambler gets $+1$ for head, and $-1$ for tail. Continue until he is broke or achieves $\$a$. Let
  \[
    p_x = \P(\text{goes broke}\mid \text{starts with \$}x),
  \]
  and $B_1$ be the event that he gets head on the first toss. Then
  \begin{align*}
    p_x &= \P(B_1)p_{x + 1} + \P(B_1^C) p_{x - 1}\\
    p_x &= \frac{1}{2}p_{x + 1} + \frac{1}{2}p_{x - 1}
  \end{align*}
  We have two boundary conditions $p_0 = 1$, $p_a = 0$. Then solving the recurrence relation, we have
  \[
    p_x = 1 - \frac{x}{a}.
  \]
\end{eg}

\begin{thm}[Bayes' formula]
  Suppose $B_i$ is a partition of the sample space, and $A$ and $B_i$ all have non-zero probability. Then for any $B_i$,
  \[
    \P(B_i \mid A) = \frac{\P(A \mid B_i)\P(B_i)}{\sum_j\P(A \mid B_j)\P(B_j)}.
  \]
  Note that the denominator is simply $\P(A)$ written in a fancy way.
\end{thm}

\begin{eg}[Screen test]
  Suppose we have a screening test that tests whether a patient has a particular disease. We denote positive and negative results as $+$ and $-$ respectively, and $D$ denotes the person having disease. Suppose that the test is not absolutely accurate, and
  \begin{align*}
    \P(+ \mid D) &= 0.98\\
    \P(+ \mid D^C) &= 0.01\\
    \P(D) &= 0.001.
  \end{align*}
  So what is the probability that a person has the disease given that he received a positive result?
  \begin{align*}
    \P(D\mid +) &= \frac{\P(+ \mid D)\P(D)}{\P(+\mid D)\P(D) + \P(+\mid D^C)\P(D^C)}\\
    &= \frac{0.98\cdot 0.001}{0.98\cdot 0.001 + 0.01\cdot 0.999}\\
    &= 0.09
  \end{align*}
  So this test is pretty useless. Even if you get a positive result, since the disease is so rare, it is more likely that you don't have the disease and get a false positive.
\end{eg}

\begin{eg}
  Consider the two following cases:
  \begin{enumerate}
    \item I have 2 children, one of whom is a boy.
    \item I have two children, one of whom is a son born on a Tuesday.
  \end{enumerate}
  What is the probability that both of them are boys?

  \begin{enumerate}
    \item $\P(BB\mid BB\cup BG) = \frac{1/4}{1/4 + 2/4} = \frac{1}{3}$.
    \item Let $B^*$ denote a boy born on a Tuesday, and $B$ a boy not born on a Tuesday. Then
      \begin{align*}
        \P(B^*B^* \cup B^*B\mid BB^* \cup B^*B^*\cup B^*G)&=\frac{\frac{1}{14}\cdot \frac{1}{14} + 2\cdot \frac{1}{14}\cdot\frac{6}{14}}{\frac{1}{14}\cdot \frac{1}{14} + 2\cdot \frac{1}{14}\cdot\frac{6}{14} + 2\cdot \frac{1}{14}\cdot \frac{1}{2}}\\
        &= \frac{13}{27}.
      \end{align*}
  \end{enumerate}
  How can we understand this? It is much easier to have a boy born on a Tuesday if you have two boys than one boy. So if we have the information that a boy is born on a Tuesday, it is now less likely that there is just one boy. In other words, it is more likely that there are two boys.
\end{eg}

\section{Discrete random variables}
With what we've got so far, we are able to answer questions like ``what is the probability of getting a heads?'' or ``what is the probability of getting 10 heads in a row?''. However, we cannot answer questions like ``what do we expect to get on average?''. What does it even mean to take the average of a ``heads'' and a ``tail''?

To make some sense of this notion, we have to assign, to each outcome, a number. For example, if let ``heads'' correspond to $1$ and ``tails'' correspond to $0$. Then on average, we can expect to get $0.5$. This is the idea of a random variable.

\subsection{Discrete random variables}
\begin{defi}[Random variable]
  A \emph{random variable} $X$ taking values in a set $\Omega_X$ is a function $X: \Omega \to \Omega_X$. $\Omega_X$ is usually a set of numbers, e.g.\ $\R$ or $\N$.
\end{defi}
Intuitively, a random variable assigns a ``number'' (or a thing in $\Omega_X$) to each event (e.g.\ assign $6$ to the event ``dice roll gives 6'').

\begin{defi}[Discrete random variables]
  A random variable is \emph{discrete} if $\Omega_X$ is finite or countably infinite.
\end{defi}

\begin{notation}
  Let $T\subseteq \Omega_X$, define
  \[
    \P(X\in T) = \P(\{\omega \in \Omega: X(\omega) \in T\}).
  \]
  i.e.\ the probability that the outcome is in $T$.
\end{notation}
Here, instead of talking about the probability of getting a particular outcome or event, we are concerned with the probability of a random variable taking a particular value. If $\Omega$ is itself countable, then we can write this as
\[
  \P(X \in T) = \sum_{\omega \in \Omega: X(\omega)\in T}p_\omega.
\]
\begin{eg}
  Let $X$ be the value shown by rolling a fair die. Then $\Omega_X = \{1, 2, 3, 4, 5, 6\}$. We know that
  \[
    \P(X = i) = \frac{1}{6}.
  \]
  We call this the discrete uniform distribution.
\end{eg}
\begin{defi}[Discrete uniform distribution]
  A \emph{discrete uniform distribution} is a discrete distribution with finitely many possible outcomes, in which each outcome is equally likely.
\end{defi}

\begin{eg}
  Suppose we roll two dice, and let the values obtained by $X$ and $Y$. Then the sum can be represented by $X + Y$, with
  \[
    \Omega_{X + Y} = \{2, 3, \cdots, 12\}.
  \]
\end{eg}
This shows that we can add random variables to get a new random variable.

\begin{notation}
  We write
  \[
    \P_X(x) = \P(X = x).
  \]
  We can also write $X\sim B(n, p)$ to mean
  \[
    \P(X = r) = \binom{n}{r}p^r(1 - p)^{n - r},
  \]
  and similarly for the other distributions we have come up with before.
\end{notation}

\begin{defi}[Expectation]
  The \emph{expectation} (or \emph{mean}) of a real-valued $X$ is equal to
  \[
    \E[X] = \sum_{\omega\in \Omega}p_\omega X(\omega).
  \]
  provided this is \emph{absolutely convergent}. Otherwise, we say the expectation doesn't exist. Alternatively,
  \begin{align*}
    \E[X] &= \sum_{x\in \Omega_X}\sum_{\omega: X(\omega) = x}p_\omega X(\omega)\\
    &= \sum_{x\in \Omega_X}x\sum_{\omega:X(\omega) = x}p_\omega\\
    &= \sum_{x\in \Omega_X}xP(X = x).
  \end{align*}
  We are sometimes lazy and just write $\E X$.
\end{defi}
This is the ``average'' value of $X$ we expect to get. Note that this definition only holds in the case where the sample space $\Omega$ is countable. If $\Omega$ is continuous (e.g.\ the whole of $\R$), then we have to define the expectation as an integral.

\begin{eg}
  Let $X$ be the sum of the outcomes of two dice. Then
  \[
    \E[X] = 2\cdot \frac{1}{36} + 3\cdot \frac{2}{36} + \cdots + 12\cdot \frac{1}{36} = 7.
  \]
\end{eg}
Note that $\E[X]$ can be non-existent if the sum is not absolutely convergent. However, it is possible for the expected value to be infinite:

\begin{eg}[St. Petersburg paradox]
  Suppose we play a game in which we keep tossing a coin until you get a tail. If you get a tail on the $i$th round, then I pay you $\$2^i$. The expected value is
  \[
    \E[X] = \frac{1}{2}\cdot 2 + \frac{1}{4}\cdot 4 + \frac{1}{8}\cdot 8 + \cdots = \infty.
  \]
  This means that on average, you can expect to get an infinite amount of money! In real life, though, people would hardly be willing to pay $\$20$ to play this game. There are many ways to resolve this paradox, such as taking into account the fact that the host of the game has only finitely many money and thus your real expected gain is much smaller.
\end{eg}

\begin{eg}
  We calculate the expected values of different distributions:
  \begin{enumerate}
    \item Poisson $P(\lambda)$. Let $X\sim P(\lambda)$. Then
      \[
        P_X(r) = \frac{\lambda^r e^{-\lambda}}{r!}.
      \]
      So
      \begin{align*}
        \E[X] &= \sum_{r = 0}^\infty rP(X = r)\\
        &= \sum_{r = 0}^\infty \frac{r \lambda^r e^{-\lambda}}{r!}\\
        &= \sum_{r = 1}^\infty \lambda \frac{\lambda^{r - 1}e^{-\lambda}}{(r - 1)!}\\
        &= \lambda\sum_{r = 0}^\infty\frac{\lambda^r e^{-\lambda}}{r!}\\
        &= \lambda.
      \end{align*}
    \item Let $X\sim B(n, p)$. Then
      \begin{align*}
        \E[X] &= \sum_0^n rP(x = r)\\
        &= \sum_0^n r\binom{n}{r} p^r(1 - p)^{n - r}\\
        &= \sum_0^n r\frac{n!}{r!(n - r)!}p^r (1 - p)^{n - r}\\
        &= np\sum_{r = 1}^n \frac{(n - 1)!}{(r - 1)![(n - 1) - (r - 1)]!}p^{r - 1}(1 - p)^{(n - 1) - (r - 1)}\\
        &= np\sum_{0}^{n - 1}\binom{n - 1}{r}p^r(1 - p)^{n - 1 - r}\\
        &= np.
      \end{align*}
  \end{enumerate}
\end{eg}
Given a random variable $X$, we can create new random variables such as $X + 3$ or $X^2$. Formally, let $f: \R \to \R$ and $X$ be a real-valued random variable. Then $f(X)$ is a new random variable that maps $\omega \mapsto f(X(\omega))$.

\begin{eg}
  if $a, b, c$ are constants, then $a + bX$ and $(X - c)^2$ are random variables, defined as
  \begin{align*}
    (a + bX)(\omega) &= a + bX(\omega)\\
    (X - c)^2(\omega) &= (X(\omega) - c)^2.
  \end{align*}
\end{eg}

\begin{thm}\leavevmode
\begin{enumerate}
  \item If $X \geq 0$, then $\E[X] \geq 0$.
  \item If $X\geq 0$ and $\E[X] = 0$, then $\P(X = 0) = 1$.
  \item If $a$ and $b$ are constants, then $\E[a + bX] = a + b\E[X]$.
  \item If $X$ and $Y$ are random variables, then $\E[X + Y] = \E[X] + \E[Y]$. This is true even if $X$ and $Y$ are not independent.
  \item $\E[X]$ is a constant that minimizes $\E[(X - c)^2]$ over $c$.
\end{enumerate}
\end{thm}
\begin{proof}\leavevmode
  \begin{enumerate}
    \item $X \geq 0$ means that $X(\omega) \geq 0$ for all $\omega$. Then
      \[
        \E[X] = \sum_\omega p_\omega X(\omega) \geq 0.
      \]
    \item If there exists $\omega$ such that $X(\omega) > 0$ and $p_\omega > 0$, then $\E[X] > 0$. So $X(\omega) = 0$ for all $\omega$.
    \item
      \[
        \E[a + bX] = \sum_\omega (a + bX(\omega))p_\omega = a + b\sum_\omega p_\omega= a + b\ \E[X].
      \]
    \item
      \[ \E[X + Y] = \sum_\omega p_\omega[X(\omega) + Y(\omega)] = \sum_\omega p_\omega X(\omega) + \sum_\omega p_\omega Y(\omega) = \E[X] + \E[Y].
      \]
    \item
      \begin{align*}
        \E[(X - c)^2] &= \E[(X - \E[X] + \E[X] - c)^2]\\
        &= \E[(X - \E[X])^2 + 2(\E[X] - c)(X - \E[X]) + (\E[X] - c)^2]\\
        &= \E(X - \E[X])^2 + 0 + (\E[X] - c)^2.
      \end{align*}
      This is clearly minimized when $c = \E[X]$. Note that we obtained the zero in the middle because $\E[X - \E[X]] = \E[X] - \E[X] = 0$.\qedhere
  \end{enumerate}
\end{proof}
An easy generalization of (iv) above is
\begin{thm}
  For any random variables $X_1, X_2, \cdots X_n$, for which the following expectations exist,
  \[
    \E\left[\sum_{i = 1}^n X_i\right] = \sum_{i = 1}^n \E[X_i].
  \]
\end{thm}

\begin{proof}
  \[
    \sum_\omega p(\omega)[X_1(\omega) + \cdots + X_n(\omega)] = \sum_\omega p(\omega)X_1(\omega) + \cdots + \sum_\omega p(\omega) X_n(\omega).\qedhere
  \]
\end{proof}

\begin{defi}[Variance and standard deviation]
  The \emph{variance} of a random variable $X$ is defined as
  \[
    \var(X) = \E[(X - \E[X])^2].
  \]
  The \emph{standard deviation} is the square root of the variance, $\sqrt{\var(X)}$.
\end{defi}
This is a measure of how ``dispersed'' the random variable $X$ is. If we have a low variance, then the value of $X$ is very likely to be close to $\E[X]$.

\begin{thm}\leavevmode
  \begin{enumerate}
    \item $\var X \geq 0$. If $\var X = 0$, then $\P(X = \E[X]) = 1$.
    \item $\var (a + bX) = b^2 \var(X)$. This can be proved by expanding the definition and using the linearity of the expected value.
    \item $\var(X) = \E[X^2] - \E[X]^2$, also proven by expanding the definition.
  \end{enumerate}
\end{thm}

\begin{eg}[Binomial distribution]
  Let $X\sim B(n, p)$ be a binomial distribution. Then $\E[X] = np$. We also have
  \begin{align*}
    \E[X(X - 1)] &= \sum_{r = 0}^n r(r - 1)\frac{n!}{r!(n - r)!} p^r(1 - p)^{n - r}\\
    &= n(n - 1)p^2\sum_{r = 2}^n \binom{n - 2}{r - 2} p^{r - 2}(1 - p)^{(n - 2) - (r - 2)}\\
    &= n(n - 1)p^2.
  \end{align*}
  The sum goes to $1$ since it is the sum of all probabilities of a binomial $N(n - 2, p)$
  So $\E[X^2] = n(n - 1)p^2 + \E[X] = n(n - 1)p^2 + np$. So
  \[
    \var(X) = \E[X^2] - (\E[X])^2 = np(1 - p) = npq.
  \]
\end{eg}

\begin{eg}[Poisson distribution]
  If $X\sim P(\lambda)$, then $\E[X] = \lambda$, and $\var(X) = \lambda$, since $P(\lambda)$ is $B(n, p)$ with $n\to \infty, p \to 0, np \to \lambda$.
\end{eg}

\begin{eg}[Geometric distribution]
  Suppose $\P(X = r) = q^r p$ for $r= 0, 1, 2, \cdots$. Then
  \begin{align*}
    \E[X] &= \sum_{0}^\infty rpq^r \\
    &= pq\sum_{0}^\infty rq^{r - 1}\\
    &= pq\sum_{0}^\infty \frac{\d }{\d q}q^r\\
    &= pq\frac{\d }{\d q}\sum_{0}^\infty q^r\\
    &= pq\frac{\d }{\d q}\frac{1}{1 - q}\\
    &= \frac{pq}{(1 - q)^2}\\
    &= \frac{q}{p}.
  \end{align*}
  Then
  \begin{align*}
    \E[X(X - 1)] &= \sum_{0}^\infty r(r - 1)pq^r\\
    &= pq^2 \sum_0^\infty r(r - 1)q^{r - 2}\\
    &= pq^2\frac{\d ^2}{\d q^2}\frac{1}{1 - q}\\
    &= \frac{2pq^2}{(1 - q)^3}
  \end{align*}
  So the variance is
  \[
    \var(X) = \frac{2pq^2}{(1 - q)^3}+ \frac{q}{p} - \frac{q^2}{p^2} = \frac{q}{p^2}.
  \]
\end{eg}

\begin{defi}[Indicator function]
  The \emph{indicator function} or \emph{indicator variable} $I[A]$ (or $I_A$) of an event $A\subseteq \Omega$ is
  \[
    I[A](\omega) =
    \begin{cases}
      1 & \omega\in A\\
      0 & \omega\not\in A
    \end{cases}
  \]
\end{defi}
This indicator random variable is not interesting by itself. However, it is a rather useful tool to prove results.

It has the following properties:
\begin{prop}\leavevmode
  \begin{itemize}
    \item $\E[I[A]] = \sum_\omega p(\omega) I[A](\omega) = \P(A)$.
    \item $I[A^C] = 1 - I[A]$.
    \item $I[A\cap B] = I[A]I[B]$.
    \item $I[A\cup B] = I[A] + I[B] - I[A]I[B]$.
    \item $I[A]^2 = I[A]$.
  \end{itemize}
\end{prop}
These are easy to prove from definition. In particular, the last property comes from the fact that $I[A]$ is either $0$ and $1$, and $0^2 = 0, 1^2 = 1$.

\begin{eg}
  Let $2n$ people ($n$ husbands and $n$ wives, with $n > 2$) sit alternate man-woman around the table randomly. Let $N$ be the number of couples sitting next to each other.

  Let $A_i = [i$th couple sits together$]$. Then
  \[
    N = \sum_{i = 1}^n I[A_i].
  \]
  Then
  \[
    \E[N] = \E\left[\sum I[A_i]\right] = \sum_{1}^n \E\big[I[A_i]\big] = n\E\big[I[A_1]\big] = n\P(A_i) = n\cdot \frac{2}{n} = 2.
  \]
  We also have
  \begin{align*}
    \E[N^2] &= \E\left[\left(\sum I[A_i]\right)^2\right]\\
    &= \E\left[\sum_i I[A_i]^2 + 2\sum_{i < j}I[A_i]I[A_j]\right]\\
    &= n\E\big[I[A_i]\big] + n(n - 1)\E\big[I[A_1]I[A_2]\big]
  \end{align*}
  We have $\E[I[A_1]I[A_2]] = \P(A_1\cap A_2) = \frac{2}{n}\left(\frac{1}{n - 1}\frac{1}{n - 1} + \frac{n - 2}{n - 1}\frac{2}{n - 1}\right)$. Plugging in, we ultimately obtain $\var(N) = \frac{2(n- 2)}{n - 1}$.

  In fact, as $n\to \infty$, $N\sim P(2)$.
\end{eg}

We can use these to prove the inclusion-exclusion formula:
\begin{thm}[Inclusion-exclusion formula]
  \begin{align*}
    \P\left(\bigcup_i^n A_i\right) &= \sum_1^n \P(A_i) - \sum_{i_1 < i_2} \P(A_{i_1}\cap A_{j_2}) + \sum_{i_1 < i_2 < i_3}\P(A_{i_1}\cap A_{i_2} \cap A_{i_3}) - \cdots\\
    &+ (-1)^{n - 1} \P(A_1\cap \cdots \cap A_n).
  \end{align*}
\end{thm}

\begin{proof}
  Let $I_j$ be the indicator function for $A_j$. Write
  \[
    S_r = \sum_{i_1 < i_2 < \cdots < i_r}I_{i_1}I_{i_2}\cdots I_{i_r},
  \]
  and
  \[
    s_r = \E[S_r] = \sum_{i_1 < \cdots < i_r}\P(A_{i_1}\cap \cdots \cap A_{i_r}).
  \]
  Then
  \[
    1 - \prod_{j = 1}^n(1 - I_j) = S_1 - S_2 + S_3 \cdots + (-1)^{n - 1}S_n.
  \]
  So
  \[
    \P\left(\bigcup_1^n A_j\right) = \E\left[1 - \prod_1^n(1 - I_j)\right] = s_1 - s_2 + s_3 - \cdots + (-1)^{n - 1}s_n.\qedhere
  \]
\end{proof}

We can extend the idea of independence to random variables. Two random variables are independent if the value of the first does not affect the value of the second.

\begin{defi}[Independent random variables]
  Let $X_1, X_2, \cdots, X_n$ be discrete random variables. They are \emph{independent} iff for any $x_1, x_2, \cdots, x_n$,
  \[
    \P(X_1 = x_1, \cdots, X_n = x_n) = \P(X_1 = x_1)\cdots \P(X_n = x_n).
  \]
\end{defi}

\begin{thm}
  If $X_1, \cdots, X_n$ are independent random variables, and $f_1, \cdots, f_n$ are functions $\R \to \R$, then $f_1(X_1), \cdots, f_n(X_n)$ are independent random variables.
\end{thm}

\begin{proof}
  Note that given a particular $y_i$, there can be many different $x_i$ for which $f_i(x_i) = y_i$. When finding $\P(f_i(x_i)= y_i)$, we need to sum over all $x_i$ such that $f_i(x_i) = f_i$. Then
  \begin{align*}
    \P(f_1(X_1) = y_1, \cdots f_n(X_n) = y_n) &= \sum_{\substack{x_1: f_1(x_1) = y_1\\\cdot\vspace{-2pt}\\\cdot\\x_n: f_n(x_n) = y_n}} \P(X_1 = x_1, \cdots, X_n = x_n)\\
    &= \sum_{\substack{x_1: f_1(x_1) = y_1\\\cdot\vspace{-2pt}\\\cdot\\x_n: f_n(x_n) = y_n}} \prod_{i = 1}^n \P(X_i = x_i)\\
    &= \prod_{i = 1}^n \sum_{x_i:f_i(x_i) = y_i} \P(X_i = x_i)\\
    &= \prod_{i = 1}^n \P(f_i(x_i) = y_i).
  \end{align*}
  Note that the switch from the second to third line is valid since they both expand to the same mess.
\end{proof}

\begin{thm}
  If $X_1, \cdots, X_n$ are independent random variables and all the following expectations exists, then
  \[
    \E\left[\prod X_i\right] = \prod \E[X_i].
  \]
\end{thm}

\begin{proof}
  Write $R_i$ for the range of $X_i$. Then
  \begin{align*}
    \E\left[\prod_1^n X_i\right] &= \sum_{x_1\in R_1}\cdots \sum_{x_n\in R_n}x_1x_2\cdots x_n\times \P(X_1 = x_1, \cdots, X_n = x_n)\\
    &= \prod_{i = 1}^n \sum_{x_i \in R_i}x_i\P(X_i = x_i)\\
    &= \prod_{i = 1}^n \E[X_i].\qedhere
  \end{align*}
\end{proof}

\begin{cor}
  Let $X_1,\cdots X_n$ be independent random variables, and $f_1, f_2, \cdots f_n$ are functions $\R\to \R$. Then
  \[
    \E\left[\prod f_i(x_i)\right] = \prod \E[f_i(x_i)].
  \]
\end{cor}

\begin{thm}
  If $X_1, X_2, \cdots X_n$ are independent random variables, then
  \[
    \var\left(\sum X_i\right) = \sum \var (X_i).
  \]
\end{thm}

\begin{proof}
  \begin{align*}
    \var\left(\sum X_i\right) &= \E\left[\left(\sum X_i\right)^2\right] - \left(\E\left[\sum X_i\right]\right)^2\\
    &= \E\left[\sum X_i^2 + \sum_{i \not =j} X_iX_j\right] - \left(\sum\E [X_i]\right)^2\\
    &= \sum \E[X_i^2] + \sum_{i\not= j}\E[X_i]\E[X_j] - \sum(\E[X_i])^2 - \sum_{i\not= j}\E[X_i]\E[X_j]\\
    &= \sum \E[X_i^2] - (\E[X_i])^2.\qedhere
  \end{align*}
\end{proof}

\begin{cor}
 Let $X_1, X_2, \cdots X_n$ be independent identically distributed random variables (iid rvs). Then
 \[
   \var\left(\frac{1}{n}\sum X_i\right) = \frac{1}{n}\var (X_1).
 \]
\end{cor}

\begin{proof}
  \begin{align*}
    \var\left(\frac{1}{n}\sum X_i\right) &= \frac{1}{n^2}\var\left(\sum X_i\right)\\
    &= \frac{1}{n^2}\sum \var(X_i)\\
    &= \frac{1}{n^2} n \var(X_1)\\
    &= \frac{1}{n}\var (X_1)
  \end{align*}
\end{proof}
This result is important in statistics. This means that if we want to reduce the variance of our experimental results, then we can repeat the experiment many times (corresponding to a large $n$), and then the sample average will have a small variance.

\begin{eg}
  Let $X_i$ be iid $B(1, p)$, i.e.\ $\P(1) = p$ and $\P(0) = 1 - p$. Then $Y = X_1 + X_2 + \cdots + X_n \sim B(n, p)$.

  Since $\var(X_i) = \E[X_i^2] - (\E[X_i])^2 = p - p^2 = p(1 - p)$, we have $\var (Y) = np(1 - p)$.
\end{eg}

\begin{eg}
  Suppose we have two rods of unknown lengths $a, b$. We can measure the lengths, but is not accurate. Let $A$ and $B$ be the measured value. Suppose
  \[
    \E[A] = a,\quad \var(A) = \sigma^2
  \]
  \[
    \E[B] = b, \quad \var(B) = \sigma^2.
  \]
  We can measure it more accurately by measuring $X = A + B$ and $Y = A - B$. Then we estimate $a$ and $b$ by
  \[
    \hat{a} = \frac{X + Y}{2},\; \hat{b} = \frac{X - Y}{2}.
  \]
  Then $\E[\hat{a}] = a$ and $\E[\hat{b}] = b$, i.e.\ they are unbiased. Also
  \[
    \var (\hat{a}) = \frac{1}{4}\var(X + Y) = \frac{1}{4}2\sigma^2 = \frac{1}{2}\sigma^2,
  \]
  and similarly for $b$. So we can measure it more accurately by measuring the sticks together instead of separately.
\end{eg}

\subsection{Inequalities}
Here we prove a lot of different inequalities which may be useful for certain calculations. In particular, Chebyshev's inequality will allow us to prove the weak law of large numbers.

\begin{defi}[Convex function]
  A function $f: (a, b) \to \R$ is \emph{convex} if for all $x_1, x_2\in (a, b)$ and $\lambda_1, \lambda_2 \geq 0$ such that $\lambda_1 + \lambda_2 = 1$,\[
    \lambda_1f(x_1) + \lambda_2 f(x_2) \geq f(\lambda_1x_1 + \lambda_2 x_2).
  \]
  It is \emph{strictly convex} if the inequality above is strict (except when $x_1 = x_2$ or $\lambda_1$ or $\lambda_2 = 0$).
  \begin{center}
    \begin{tikzpicture}
      \draw(-2, 4) -- (-2, 0) -- (2, 0) -- (2, 4);
      \draw (-1.3, 0.1) -- (-1.3, -0.1) node [below] {$x_1$};
      \draw (1.3, 0.1) -- (1.3, -0.1) node [below] {$x_2$};
      \draw (-1.7, 2) parabola bend (-.2, 1) (1.7, 3.3);
      \draw [dashed] (-1.3, 0) -- (-1.3, 1.53) node [circ] {};
      \draw [dashed] (1.3, 0) -- (1.3, 2.42) node [circ] {};
      \draw (-1.3, 1.53) -- (1.3, 2.42);
      \draw [dashed] (0, 0) node [below] {\tiny $\lambda_1x_1 + \lambda_2x_2$} -- (0, 1.975) node [above] {\tiny$\lambda_1 f(x_1) + \lambda_2 f(x_2)\quad\quad\quad\quad\quad\quad$} node [circ] {};
    \end{tikzpicture}
  \end{center}
  A function is \emph{concave} if $-f$ is convex.
\end{defi}

A useful criterion for convexity is
\begin{prop}
  If $f$ is differentiable and $f''(x) \geq 0$ for all $x\in (a, b)$, then it is convex. It is strictly convex if $f''(x) > 0$.
\end{prop}

\begin{thm}[Jensen's inequality]
  If $f:(a, b) \to \R$ is convex, then
  \[
    \sum_{i = 1}^n p_i f(x_i) \geq f\left(\sum_{i = 1}^n p_ix_i\right)
  \]
  for all $p_1, p_2, \cdots, p_n$ such that $p_i \geq 0$ and $\sum p_i = 1$, and $x_i \in (a, b)$.

  This says that $\E[f(X)] \geq f(\E[X])$ (where $\P(X=x_i) = p_i$).

  If $f$ is strictly convex, then equalities hold only if all $x_i$ are equal, i.e.\ $X$ takes only one possible value.
\end{thm}

\begin{proof}
  Induct on $n$. It is true for $n = 2$ by the definition of convexity. Then
  \begin{align*}
    f(p_1x_1 + \cdots + p_nx_n) &= f\left(p_1x_1 + (p_2 + \cdots + p_n)\frac{p_2x_2 + \cdots + l_nx_n}{p_2 + \cdots + p_n}\right)\\
    &\leq p_1f(x_1) + (p_2 + \cdots p_n)f\left(\frac{p_2x_2 + \cdots + p_n x_n}{p_2 + \cdots + p_n }\right).\\
    &\leq p_1f(x_1) + (p_2 + \cdots + p_n)\left[\frac{p_2}{(\;)}f(x_2) + \cdots + \frac{p_n}{(\;)}f(x_n)\right]\\
    &= p_1f(x_1) + \cdots + p_n(x_n).
  \end{align*}
  where the $(\;)$ is $p_2 + \cdots + p_n$.

  Strictly convex case is proved with $\leq$ replaced by $<$ by definition of strict convexity.
\end{proof}

\begin{cor}[AM-GM inequality]
  Given $x_1, \cdots, x_n$ positive reals, then
  \[
    \left(\prod x_i\right)^{1/n} \leq \frac{1}{n}\sum x_i.
  \]
\end{cor}

\begin{proof}
  Take $f(x) = -\log x$. This is convex since its second derivative is $x^{-2} > 0$.

  Take $\P(x = x_i) = 1/n$. Then
  \[
    \E[f(x)] = \frac{1}{n}\sum -\log x_i = -\log \text{GM}
  \]
  and
  \[
    f(\E[x]) = -\log \frac{1}{n}\sum x_i = -\log\text{AM}
  \]
  Since $f(\E[x]) \leq \E[f(x)]$, AM $\geq$ GM. Since $-\log x$ is strictly convex, AM $=$ GM only if all $x_i$ are equal.
\end{proof}

\begin{thm}[Cauchy-Schwarz inequality]
  For any two random variables $X, Y$,
  \[
    (\E[XY])^2 \leq \E[X^2]\E[Y^2].
  \]
\end{thm}

\begin{proof}
  If $Y = 0$, then both sides are $0$. Otherwise, $\E[Y^2] > 0$. Let
  \[
    w = X - Y\cdot \frac{\E[XY]}{\E[Y^2]}.
  \]
  Then
  \begin{align*}
    \E[w^2] &= \E\left[X^2 - 2XY\frac{\E[XY]}{\E[Y^2]} + Y^2\frac{(\E[XY])^2}{(\E[Y^2])^2}\right]\\
    &= \E[X^2] - 2\frac{(\E[XY])^2}{\E[Y^2]} + \frac{(\E[XY])^2}{\E[Y^2]}\\
    &= \E[X^2] - \frac{(\E[XY])^2}{\E[Y^2]}
  \end{align*}
  Since $\E[w^2] \geq 0$, the Cauchy-Schwarz inequality follows.
\end{proof}

\begin{thm}[Markov inequality]
  If $X$ is a random variable with $\E|X| < \infty$ and $\varepsilon > 0$, then
  \[
    \P(|X| \geq \varepsilon) \leq \frac{\E|X|}{\varepsilon}.
  \]
\end{thm}

\begin{proof}
  We make use of the indicator function. We have
  \[
    I[|X|\geq \varepsilon] \leq \frac{|X|}{\varepsilon}.
  \]
  This is proved by exhaustion: if $|X| \geq \varepsilon$, then LHS = 1 and RHS $\geq$ 1; If $|X| < \varepsilon$, then LHS = 0 and RHS is non-negative.

  Take the expected value to obtain
  \[
    \P(|X| \geq \varepsilon) \leq \frac{\E |X|}{\varepsilon}.\qedhere
  \]
\end{proof}

Similarly, we have
\begin{thm}[Chebyshev inequality]
  If $X$ is a random variable with $\E[X^2] < \infty$ and $\varepsilon > 0$, then
  \[
    \P(|X| \geq \varepsilon) \leq \frac{\E[X^2]}{\varepsilon^2}.
  \]
\end{thm}

\begin{proof}
  Again, we have
  \[
    I[\{|X|\geq \varepsilon\}] \leq \frac{x^2}{\varepsilon^2}.
  \]
  Then take the expected value and the result follows.
\end{proof}
Note that these are really powerful results, since they do not make \emph{any} assumptions about the distribution of $X$. On the other hand, if we know something about the distribution, we can often get a larger bound.

An important corollary is that if $\mu = \E[X]$, then
\[
  \P(|X - \mu| \geq \varepsilon) \leq \frac{\E[(X - \mu)^2]}{\varepsilon^2} = \frac{\var X}{\varepsilon^2}
\]

\subsection{Weak law of large numbers}
\begin{thm}[Weak law of large numbers]
  Let $X_1, X_2, \cdots$ be iid random variables, with mean $\mu$ and $\var \sigma^2$.

  Let $S_n = \sum_{i = 1}^n X_i$.

  Then for all $\varepsilon > 0$,
  \[
    \P\left(\left|\frac{S_n}{n} - \mu\right| \geq \varepsilon\right) \to 0
  \]
  as $n\to \infty$.

  We say, $\frac{S_n}{n}$ tends to $\mu$ (in probability), or
  \[
    \frac{S_n}{n}\to_p \mu.
  \]
\end{thm}

\begin{proof}
  By Chebyshev,
  \begin{align*}
    \P\left(\left|\frac{S_n}{n} - \mu\right| \geq \varepsilon\right)&\leq \frac{\E\left(\frac{S_n}{n} - \mu\right)^2}{\varepsilon^2}\\
    &= \frac{1}{n^2}\frac{\E(S_n -n\mu)^2}{\varepsilon^2}\\
    &= \frac{1}{n^2\varepsilon^2}\var (S_n)\\
    &= \frac{n}{n^2\varepsilon^2}\var (X_1)\\
    &= \frac{\sigma^2}{n\varepsilon^2} \to 0\qedhere
  \end{align*}
\end{proof}

Note that we cannot relax the ``independent'' condition. For example, if $X_1 = X_2 = X_3 = \cdots = $ 1 or 0, each with probability $1/2$. Then $S_n/n \not\to 1/2$ since it is either 1 or 0.

\begin{eg}
  Suppose we toss a coin with probability $p$ of heads. Then
  \[
    \frac{S_n}{n} = \frac{\text{number of heads}}{\text{number of tosses}}.
  \]
  Since $\E[X_i] = p$, then the weak law of large number tells us that
  \[
    \frac{S_n}{n} \to_p p.
  \]
  This means that as we toss more and more coins, the proportion of heads will tend towards $p$.
\end{eg}

Since we called the above the \emph{weak} law, we also have the \emph{strong} law, which is a stronger statement.
\begin{thm}[Strong law of large numbers]
  \[
    \P\left(\frac{S_n}{n}\to \mu\text{ as }n\to \infty\right) = 1.
  \]
  We say
  \[
    \frac{S_n}{n}\to_{\mathrm{as}} \mu,
  \]
  where ``as'' means ``almost surely''.
\end{thm}
It can be shown that the weak law follows from the strong law, but not the other way round. The proof is left for Part II because it is too hard.


\subsection{Multiple random variables}
If we have two random variables, we can study the relationship between them.

\begin{defi}[Covariance]
  Given two random variables $X, Y$, the \emph{covariance} is
  \[
    \cov(X, Y) = \E[(X - \E[X])(Y - \E[Y])].
  \]
\end{defi}

\begin{prop}\leavevmode
  \begin{enumerate}
    \item $\cov(X, c) = 0$ for constant $c$.
    \item $\cov(X + c, Y) = \cov (X, Y)$.
    \item $\cov (X, Y) = \cov (Y, X)$.
    \item $\cov (X, Y) = \E[XY] - \E[X]\E[Y]$.
    \item $\cov (X, X) = \var (X)$.
    \item $\var (X + Y) = \var (X) + \var (Y) + 2\cov (X, Y)$.
    \item If $X$, $Y$ are independent, $\cov(X, Y) = 0$.
  \end{enumerate}
\end{prop}
These are all trivial to prove and proof is omitted.

It is important to note that $\cov(X, Y) = 0$ does not imply $X$ and $Y$ are independent.
\begin{eg}\leavevmode
  \begin{itemize}
    \item Let $(X, Y) = (2, 0), (-1, -1)$ or $(-1, 1)$ with equal probabilities of $1/3$. These are not independent since $Y = 0\Rightarrow X = 2$.

      However, $\cov (X, Y) = \E[XY] - \E[X]\E[Y] = 0 - 0\cdot 0 = 0$.

    \item If we randomly pick a point on the unit circle, and let the coordinates be $(X, Y)$, then $\E[X] = \E[Y] = \E[XY] = 0$ by symmetry. So $\cov(X, Y) = 0$ but $X$ and $Y$ are clearly not independent (they have to satisfy $x^2 + y^2 = 1$).
  \end{itemize}
\end{eg}

The covariance is not that useful in measuring how well two variables correlate. For one, the covariance can (potentially) have dimensions, which means that the numerical value of the covariance can depend on what units we are using. Also, the magnitude of the covariance depends largely on the variance of $X$ and $Y$ themselves. To solve these problems, we define

\begin{defi}[Correlation coefficient]
  The \emph{correlation coefficient} of $X$ and $Y$ is
  \[
    \corr(X, Y) = \frac{\cov (X, Y)}{\sqrt{\var (X)\var (Y)}}.
  \]
\end{defi}

\begin{prop}
  $|\corr(X, Y)| \leq 1$.
\end{prop}

\begin{proof}
  Apply Cauchy-Schwarz to $X - \E[X]$ and $Y - \E[Y]$.
\end{proof}

Again, zero correlation does not necessarily imply independence.

Alternatively, apart from finding a fixed covariance or correlation number, we can see how the distribution of $X$ depends on $Y$. Given two random variables $X, Y$, $\P(X = x, Y = y)$ is known as the \emph{joint distribution}. From this joint distribution, we can retrieve the probabilities $\P(X = x)$ and $\P(Y = y)$. We can also consider different conditional expectations.

\begin{defi}[Conditional distribution]
  Let $X$ and $Y$ be random variables (in general not independent) with joint distribution $\P(X = x, Y = y)$. Then the \emph{marginal distribution} (or simply \emph{distribution}) of $X$ is
  \[
    \P(X = x) = \sum_{y\in \Omega_y}\P(X = x, Y = y).
  \]
  The \emph{conditional distribution} of $X$ given $Y$ is
  \[
    \P(X = x\mid Y = y) = \frac{\P(X = x, Y = y)}{\P(Y = y)}.
  \]
  The \emph{conditional expectation} of $X$ given $Y$ is
  \[
    \E[X\mid Y = y] = \sum_{x\in \Omega_X}x\P(X = x\mid Y = y).
  \]
  We can view $\E[X\mid Y]$ as a random variable in $Y$: given a value of $Y$, we return the expectation of $X$.
\end{defi}

\begin{eg}
  Consider a dice roll. Let $Y = 1$ denote an even roll and $Y = 0$ denote an odd roll. Let $X$ be the value of the roll. Then $\E[X\mid Y] = 3 + Y$, ie 4 if even, 3 if odd.
\end{eg}

\begin{eg}
  Let $X_1, \cdots, X_n$ be iid $B(1, p)$. Let $Y = X_1 + \cdots + X_n$. Then
  \begin{align*}
    \P(X_1 = 1\mid Y = r) &= \frac{\P(X_1 = 1, \sum_{2}^n X_i = r - 1)}{\P(Y = r)}\\
    &= \frac{p\binom{n - 1}{r - 1}p^{r - 1}(1 - p)^{(n - 1) - (r - 1)}}{\binom{n}{r} p^r (1 - p)^{n - 1}} = \frac{r}{n}.
  \end{align*}
  So
  \[
    \E[X_1\mid Y] = 1 \cdot \frac{r}{n} + 0\left(1 - \frac{r}{n}\right) = \frac{r}{n} = \frac{Y}{n}.
  \]
  Note that this is a random variable!
\end{eg}

\begin{thm}
  If $X$ and $Y$ are independent, then
  \[
    \E[X\mid Y] = \E[X]
  \]
\end{thm}

\begin{proof}
  \begin{align*}
    \E[X\mid Y = y] &= \sum_{x}x\P(X = x\mid Y = y)\\
    &= \sum_x x\P(X = x)\\
    &= \E[X]\qedhere
  \end{align*}
\end{proof}

We know that the expected value of a dice roll given it is even is 4, and the expected value given it is odd is 3. Since it is equally likely to be even or odd, the expected value of the dice roll is 3.5. This is formally captured by
\begin{thm}[Tower property of conditional expectation]
  \[
    \E_Y[\E_X[X\mid Y]] = \E_X[X],
  \]
  where the subscripts indicate what variable the expectation is taken over.
\end{thm}

\begin{proof}
  \begin{align*}
    \E_Y[\E_X[X\mid Y]] &= \sum_y \P(Y = y)\E[X\mid Y = y]\\
    &= \sum _y \P(Y = y) \sum_{x}x\P(X = x\mid Y = y)\\
    &= \sum_x\sum_y x\P(X = x, Y = y)\\
    &= \sum _x x\sum_y \P(X = x, Y = y)\\
    &= \sum_x x\P(X = x)\\
    &= \E[X].\qedhere
  \end{align*}
\end{proof}
This is also called the law of total expectation. We can also state it as: suppose $A_1, A_2, \cdots, A_n$ is a partition of $\Omega$. Then
\[
  \E[X] = \sum_{i: \P(A_i) > 0}\E[X\mid A_i]\P(A_i).
\]

\subsection{Probability generating functions}
Consider a random variable $X$, taking values $0, 1, 2, \cdots$. Let $p_r = \P(X = r)$.
\begin{defi}[Probability generating function (pgf)]
  The \emph{probability generating function (pgf)} of $X$ is
  \[
    p(z) = \E[z^X] = \sum_{r = 0}^\infty \P(X = r)z^r = p_0 + p_1z + p_2z^2\cdots = \sum_0^\infty p_rz^r.
  \]
  This is a power series (or polynomial), and converges if $|z| \leq 1$, since
  \[
    |p(z)| \leq \sum_r p_r |z^r| \leq \sum_r p_r = 1.
  \]
  We sometimes write as $p_X(z)$ to indicate what the random variable.
\end{defi}
This definition might seem a bit out of the blue. However, it turns out to be a rather useful algebraic tool that can concisely summarize information about the probability distribution.

\begin{eg}
  Consider a fair di.e.\ Then $p_r = 1/6$ for $r = 1, \cdots, 6$. So
  \[
    p(z) = \E[z^X] = \frac{1}{6}(z + z^2 + \cdots + z^6) = \frac{1}{6}z\left(\frac{1 - z^6}{1 - z}\right).
  \]
\end{eg}

\begin{thm}
  The distribution of $X$ is uniquely determined by its probability generating function.
\end{thm}

\begin{proof}
  By definition, $p_0 = p(0)$, $p_1 = p'(0)$ etc. (where $p'$ is the derivative of $p$). In general,
  \[
    \left. \frac{\d ^i}{\d z^i}p(z) \right|_{z = 0} = i! p_i.
  \]
  So we can recover $(p_0, p_1, \cdots)$ from $p(z)$.
\end{proof}

\begin{thm}[Abel's lemma]
  \[
    \E[X] = \lim_{z\to 1}p'(z).
  \]
  If $p'(z)$ is continuous, then simply $\E[X] = p'(1)$.
\end{thm}
Note that this theorem is trivial if $p'(1)$ exists, as long as we know that we can differentiate power series term by term. What is important here is that even if $p'(1)$ doesn't exist, we can still take the limit and obtain the expected value, e.g.\ when $\E[X] = \infty$.

\begin{proof}
  For $z < 1$, we have
  \[
    p'(z) = \sum_1^\infty rp_r z^{r - 1} \leq \sum_1^\infty rp_r = \E[X].
  \]
  So we must have
  \[
    \lim_{z\to 1}p'(z) \leq \E[X].
  \]
  On the other hand, for any $\varepsilon$, if we pick $N$ large, then
  \[
    \sum_{1}^N rp_r \geq \E[X] - \varepsilon.
  \]
  So
  \[
    \E[X] - \varepsilon \leq \sum_1^N rp_r = \lim_{z\to 1}\sum _1^N rp_r z^{r - 1}\leq \lim_{z \to 1} \sum_1^\infty rp_r z^{r - 1} = \lim_{z\to 1} p' (z).
  \]
  So $\E[X] \leq \lim\limits_{z \to 1}p'(z)$. So the result follows
\end{proof}

\begin{thm}
  \[
    \E[X(X - 1)] = \lim_{z \to 1}p''(z).
  \]
\end{thm}
\begin{proof}
  Same as above.
\end{proof}

\begin{eg}
  Consider the Poisson distribution. Then
  \[
    p_r = \P(X = r) = \frac{1}{r!}\lambda^r e^{-\lambda}.
  \]
  Then
  \[
    p(z) = \E[z^X] = \sum_0^\infty z^r \frac{1}{r!}\lambda^r e^{-\lambda} = e^{\lambda z}e^{-\lambda} = e^{\lambda(z - 1)}.
  \]
  We can have a sanity check: $p(1) = 1$, which makes sense, since $p(1)$ is the sum of probabilities.

  We have
  \[
    \E[X]= \left.\frac{\d }{\d z}e^{\lambda(z - 1)}\right|_{z = 1} = \lambda,
  \]
  and
  \[
    \E[X(X - 1)] = \left.\frac{\d ^2}{\d x^2}e^{\lambda(z - 1)}\right|_{z = 1} = \lambda^2
  \]
  So
  \[
    \var(X) = \E[X^2] - \E[X]^2 = \lambda^2 + \lambda - \lambda^2 = \lambda.
  \]
\end{eg}

\begin{thm}
  Suppose $X_1, X_2, \cdots, X_n$ are independent random variables with pgfs $p_1, p_2, \cdots, p_n$. Then the pgf of $X_1 + X_2 + \cdots + X_n$ is $p_1(z)p_2(z)\cdots p_n(z)$.
\end{thm}

\begin{proof}
  \[
    \E[z^{X_1 + \cdots + X_n}] = \E[z^{X_1}\cdots z^{X_n}] = \E[z^{X_1}]\cdots\E[z^{X_n}] = p_1(z)\cdots p_n(z).\qedhere
  \]
\end{proof}

\begin{eg}
  Let $X\sim B(n, p)$. Then
  \[
    p(z) = \sum_{r = 0}^{n} \P(X = r)z^r = \sum \binom{n}{r} p^r(1 - p)^{n - r}z^r = (pz + (1 - p))^n = (pz + q)^n.
  \]
  So $p(z)$ is the product of $n$ copies of $pz + q$. But $pz + q$ is the pgf of $Y\sim B(1, p)$.

  This shows that $X = Y_1 + Y_2 + \cdots + Y_n$ (which we already knew), i.e.\ a binomial distribution is the sum of Bernoulli trials.
\end{eg}

\begin{eg}
  If $X$ and $Y$ are independent Poisson random variables with parameters $\lambda, \mu$ respectively, then
  \[
    \E[t^{X + Y}] = \E[t^X]\E[t^Y] = e^{\lambda(t - 1)}e^{\mu(t - 1)} = e^{(\lambda + \mu)(t - 1)}
  \]
  So $X + Y\sim \P(\lambda + \mu)$.

  We can also do it directly:
  \[
    \P(X + Y = r) = \sum_{i = 0}^r \P(X = i, Y = r - i) = \sum_{i = 0}^r \P(X = i)\P(X = r - i),
  \]
  but is much more complicated.
\end{eg}

We can use pgf-like functions to obtain some combinatorial results.
\begin{eg}
  Suppose we want to tile a $2\times n$ bathroom by $2\times 1$ tiles. One way to do it is
  \begin{center}
    \begin{tikzpicture}
     \draw (0, 0) rectangle (6, 2);
     \draw (1, 0) rectangle (3, 1);
     \draw (1, 1) rectangle (3, 2);
     \draw (4, 0) rectangle (6, 1);
     \draw (4, 1) rectangle (6, 2);
    \end{tikzpicture}
  \end{center}
  We can do it recursively: suppose there are $f_n$ ways to tile a $2\times n$ grid. Then if we start tiling, the first tile is either vertical, in which we have $f_{n - 1}$ ways to tile the remaining ones; or the first tile is horizontal, in which we have $f_{n - 2}$ ways to tile the remaining. So
  \[
    f_n = f_{n - 1} + f_{n - 2},
  \]
  which is simply the Fibonacci sequence, with $f_0 = f_1 = 1$.

  Let
  \[
    F(z) = \sum_{n = 0}^\infty f_nz^n.
  \]
  Then from our recurrence relation, we obtain
  \[
    f_nz^n = f_{n - 1}z^n + f_{n - 2}z^n.
  \]
  So
  \[
    \sum_{n = 2}^\infty f_n z^n = \sum_{n = 2}^{\infty} f_{n - 1}z^n + \sum_{n = 2}^\infty f_{n - 2}z^n.
  \]
  Since $f_0 = f_1 = 1$, we have
  \[
    F(z) - f_0 - zf_1 = z(F(z) - f_0) + z^2F(z).
  \]
  Thus $F(z) = (1 - z - z^2)^{-1}$. If we write
  \[
    \alpha_1 = \frac{1}{2}(1 + \sqrt{5}),\quad \alpha_2 = \frac{1}{2}(1 - \sqrt{5}).
  \]
  then we have
  \begin{align*}
    F(z) &= (1 - z - z^2)^{-1}\\
    &= \frac{1}{(1 - \alpha_1 z)(1 - \alpha_2 z)}\\
    &= \frac{1}{\alpha_1 - \alpha_2}\left(\frac{\alpha_1}{1 - \alpha_1 z} - \frac{\alpha_2}{1 - \alpha_2 z}\right)\\
    &= \frac{1}{\alpha_1 - \alpha_2}\left(\alpha_1 \sum_{n = 0}^\infty \alpha_1^nz^n - \alpha_2\sum_{n = 0}^\infty \alpha_2^n z^n\right).
  \end{align*}
  So
  \[
    f_n = \frac{\alpha_1^{n + 1} - \alpha_2^{n + 1}}{\alpha_1 - \alpha_2}.
  \]
\end{eg}

\begin{eg}
  A \emph{Dyck word} is a string of brackets that match, such as (), or ((())()).

  There is only one Dyck word of length 2, (). There are 2 of length 4, (()) and ()(). Similarly, there are 5 Dyck words of length 5.

  Let $C_n$ be the number of Dyck words of length $2n$. We can split each Dyck word into $(w_1)w_2$, where $w_1$ and $w_2$ are Dyck words. Since the lengths of $w_1$ and $w_2$ must sum up to $2(n -1)$,
  \[
    C_{n + 1} = \sum_{i = 0}^n C_iC_{n - i}.\tag{$*$}
  \]
  We again use pgf-like functions: let
  \[
    c(x) = \sum_{n = 0}^\infty C_n x^n.
  \]
  From $(*)$, we can show that
  \[
    c(x) = 1 + xc(x)^2.
  \]
  We can solve to show that
  \[
    c(x) = \frac{1 - \sqrt{1 - 4x}}{2x} = \sum_0^\infty \binom{2n}{n}\frac{x^n}{n + 1},
  \]
  noting that $C_0 = 1$. Then
  \[
    C_n = \frac{1}{n + 1}\binom{2n}{n}.
  \]
\end{eg}

\subsubsection*{Sums with a random number of terms}
A useful application of generating functions is the sum with a random number of random terms. For example, an insurance company may receive a random number of claims, each demanding a random amount of money. Then we have a sum of a random number of terms. This can be answered using probability generating functions.

\begin{eg}
Let $X_1, X_2, \cdots, X_n$ be iid with pgf $p(z) = \E[z^X]$. Let $N$ be a random variable independent of $X_i$ with pgf $h(z)$. What is the pgf of $S = X_1 + \cdots + X_N$?
\begin{align*}
  \E[z^{S}] &= \E[z^{X_1 + \cdots + X_N}]\\
  &= \E_N[\underbrace{\E_{X_i}[z^{X_1 + \ldots + X_N}\mid N]}_{\text{assuming fixed }N}]\\
  &= \sum_{n = 0}^\infty \P(N = n) \E[z^{X_1 + X_2 + \cdots + X_n}]\\
  &= \sum_{n = 0}^\infty \P(N = n) \E[z^{X_1}] \E[z^{X_2}]\cdots\E[z^{X_n}]\\
  &= \sum_{n = 0}^\infty \P(N = n) (\E[z^{X_1}])^n\\
  &= \sum_{n = 0}^\infty \P(N = n) p(z)^n \\
  &= h(p(z))
\end{align*}
since $h(x) = \sum_{n = 0}^\infty \P(N = n) x^n$.

So
\begin{align*}
  \E[S] &= \left.\frac{\d }{\d z}h(p(z))\right|_{z = 1}\\
  &= h'(p(1))p'(1)\\
  &= \E[N]\E[X_1]
\end{align*}
To calculate the variance, use the fact that
\[
  \E[S(S - 1)] = \left.\frac{\d ^2}{\d z^2}h(p(z))\right|_{z = 1}.
\]
Then we can find that
\[
  \var (S) = \E[N]\var(X_1) + \E[X_1^2]\var(N).
\]
\end{eg}


\section{Interesting problems}
Here we are going to study two rather important and interesting probabilistic processes --- branching processes and random walks. Solutions to these will typically involve the use of probability generating functions.

\subsection{Branching processes}
Branching processes are used to model population growth by reproduction. At the beginning, there is only one individual. At each iteration, the individual produces a random number of offsprings. In the next iteration, each offspring will individually independently reproduce randomly according to the same distribution. We will ask questions such as the expected number of individuals in a particular generation and the probability of going extinct.

Consider $X_0, X_1, \cdots$, where $X_n$ is the number of individuals in the $n$th generation. We assume the following:
\begin{enumerate}
  \item $X_0 = 1$
  \item Each individual lives for unit time and produces $k$ offspring with probability $p_k$.
  \item Suppose all offspring behave independently. Then
    \[
      X_{n + 1} = Y_1^n + Y_2^n + \cdots + Y_{X_n}^n,
    \]
    where $Y_i^n$ are iid random variables, which is the same as $X_1$ (the superscript denotes the generation).
\end{enumerate}
It is useful to consider the pgf of a branching process. Let $F(z)$ be the pgf of $Y_i^n$. Then
\[
  F(z) = E[z^{Y_i^n}] = E[z^{X_1}] = \sum_{k = 0}^\infty p_k z^k.
\]
Define
\[
  F_n(z) = E[z^{X_n}].
\]
The main theorem of branching processes here is
\begin{thm}
  \[
    F_{n + 1}(z) = F_n(F(z)) = F(F(F(\cdots F(z) \cdots )))) = F(F_n(z)).
  \]
\end{thm}

\begin{proof}
  \begin{align*}
    F_{n + 1}(z) &= \E[z^{X_{n + 1}}]\\
    &= \E[\E[z^{X_{n + 1}}\mid X_n]]\\
    &= \sum_{k = 0}^\infty \P(X_n = k)\E[z^{X_{n + 1}}\mid X_n = k]\\
    &= \sum_{k = 0}^\infty \P(X_n = k)\E[z^{Y_1^n + \cdots + Y_k^n}\mid X_n = k]\\
    &= \sum_{k = 0}^\infty \P(X_n = k)\E[z^{Y_1}]\E[z^{Y_2}]\cdots\E[z^{Y_n}]\\
    &= \sum_{k = 0}^\infty \P(X_n = k)(\E[z^{X_1}])^k\\
    &= \sum_{k = 0}\P(X_n = k)F(z)^k\\
    &= F_n(F(z))
  \end{align*}
\end{proof}

\begin{thm}
  Suppose
  \[
    \E[X_1] = \sum k p_k = \mu
  \]
  and
  \[
    \var (X_1) = \E[(X - \mu)^2] = \sum (k - \mu)^2 p_k < \infty.
  \]
  Then
  \[
    \E[X_n] = \mu^n,\quad \var X_n = \sigma^2\mu^{n - 1}(1 + \mu + \mu^2 + \cdots + \mu^{n - 1}).
  \]
\end{thm}
\begin{proof}
\begin{align*}
  \E[X_n] &= \E[\E[X_n\mid X_{n - 1}]]\\
  &= \E[\mu X_{n - 1}]\\
  &= \mu \E[X_{n - 1}]
\end{align*}
Then by induction, $\E[X_n] = \mu^n$ (since $X_0 = 1$).

To calculate the variance, note that
\[
  \var (X_n) = \E[X_n^2] - (\E[X_n])^2
\]
and hence
\[
  \E[X_n^2] = \var(X_n) + (\E[X])^2
\]
We then calculate
\begin{align*}
  \E[X_n^2] &= \E[\E[X_n^2\mid X_{n - 1}]]\\
  &= \E[\var(X_n) + (\E[X_n])^2 \mid X_{n - 1}]\\
  &= \E[X_{n - 1}\var (X_1) + (\mu X_{n - 1})^2]\\
  &= \E[X_{n - 1}\sigma^2 + (\mu X_{n - 1})^2]\\
  &= \sigma^2 \mu^{n - 1} + \mu^2 \E[X_{n - 1}^2].
\end{align*}
So
\begin{align*}
  \var X_n &= \E[X_n^2] - (\E[X_n])^2 \\
  &= \mu^2 \E[X_{n -1}^2] + \sigma^2 \mu^{n - 1} - \mu^2 (\E[X_{n - 1}])^2\\
  &= \mu^2(\E[X_{n - 1}^2] - \E[X_{n - 1}]^2) + \sigma^2\mu^{n - 1}\\
  &= \mu^2 \var (X_{n - 1}) + \sigma^2 \mu^{n -1 }\\
  &= \mu^4 \var(X_{n - 2}) + \sigma^2(\mu^{n - 1} + \mu^n)\\
  &= \cdots\\
  &= \mu^{2(n - 1)}\var (X_1) + \sigma^2 (\mu^{n - 1} + \mu^{n} + \cdots + \mu^{2n - 3})\\
  &= \sigma^2\mu^{n - 1}(1 + \mu + \cdots + \mu^{n - 1}).
\end{align*}
Of course, we can also obtain this using the probability generating function as well.
\end{proof}
\subsubsection*{Extinction probability}
Let $A_n$ be the event $X_n = 0$, ie extinction has occurred by the $n$th generation. Let $q$ be the probability that extinction eventually occurs. Let
\[
  A = \bigcup_{n = 1}^\infty A_n = [\text{extinction eventually occurs}].
\]
Since $A_1 \subseteq A_2 \subseteq A_3 \subseteq \cdots$, we know that
\[
  q = \P(A) = \lim_{n\to \infty} \P(A_n) = \lim_{n \to \infty} \P(X_n = 0).
\]
But
\[
  \P(X_n = 0) = F_n(0),
\]
since $F_n(0) = \sum \P(X_n = k)z^k$. So
\[
  F(q) = F\left(\lim_{n\to \infty} F_n(0)\right) = \lim_{n\to \infty} F(F_n(0)) = \lim_{n\to \infty}F_{n + 1}(0) = q.
\]
So
\[
  F(q) = q.
\]
Alternatively, using the law of total probability
\[
  q = \sum_k \P(X_1 = k) \P(\text{extinction}\mid X_1 = k) = \sum_k p_k q^k = F(q),
\]
where the second equality comes from the fact that for the whole population to go extinct, each individual population must go extinct.

This means that to find the probability of extinction, we need to find a fixed point of $F$. However, if there are many fixed points, which should we pick?

\begin{thm}
  The probability of extinction $q$ is the smallest root to the equation $q = F(q)$. Write $\mu = \E[X_1]$. Then if $\mu \leq 1$, then $q = 1$; if $\mu > 1$, then $q < 1$.
\end{thm}

\begin{proof}
  To show that it is the smallest root, let $\alpha$ be the smallest root. Then note that $0 \leq \alpha \Rightarrow F(0) \leq F(\alpha) = \alpha$ since $F$ is increasing (proof: write the function out!). Hence $F(F(0)) \leq \alpha$. Continuing inductively, $F_n(0) \leq \alpha$ for all $n$. So
  \[
    q = \lim_{n \to \infty}F_n(0) \leq \alpha.
  \]
  So $q = \alpha$.

  To show that $q = 1$ when $\mu \leq 1$, we show that $q = 1$ is the only root. We know that $F'(z), F''(z) \geq 0$ for $z\in (0, 1)$ (proof: write it out again!). So $F$ is increasing and convex. Since $F'(1) = \mu \leq 1$, it must approach $(1, 1)$ from above the $F = z$ line. So it must look like this:
  \begin{center}
    \begin{tikzpicture}
      \draw (0, 0) -- (4, 0) node [right] {$z$};
      \draw (0, 0) -- (0, 4) node [above] {$F(z)$};
      \draw [dashed] (0, 0) -- (4, 4);
      \draw (0, 2.3) parabola (4, 4);
    \end{tikzpicture}
  \end{center}
  So $z = 1$ is the only root.
\end{proof}

\subsection{Random walk and gambler's ruin}
Here we'll study random walks, using gambler's ruin as an example.

\begin{defi}[Random walk]
  Let $X_1, \cdots, X_n$ be iid random variables such that $X_n = +1$ with probability $p$, and $-1$ with probability $1 - p$. Let $S_n = S_0 + X_1 + \cdots + X_n$. Then $(S_0, S_1, \cdots, S_n)$ is a \emph{1-dimensional random walk}.

  If $p = q = \frac{1}{2}$, we say it is a \emph{symmetric random walk}.
\end{defi}

\begin{eg}
  A gambler starts with $\$z$, with $z < a$, and plays a game in which he wins $\$1$ or loses $\$ 1$ at each turn with probabilities $p$ and $q$ respectively. What are
  \[
    p_z = \P(\text{random walk hits $a$ before } 0 \mid \text{starts at } z),
  \]
  and
  \[
    q_z = \P(\text{random walk hits 0 before }a \mid \text{starts at }z)?
  \]
  He either wins his first game, with probability $p$, or loses with probability $q$. So
  \[
    p_z = qp_{z - 1} + pp_{z + 1},
  \]
  for $0 < z < a$, and $p_0 = 0, p_a = 1$.

  Try $p_z = t^z$. Then
  \[
    pt^2 - t + q = (pt - q)(t - 1) = 0,
  \]
  noting that $p = 1 - q$. If $p \not = q$, then
  \[
    p_z = A1^z + B\left(\frac{q}{p}\right)^z.
  \]
  Since $p_z = 0$, we get $A = -B$. Since $p_a = 1$, we obtain
  \[
    p_z = \frac{1 - (p/q)^z}{1 - (p/q)^a}.
  \]
  If $p = q$, then $p_z = A + Bz = z/a$.

  Similarly, (or perform the substitutions $p\mapsto q$, $q\mapsto p$ and $z \mapsto a - z$)
  \[
    q_z = \frac{(p/q)^z - (q/p)^a}{1 - (p/q)^a}
  \]
  if $p\not = q$, and
  \[
    q_z = \frac{a - z}{z}
  \]
  if $p = q$. Since $p_z + q_z = 1$, we know that the game will eventually end.

  What if $a\to \infty$? What is the probability of going bankrupt?
  \begin{align*}
    \P(\text{path hits 0 ever}) &= \P\left(\bigcup_{a = z+1}^\infty [\text{path hits 0 before }a]\right)\\
    &= \lim_{a\to \infty}\P(\text{path hits 0 before }a)\\
    &= \lim_{a\to \infty}q_z\\
    &= \begin{cases}
      (p/q)^z & p > q\\
      1 & p \leq q.
    \end{cases}
  \end{align*}
  So if the odds are against you (i.e.\ the probability of losing is greater than the probability of winning), then no matter how small the difference is, you are bound to going bankrupt eventually.
\end{eg}
\subsubsection*{Duration of the game}
Let $D_z =$ expected time until the random walk hits $0$ or $a$, starting from $z$. We first show that this is bounded. We know that there is one simple way to hit $0$ or $a$: get $+1$ or $-1$ for $a$ times in a row. This happens with probability $p^a + q^a$, and takes $a$ steps. So even if this were the only way to hit $0$ or $a$, the expected duration would be $\frac{a}{p^a + q^a}$. So we must have
\[
  D_z \leq \frac{a}{p^a + q^a}
\]
This is a \emph{very} crude bound, but it is sufficient to show that it is bounded, and we can meaningfully apply formulas to this finite quantity.

We have
\begin{align*}
  D_z &= \E[\text{duration}]\\
  &= \E[\E[\text{duration}\mid X_1]]\\
  &= p\E[\text{duration}\mid X_1 = 1] + q\E[\text{duration}\mid X_1 = -1]\\
  &= p(1 + D_{z + 1}) + q(1 - D_{z - 1})
\end{align*}
So
\[
  D_z = 1 + pD_{z + 1} + qD_{z - 1},
\]
subject to $D_0 = D_a = 0$.

We first find a particular solution by trying $D_z = Cz$. Then
\[
  Cz = 1 + pC(z + 1) = qC(z - 1).
\]
So
\[
  C = \frac{1}{q - p},
\]
for $p \not = q$. The we find the complementary solution. Try $D_z = t^z$.
\[
  pt^2 - t + q = 0,
\]
which has roots $1$ and $q/p$. So the general solution is
\[
  D_z = A + B\left(\frac{q}{p}\right)^z + \frac{z}{q - p}.
\]
Putting in the boundary conditions,
\[
  D_z = \frac{z}{q - p} - \frac{a}{q - p} \cdot \frac{1 - (q/p)^z}{1 - (q/p)^a}.
\]
If $p = q$, then to find the particular solution, we have to try
\[
  D_z = Cz^2.
\]
Then we find $C = -1$. So
\[
  D_z = -z^2 + A + Bz.
\]
Then the boundary conditions give
\[
  D_z = z(a - z).
\]
\subsubsection*{Using generating functions}
We can use generating functions to solve the problem as well.

Let $U_{z,n} = \P(\text{random walk absorbed at 0 at }n\mid \text{start in }z)$.

We have the following conditions: $U_{0, 0} = 1$; $U_{z, 0} = 0$ for $0 < z \leq a$; $U_{0, n} = U_{a, n} = 0$ for $n > 0$.

We define a pgf-like function.
\[
  U_z(s) = \sum_{n = 0 }^\infty U_{z, n}s^n.
\]
We know that
\[
  U_{z, n + 1} = p U_{z + 1, n} + qU_{z - 1, n}.
\]
Multiply by $s^{n + 1}$ and sum on $n = 0, 1, \cdots$. Then
\[
  U_z(s) = psU_{z + 1}(s) + qsU_{z - 1}(s).
\]
We try $U_z(s) = [\lambda(s)]^z$. Then
\[
  \lambda(s) = ps\lambda(s)^2 + qs.
\]
Then
\[
  \lambda_1(s), \lambda_2(s) = \frac{1 \pm \sqrt{1 - 4pqs^2}}{2ps}.
\]
So
\[
  U_z(s) = A(s) \lambda_1(s)^z + B(s)\lambda_2(s)^z.
\]
Since $U_0(s) = 1$ and $U_a(s) = 0$, we know that
\[
  A(s) + B(s) = 1
\]
and
\[
  A(s)\lambda_1(s)^a + B(s)\lambda_2(s)^a = 0.
\]
Then we find that
\[
  U_z(s) = \frac{\lambda_1(s)^a\lambda_2(s)^z - \lambda_2(s)^a\lambda_1(s)^z}{\lambda_1(s)^a - \lambda_2(s)^a}.
\]
Since $\lambda_1(s)\lambda_2(s) = \frac{q}{p}$, we can ``simplify'' this to obtain
\[
  U_z(s) = \left(\frac{q}{p}\right)^z \cdot \frac{\lambda_1(s)^{a - z} - \lambda_2(s)^{a - z}}{\lambda_1(s)^a - \lambda_2(s)^a}.
\]
We see that $U_z(1) = q_z$. We can apply the same method to find the generating function for absorption at $a$, say $V_z(s)$. Then the generating function for the duration is $U_z + V_z$. Hence the expected duration is $D_z = U'_z(1) + V'_z(1).$
\section{Continuous random variables}
\subsection{Continuous random variables}
So far, we have only looked at the case where the outcomes $\Omega$ are discrete. Consider an experiment where we throw a needle randomly onto the ground and record the angle it makes with a fixed horizontal. Then our sample space is $\Omega = \{\omega\in \R: 0 \leq \omega < 2\pi\}$. Then we have
\[
  \P(\omega \in [0, \theta]) = \frac{\theta}{2\pi}, \quad 0 \leq \theta \leq 2\pi.
\]
With continuous distributions, we can no longer talk about the probability of getting a particular number, since this is always zero. For example, we will almost never get an angle of \emph{exactly} $0.42$ radians.

Instead, we can only meaningfully talk about the probability of $X$ falling into a particular range. To capture the distribution of $X$, we want to define a function $f$ such that for each $x$ and small $\delta x$, the probability of $X$ lying in $[x, x + \delta x]$ is given by $f(x) \delta x + o(\delta x)$. This $f$ is known as the \emph{probability density function}. Integrating this, we know that the probability of $X\in [a, b]$ is $\int_a^b f(x)\;\d x$. We take this as the definition of $f$.

\begin{defi}[Continuous random variable]
  A random variable $X : \Omega \to \R$ is \emph{continuous} if there is a function $f : \R \to \R_{\geq 0}$ such that
  \[
    \P(a \leq X \leq b) = \int_a^b f(x) \;\d x.
  \]
  We call $f$ the \emph{probability density function}, which satisfies
  \begin{itemize}
    \item $f \geq 0$
    \item $\int_{-\infty}^\infty f(x) = 1$.
  \end{itemize}
\end{defi}
Note that $\P(X = a) = 0$ since it is $\int_a^a f(x)\;\d x$. Then we also have
\[
  \P\left(\bigcup_{a\in \Q} [X = a]\right) = 0,
\]
since it is a countable union of probability 0 events (and axiom 3 states that the probability of the countable union is the sum of probabilities, i.e.\ 0).

\begin{defi}[Cumulative distribution function]
  The \emph{cumulative distribution function} (or simply \emph{distribution function}) of a random variable $X$ (discrete, continuous, or neither) is
  \[
    F(x) = \P(X \leq x).
  \]
\end{defi}
We can see that $F(x)$ is increasing and $F(x) \to 1$ as $x\to \infty$.

In the case of continuous random variables, we have
\[
  F(x) = \int_\infty^x f(z) \;\d z.
\]
Then $F$ is continuous and differentiable. In general, $F'(x) = f(x)$ whenever $F$ is differentiable.

The name of \emph{continuous} random variables comes from the fact that $F(x)$ is (absolutely) continuous.

The cdf of a continuous random variable might look like this:
\begin{center}
  \begin{tikzpicture}
    \draw [->] (0, 0) -- (3, 0);
    \draw [->] (0, 0) -- (0, 2) node [above] {$\P$};
    \draw (0, 0) parabola (2.5, 1.5);
  \end{tikzpicture}
\end{center}
The cdf of a discrete random variable might look like this:
\begin{center}
  \begin{tikzpicture}
    \draw [->] (0, 0) -- (3, 0);
    \draw [->] (0, 0) -- (0, 2) node [above] {$\P$};
    \draw (0, 0) -- (1, 0) -- (1, 0.5) -- (2, 0.5) -- (2, 1) -- (3, 1) -- (3, 1.5);
  \end{tikzpicture}
\end{center}
The cdf of a random variable that is neither discrete nor continuous might look like this:
\begin{center}
  \begin{tikzpicture}
    \draw [->] (0, 0) -- (3, 0);
    \draw [->] (0, 0) -- (0, 2) node [above] {$\P$};
    \draw (0, 0) parabola (1, 1);
    \draw (1, 1) -- (2.5, 1) -- (2.5, 2);
    \node [circ] at (2.5, 2) {};
  \end{tikzpicture}
\end{center}
Note that we always have
\[
  \P(a < x \leq b) = F(b) - F(a).
\]
This will be equal to $\int_a^b f(x)\;\d x$ in the case of continuous random variables.

\begin{defi}[Uniform distribution]
  The \emph{uniform distribution} on $[a, b]$ has pdf
  \[
    f(x) = \frac{1}{b - a}.
  \]
  Then
  \[
    F(x) = \int_a^x f(z)\;\d z = \frac{x - a}{b - a}
  \]
  for $a \leq x \leq b$.

  If $X$ follows a uniform distribution on $[a, b]$, we write $X\sim U[a, b]$.
\end{defi}
\begin{defi}[Exponential random variable]
  The \emph{exponential random variable with parameter $\lambda$} has pdf
  \[
    f(x) = \lambda e^{-\lambda x}
  \]
  and
  \[
    F(x) = 1 - e^{-\lambda x}
  \]
  for $x \geq 0$.

  We write $X \sim \mathcal{E}(\lambda)$.
\end{defi}

Then
\[
  \P(a \leq x \leq b) = \int_a^b f(z) \;\d z = e^{-\lambda a} - e^{-\lambda b}.
\]
\begin{prop}
  The exponential random variable is \emph{memoryless}, i.e.
  \[
    \P(X \geq x + z \mid X \geq x) = \P(X \geq z).
  \]
  This means that, say if $X$ measures the lifetime of a light bulb, knowing it has already lasted for 3 hours does not give any information about how much longer it will last.
\end{prop}
Recall that the geometric random variable is the discrete memoryless random variable.

\begin{proof}
  \begin{align*}
    \P(X \geq x + z \mid X \geq x) &= \frac{\P(X \geq x + z)}{\P(X \geq x)}\\
    &= \frac{\int_{x + z}^\infty f(u)\;\d u}{\int_x^\infty f(u)\;\d u}\\
    &= \frac{e^{-\lambda(x + z)}}{e^{-\lambda x}}\\
    &= e^{-\lambda z}\\
    &= \P(X \geq z).\qedhere
  \end{align*}
\end{proof}

Similarly, given that, you have lived for $x$ days, what is the probability of dying within the next $\delta x$ days?
\begin{align*}
  \P(X \leq x + \delta x \mid X \geq x) &= \frac{\P(x \leq X \leq x + \delta x)}{\P(X \geq x)}\\
  &= \frac{\lambda e^{-\lambda x} \delta x}{e^{-\lambda x}}\\
  &= \lambda \delta x.
\end{align*}
So it is independent of how old you currently are, assuming your survival follows an exponential distribution.

In general, we can define the hazard rate to be
\[
  h(x) = \frac{f(x)}{1 - F(x)}.
\]
Then
\[
  \P(x \leq X \leq x + \delta x \mid X \geq x) = \frac{\P(x \leq X \leq x + \delta x)}{\P(X \geq x)} = \frac{\delta x f(x)}{1 - F(x)} = \delta x\cdot h(x).
\]
In the case of exponential distribution, $h(x)$ is constant and equal to $\lambda$.

Similar to discrete variables, we can define the expected value and the variance. However, we will (obviously) have to replace the sum with an integral.
\begin{defi}[Expectation]
  The \emph{expectation} (or \emph{mean}) of a continuous random variable is
  \[
    \E[X] = \int_{-\infty}^\infty xf(x) \;\d x,
  \]
  provided not both $\int_0^\infty xf(x)\;\d x$ and $\int_{-\infty}^0 xf(x) \;\d x$ are infinite.
\end{defi}

\begin{thm}
  If $X$ is a continuous random variable, then
  \[
    \E[X] = \int_0^\infty \P(X \geq x)\;\d x - \int_0^\infty \P(X \leq -x)\;\d x.
  \]
\end{thm}
This result is more intuitive in the discrete case:
\[
  \sum_{x = 0}^\infty x \P(X = x) = \sum_{x = 0}^\infty \sum_{y = x + 1}^\infty \P(X = y) = \sum_{x = 0}^\infty \P(X > x),
\]
where the first equality holds because on both sides, we have $x$ copies of $\P(X = x)$ in the sum.

\begin{proof}
  \begin{align*}
    \int_0^\infty \P(X \geq x)\;\d x &= \int_0^\infty \int_x^\infty f(y)\;\d y\;\d x\\
    &= \int_0^\infty \int_0^\infty I[y\geq x]f(y)\;\d y\;\d x\\
    &= \int_0^\infty \left(\int_0^\infty I[x\leq y]\;\d x\right) f(y)\;\d y\\
    &= \int_0^\infty yf(y)\;\d y.
  \end{align*}
  We can similarly show that $\int_0^\infty \P(X \leq -x)\;\d x = -\int_{-\infty}^0 yf(y)\;\d y$.
\end{proof}

\begin{eg}
  Suppose $X\sim \mathcal{E}(\lambda)$. Then
  \[
    \P(X \geq x) = \int_x^{\infty} \lambda e^{-\lambda t}\;\d t = e^{-\lambda x}.
  \]
  So
  \[
    \E[X] = \int_0^\infty e^{-\lambda x}\;\d x = \frac{1}{\lambda}.\qedhere
  \]
\end{eg}

\begin{defi}[Variance]
  The \emph{variance} of a continuous random variable is
  \[
    \var (X) = \E[(X - \E[X])^2] = \E[X^2] - (E[X])^2.
  \]
\end{defi}
So we have
\[
  \var (X) = \int_{-\infty}^\infty x^2 f(x) \;\d x - \left(\int_{-\infty}^\infty xf(x)\;\d x\right)^2.
\]
\begin{eg}
  Let $X\sim U[a, b]$. Then
  \[
    \E[X] = \int_a^b x\frac{1}{b - a}\;d x = \frac{a + b}{2}.
  \]
  So
  \begin{align*}
    \var (X) &= \int_a^b x^2\frac{1}{b - a}\;\d x - (\E[X])^2\\
    &= \frac{1}{12}(b - a)^2.
  \end{align*}
\end{eg}

Apart from the mean, or expected value, we can also have other notions of ``average values''.
\begin{defi}[Mode and median]
  Given a pdf $f(x)$, we call $\hat x$ a \emph{mode} if
  \[
    f(\hat x) \geq f(x)
  \]
  for all $x$. Note that a distribution can have many modes. For example, in the uniform distribution, all $x$ are modes.

  We say it is a median if
  \[
    \int_{-\infty}^{\hat x}f(x)\;\d x = \frac{1}{2} = \int_{\hat x}^\infty f(x)\;\d x.
  \]
  For a discrete random variable, the median is $\hat x$ such that
  \[
    \P(X \leq \hat x) \geq \frac{1}{2}, \quad \P(X \geq \hat x) \geq \frac{1}{2}.
  \]
  Here we have a non-strict inequality since if the random variable, say, always takes value $0$, then both probabilities will be 1.
\end{defi}

Suppose that we have an experiment whose outcome follows the distribution of $X$. We can perform the experiment many times and obtain many results $X_1, \cdots, X_n$. The average of these results is known as the \emph{sample mean}.

\begin{defi}[Sample mean]
  If $X_1, \cdots, X_n$ is a random sample from some distribution, then the \emph{sample mean} is
  \[
    \bar X = \frac{1}{n}\sum_1^n X_i.
  \]
\end{defi}

\subsection{Stochastic ordering and inspection paradox}
We want to define a (partial) order on two different random variables. For example, we might want to say that $X + 2 > X$, where $X$ is a random variable.

A simple definition might be \emph{expectation ordering}, where $X \geq Y$ if $\E[X] \geq \E[Y]$. This, however, is not satisfactory since $X \geq Y$ and $Y \geq X$ does not imply $X = Y$. Instead, we can use the \emph{stochastic order}.
\begin{defi}[Stochastic order]
  The \emph{stochastic order} is defined as: $X \geq_{\mathrm{st}} Y$ if $\P(X > t) \geq \P(Y > t)$ for all $t$.
\end{defi}
This is clearly transitive. Stochastic ordering implies expectation ordering, since
\[
  X \geq_{\mathrm{st}} Y \Rightarrow \int_0^\infty \P(X > x)\;\d x \geq \int_0^\infty \P(y > x)\;\d x \Rightarrow \E[X] \geq \E[Y].
\]
Alternatively, we can also order things by hazard rate.
\begin{eg}[Inspection paradox]
  Suppose that $n$ families have children attending a school. Family $i$ has $X_i$ children at the school, where $X_1, \cdots, X_n$ are iid random variables, with $P(X_i = k) = p_k$. Suppose that the average family size is $\mu$.

  Now pick a child at random. What is the probability distribution of his family size? Let $J$ be the index of the family from which she comes (which is a random variable). Then
  \[
    \P(X_J = k\mid J = j) = \frac{\P(J = j, X_j = k)}{\P(J = j)}.
  \]
  The denominator is $1/n$. The numerator is more complex. This would require the $j$th family to have $k$ members, which happens with probability $p_k$; and that we picked a member from the $j$th family, which happens with probability $\E\left[\frac{k}{k + \sum_{i \not=j} X_i}\right]$. So
  \[
    \P(X_J = k \mid J = j) = \E\left[\frac{nkp_k}{k + \sum_{i \not=j} X_i}\right].
  \]
  Note that this is independent of $j$. So
  \[
    \P(X_J = k) = \E\left[\frac{nkp_k}{k + \sum_{i \not=j} X_i}\right].
  \]
  Also, $\P(X_1 = k) = p_k$. So
  \[
    \frac{\P(X_J = k)}{\P(X_1 = k)} = \E\left[\frac{nk}{k + \sum_{i \not= j}X_i}\right].
  \]
  This is increasing in $k$, and greater than $1$ for $k > \mu$. So the average value of the family size of the child we picked is greater than the average family size. It can be shown that $X_J$ is stochastically greater than $X_1$.

  This means that if we pick the children randomly, the sample mean of the family size will be greater than the actual mean. This is since for the larger a family is, the more likely it is for us to pick a child from the family.
\end{eg}

\subsection{Jointly distributed random variables}
\begin{defi}[Joint distribution]
  Two random variables $X, Y$ have \emph{joint distribution} $F: \R^2 \mapsto [0, 1]$ defined by
  \[
    F(x, y) = \P(X \leq x, Y \leq y).
  \]
  The \emph{marginal distribution} of $X$ is
  \[
    F_X(x) = \P(X \leq x) = \P(X \leq x, Y < \infty) = F(x, \infty) = \lim_{y \to \infty}F(x, y)
  \]
\end{defi}

\begin{defi}[Jointly distributed random variables]
  We say $X_1, \cdots, X_n$ are \emph{jointly distributed continuous random variables} and have \emph{joint pdf} $f$ if for any set $A\subseteq \R^n$
  \[
    \P((X_1, \cdots, X_n)\in A) = \int_{(x_1, \cdots x_n)\in A} f(x_1, \cdots, x_n)\;\d x_1\cdots \d x_n.
  \]
  where
  \[
    f(x_1, \cdots, x_n) \geq 0
  \]
  and
  \[
    \int_{\R^n} f(x_1, \cdots, x_n)\;\d x_1 \cdots \d x_n = 1.
  \]
\end{defi}

\begin{eg}
  In the case where $n = 2$,
  \[
    F(x, y) = \P(X \leq x,Y \leq y) = \int_{-\infty}^x\int_{-\infty}^y f(x, y)\;\d x\;\d y.
  \]
  If $F$ is differentiable, then
  \[
    f(x, y) = \frac{\partial^2}{\partial x\partial y}F(x, y).
  \]
\end{eg}

\begin{thm}
  If $X$ and $Y$ are jointly continuous random variables, then they are individually continuous random variables.
\end{thm}

\begin{proof}
  We prove this by showing that $X$ has a density function.

  We know that
  \begin{align*}
    \P(X \in A) &= \P(X \in A, Y\in (-\infty, +\infty))\\
    &= \int_{x\in A}\int_{-\infty}^\infty f(x, y)\;\d y\;\d x\\
    &= \int_{x\in A} f_X(x)\;\d x
  \end{align*}
  So
  \[
    f_X(x) = \int_{-\infty}^\infty f(x, y)\;\d y
  \]
  is the (marginal) pdf of $X$.
\end{proof}

\begin{defi}[Independent continuous random variables]
  Continuous random variables $X_1, \cdots, X_n$ are independent if
  \[
    \P(X_1\in A_1, X_2 \in A_2, \cdots, X_n\in A_n) = \P(X_1 \in A_1)\P(X_2 \in A_2) \cdots \P(X_n \in A_n)
  \]
  for all $A_i\subseteq \Omega_{X_i}$.

  If we let $F_{X_i}$ and $f_{X_i}$ be the cdf, pdf of $X$, then
  \[
    F(x_1, \cdots, x_n) = F_{X_1}(x_1)\cdots F_{X_n}(x_n)
  \]
  and
  \[
    f(x_1, \cdots, x_n) = f_{X_1}(x_1) \cdots f_{X_n}(x_n)
  \]
  are each individually equivalent to the definition above.
\end{defi}

To show that two (or more) random variables are independent, we only have to factorize the joint pdf into factors that each only involve one variable.
\begin{eg}
  If $(X_1, X_2)$ takes a random value from $[0, 1] \times [0, 1]$, then $f(x_1, x_2) = 1$. Then we can see that $f(x_1, x_2) = 1\cdot 1 = f(x_1)\cdot f(x_2)$. So $X_1$ and $X_2$ are independent.

  On the other hand, if $(Y_1, Y_2)$ takes a random value from $[0, 1] \times [0, 1]$ with the restriction that $Y_1 \leq Y_2$, then they are not independent, since $f(x_1, x_2) = 2 I[Y_1 \leq Y_2]$, which cannot be split into two parts.
\end{eg}

\begin{prop}
  For independent continuous random variables $X_i$,
  \begin{enumerate}
    \item $\E [\prod X_i] = \prod \E [X_i]$
    \item $\var (\sum X_i) = \sum \var (X_i)$
  \end{enumerate}
\end{prop}

\subsection{Geometric probability}
Often, when doing probability problems that involve geometry, we can visualize the outcomes with the aid of a picture.

\begin{eg}
  Two points $X$ and $Y$ are chosen independently on a line segment of length $L$. What is the probability that $|X - Y| \leq \ell$? By ``at random'', we mean
  \[
    f(x, y) = \frac{1}{L^2},
  \]
  since each of $X$ and $Y$ have pdf $1/L$.

  We can visualize this on a graph:
  \begin{center}
    \begin{tikzpicture}
      \draw (0, 0) rectangle (3, 3);
      \draw [fill=gray!50!white] (0, 0) -- (0.5, 0) -- (3, 2.5) -- (3, 3) -- (2.5, 3) -- (0, 0.5) -- cycle;
      \node at (1.5, 1.5) {A};
      \node at (0.5, 0) [below] {$\ell$};
      \node at (3, 0) [below] {$L$};
      \node at (0, 0.5) [left] {$\ell$};
      \node at (0, 3) [left] {$L$};
    \end{tikzpicture}
  \end{center}
  Here the two axes are the values of $X$ and $Y$, and $A$ is the permitted region. The total area of the white part is simply the area of a square with length $L - \ell$. So the area of $A$ is $L^2 - (L - \ell)^2 = 2L\ell - \ell^2$. So the desired probability is
  \[
    \int_A f(x, y)\;\d x\;\d y = \frac{2L\ell - \ell^2}{L^2}.
  \]
\end{eg}

\begin{eg}[Bertrand's paradox]
  Suppose we draw a random chord in a circle. What is the probability that the length of the chord is greater than the length of the side of an inscribed equilateral triangle?

  There are three ways of ``drawing a random chord''.
  \begin{enumerate}
    \item We randomly pick two end points over the circumference independently. Now draw the inscribed triangle with the vertex at one end point. For the length of the chord to be longer than a side of the triangle, the other end point must between the two other vertices of the triangle. This happens with probability $1/3$.
      \begin{center}
        \begin{tikzpicture}
          \draw circle [radius = 1];
          \draw (0, 1) -- (0.866, -0.5) -- (-0.866, -0.5) -- cycle;
          \node [circ, fill=red] at (0, 1) {};
          \draw [mblue] (0, 1) -- (-1, 0);
          \draw [mred] (0, 1) -- (0, -1);
          \draw [mblue] (0, 1) -- (1, 0);
        \end{tikzpicture}
      \end{center}
    \item wlog the chord is horizontal, on the lower side of the circle. The mid-point is uniformly distributed along the middle (dashed) line. Then the probability of getting a long line is $1/2$.
      \begin{center}
        \begin{tikzpicture}
          \draw circle [radius = 1];
          \draw (0, 1) -- (0.866, -0.5) -- (-0.866, -0.5) -- cycle;
          \node [circ, fill=mred] at (0, 1) {};
          \draw [dashed] (0, 1) -- (0, -1);
          \draw [mblue] (-0.4359, -0.9) -- (0.4359, -0.9);
          \draw [mblue] (-0.8, -0.6) -- (0.8, -0.6);
          \draw [mred] (-0.9539, -0.3) -- (0.9539, -0.3);
          \draw [mred] (-1, 0) -- (1, 0);
        \end{tikzpicture}
      \end{center}
    \item The mid point of the chord is distributed uniformly across the circle. Then you get a long line if and only if the mid-point lies in the smaller circle shown below. This occurs with probability $1/4$.
       \begin{center}
        \begin{tikzpicture}
          \draw circle [radius = 1];
          \draw (0, 1) -- (0.866, -0.5) -- (-0.866, -0.5) -- cycle;
          \node [circ, fill=mred] at (0, 1) {};
          \draw circle [radius = 0.5];
          \draw [mred] (0, -1) -- (-0.6, 0.8);
          \draw [mblue] (-0.9539, 0.3) -- (0.4359, 0.9);
          \draw [mblue] (0.6, -0.8) -- (0.9165, 0.4);
          \draw [mblue] (-1, 0) -- (0.3, -0.9539);
        \end{tikzpicture}
      \end{center}
  \end{enumerate}
  We get different answers for different notions of ``random''! This is why when we say ``randomly'', we should be explicit in what we mean by that.
\end{eg}

\begin{eg}[Buffon's needle]
  A needle of length $\ell$ is tossed at random onto a floor marked with parallel lines a distance $L$ apart, where $\ell \leq L$. Let $A$ be the event that the needle intersects a line. What is $\P(A)$?
  \begin{center}
    \begin{tikzpicture}
      \draw (-3, -1) -- (3, -1);
      \draw (-3, 1) -- (3, 1);
      \draw (-2, -0.1) -- (-0.5, 0.7) node [anchor = south east, pos = 0.5] {$\ell$};
      \draw [->] (-2, -0.1) -- (-2, 1) node [left, pos = 0.5] {$X$};
      \draw [->] (-2, 1) -- (-2, -0.1);
      \draw [dashed] (-2.5, -0.1) -- (-0.5, -0.1);
      \draw (-1.5, -0.1) arc (0:29:0.5);
      \node at (-1.2, 0.1) {$\theta$};
      \draw [->] (2, -1) -- (2, 1) node [right, pos = 0.5] {$L$};
      \draw [->] (2, 1) -- (2, -1);
    \end{tikzpicture}
  \end{center}
  Suppose that $X\sim U[0, L]$ and $\theta\sim U[0, \pi]$. Then
  \[
    f(x, \theta) = \frac{1}{L\pi}.
  \]
  This touches a line iff $X \leq \ell \sin \theta$. Hence
  \[
    \P(A) = \int_{\theta = 0}^\pi \underbrace{\frac{\ell \sin \theta}{\ell}}_{\P(X \leq \ell\sin \theta)}\frac{1}{\pi}\d \theta = \frac{2\ell }{\pi L}.
  \]
  Since the answer involves $\pi$, we can estimate $\pi$ by conducting repeated experiments! Suppose we have $N$ hits out of $n$ tosses. Then an estimator for $p$ is $\hat{p} = \frac{N}{n}$. Hence
  \[
    \hat{\pi} = \frac{2\ell}{\hat{p}L}.
  \]
  We will later find out that this is a really inefficient way of estimating $\pi$ (as you could have guessed).
\end{eg}
\subsection{The normal distribution}
\begin{defi}[Normal distribution]
  The \emph{normal distribution} with parameters $\mu, \sigma^2$, written $N(\mu, \sigma^2)$ has pdf
  \[
    f(x) = \frac{1}{\sqrt{2 \pi}\sigma}\exp\left(-\frac{(x - \mu)^2}{2\sigma^2}\right),
  \]
  for $-\infty < x < \infty$.

  It looks approximately like this:
  \begin{center}
    \begin{tikzpicture}[yscale=1.5]
      \draw (-3, 0) -- (3, 0);
      \draw (0, 1.3) -- (0, 0) node [below] {$\mu$};
      \draw [domain=-3:3,samples=50, mblue] plot (\x, {exp(-\x * \x)});
    \end{tikzpicture}
  \end{center}
  The \emph{standard normal} is when $\mu = 0, \sigma^2 = 1$, i.e.\ $X\sim N(0, 1)$.

  We usually write $\phi(x)$ for the pdf and $\Phi(x)$ for the cdf of the standard normal.
\end{defi}
This is a rather important probability distribution. This is partly due to the central limit theorem, which says that if we have a large number of iid random variables, then the distribution of their averages are approximately normal. Many distributions in physics and other sciences are also approximately or exactly normal.

We first have to show that this makes sense, i.e.
\begin{prop}
  \[
    \int_{-\infty}^\infty \frac{1}{\sqrt{2\pi \sigma^2}} e^{-\frac{1}{2\sigma^2}(x - \mu)^2}\;\d x = 1.
  \]
\end{prop}
\begin{proof}
  Substitute $z = \frac{(x - \mu)}{\sigma}$. Then
  \[
    I = \int_{-\infty}^\infty \frac{1}{\sqrt{2\pi}}e^{-\frac{1}{2}z^2}\;\d z.
  \]
  Then
  \begin{align*}
    I^2 &= \int_{-\infty}^\infty \frac{1}{\sqrt{2\pi}} e^{-x^2/2}\;\d x\int_{\infty}^\infty \frac{1}{\sqrt{2\pi}} e^{-y^2/2}\;\d y\\
    &= \int_0^\infty \int_0^{2\pi}\frac{1}{2\pi}e^{-r^2/2}r\;\d r\;\d \theta\\
    &= 1.\qedhere
  \end{align*}
\end{proof}

We also have
\begin{prop}
  $\E[X] = \mu$.
\end{prop}
\begin{proof}
\begin{align*}
  \E[X] &= \frac{1}{\sqrt{2\pi}\sigma} \int_{-\infty}^\infty x e^{-(x - \mu)^2/2\sigma^2}\;\d x\\
  &= \frac{1}{\sqrt{2\pi} \sigma}\int_{-\infty}^\infty (x - \mu)e^{-(x - \mu)^2/2\sigma^2}\;\d x + \frac{1}{\sqrt{2\pi}\sigma}\int_{-\infty}^\infty \mu e^{-(x - \mu)^2/2\sigma^2}\;\d x.
\end{align*}
The first term is antisymmetric about $\mu$ and gives $0$. The second is just $\mu$ times the integral we did above. So we get $\mu$.
\end{proof}
Also, by symmetry, the mode and median of a normal distribution are also both $\mu$.

\begin{prop}
  $\var(X) = \sigma^2$.
\end{prop}

\begin{proof}
  We have $\var (X) = \E[X^2] - (\E[X])^2$. Substitute $Z = \frac{X - \mu}{\sigma}$. Then $\E[Z] = 0$, $\E[Z^2] = \frac{1}{\sigma^2}\E[X^2]$.

  Then
  \begin{align*}
    \var(Z) &= \frac{1}{\sqrt{2\pi}} \int_{-\infty}^\infty z^2 e^{-z^2/2}\;\d z\\
    &= \left[-\frac{1}{\sqrt{2\pi}}ze^{-z^2/2}\right]_{-\infty}^\infty + \frac{1}{\sqrt{2\pi}}\int_{-\infty}^\infty e^{-z^2/2}\;\d z\\
    &= 0 + 1\\
    &= 1
  \end{align*}
  So $\var X = \sigma^2$.
\end{proof}

\begin{eg}
  UK adult male heights are normally distributed with mean 70'' and standard deviation 3''. In the Netherlands, these figures are 71'' and 3''.

  What is $\P(Y > X)$, where $X$ and $Y$ are the heights of randomly chosen UK and Netherlands males, respectively?

  We have $X\sim N(70, 3^2)$ and $Y\sim N(71, 3^2)$. Then (as we will show in later lectures) $Y - X \sim N(1, 18)$.
  \[
    \P(Y > X) = \P(Y - X > 0) = \P\left(\frac{Y - X - 1}{\sqrt{18}} > \frac{-1}{\sqrt{18}}\right) = 1 - \Phi(-1/\sqrt{18}),
  \]
  since $\frac{(Y - X) - 1}{\sqrt{18}}\sim N(0, 1)$, and the answer is approximately 0.5931.

  Now suppose that in both countries, the Olympic male basketball teams are selected from that portion of male whose hight is at least above 4'' above the mean (which corresponds to the $9.1\%$ tallest males of the country). What is the probability that a randomly chosen Netherlands player is taller than a randomly chosen UK player?

  For the second part, we have
  \[
    \P(Y > X\mid X \geq 74, Y \geq 75) = \frac{\int_{x = 74}^{75} \phi_X(x)\;\d x + \int_{x = 75}^\infty \int_{y=x}^\infty \phi_Y(y)\phi_X(x)\;\d y\;\d x}{\int_{x=74}^\infty \phi_X(x)\;\d x \int_{y=75}^\infty \phi_Y(y)\;\d y},
  \]
  which is approximately 0.7558. So even though the Netherlands people are only slightly taller, if we consider the tallest bunch, the Netherlands people will be much taller on average.
\end{eg}

\subsection{Transformation of random variables}
We will now look at what happens when we apply a function to random variables. We first look at the simple case where there is just one variable, and then move on to the general case where we have multiple variables and can mix them together.
\subsubsection*{Single random variable}
\begin{thm}
  If $X$ is a continuous random variable with a pdf $f(x)$, and $h(x)$ is a continuous, strictly increasing function with $h^{-1}(x)$ differentiable, then $Y = h(X)$ is a random variable with pdf
  \[
    f_Y(y) = f_X(h^{-1}(y))\frac{\d }{\d y}h^{-1}(y).
  \]
\end{thm}
\begin{proof}
  \begin{align*}
    F_Y(y) &= \P(Y \leq y)\\
    &= \P(h(X) \leq y)\\
    &= \P(X \leq h^{-1}(y))\\
    &= F(h^{-1}(y))
  \end{align*}
  Take the derivative with respect to $y$ to obtain
  \[
    f_Y(y) = F'_Y(y) = f(h^{-1}(y))\frac{\d}{\d y}h^{-1}(y).\qedhere
  \]
\end{proof}
It is often easier to redo the proof than to remember the result.

\begin{eg}
  Let $X\sim U[0, 1]$. Let $Y = -\log X$. Then
  \begin{align*}
    \P(Y \leq y) &= \P(-\log X \leq y)\\
    &= \P(X \geq e^{-y})\\
    &= 1 - e^{-y}.
  \end{align*}
  But this is the cumulative distribution function of $\mathcal{E}(1)$. So $Y$ is exponentially distributed with $\lambda = 1$.
\end{eg}

In general, we get the following result:
\begin{thm}
  Let $U\sim U[0, 1]$. For any strictly increasing distribution function $F$, the random variable $X = F^{-1}U$ has distribution function $F$.
\end{thm}

\begin{proof}
  \[
    \P(X \leq x) = \P(F^{-1}(U) \leq x) = \P(U \leq F(x)) = F(x).\qedhere
  \]
\end{proof}
This condition ``strictly increasing'' is needed for the inverse to exist. If it is not, we can define
\[
  F^{-1}(u) = \inf\{x: F(x)\geq u, 0 < u < 1\},
\]
and the same result holds.

This can also be done for discrete random variables $\P(X = x_i) = p_i$ by letting
\[
  X = x_j\text{ if }\sum_{i = 0}^{j - 1}p_i \leq U < \sum_{i = 0}^j p_i,
\]
for $U\sim U[0, 1]$.

\subsubsection*{Multiple random variables}
Suppose $X_1, X_2, \cdots, X_n$ are random variables with joint pdf $f$. Let
\begin{align*}
  Y_1 &= r_1(X_1, \cdots, X_n)\\
  Y_2 &= r_2(X_1, \cdots, X_n)\\
  &\vdots \\
  Y_n &= r_n(X_1, \cdots, X_n).
\end{align*}
For example, we might have $Y_1 = \frac{X_1}{X_1 + X_2}$ and $Y_2 = X_1 + X_2$.

Let $R\subseteq \R^n$ such that $\P((X_1, \cdots, X_n)\in R) = 1$, i.e.\ $R$ is the set of all values $(X_i)$ can take.

Suppose $S$ is the image of $R$ under the above transformation, and the map $R\to S$ is bijective. Then there exists an inverse function
\begin{align*}
  X_1 &= s_1(Y_1, \cdots, Y_n)\\
  X_2 &= s_2(Y_1, \cdots, Y_n)\\
  &\vdots\\
  X_n &= s_n(Y_1, \cdots, Y_n).
\end{align*}
For example, if $X_1, X_2$ refers to the coordinates of a random point in Cartesian coordinates, $Y_1, Y_2$ might be the coordinates in polar coordinates.

\begin{defi}[Jacobian determinant]
  Suppose $\frac{\partial s_i}{\partial y_j}$ exists and is continuous at every point $(y_1, \cdots, y_n)\in S$. Then the \emph{Jacobian determinant} is
  \[
    J = \frac{\partial (s_1, \cdots, s_n)}{\partial (y_1, \cdots, y_n)} =
    \det
    \begin{pmatrix}
      \frac{\partial s_1}{\partial y_1} & \cdots & \frac{\partial s_1}{\partial y_n}\\
      \vdots & \ddots & \vdots\\
      \frac{\partial s_n}{\partial y_1} & \cdots & \frac{\partial s_n}{\partial y_n}\\
    \end{pmatrix}
  \]
\end{defi}
Take $A\subseteq \R$ and $B = r(A)$. Then using results from IA Vector Calculus, we get
\begin{align*}
  \P((X_1, \cdots, X_n)\in A) &= \int_A f(x_1, \cdots, x_n) \;\d x_1\cdots \d x_n\\
  &= \int_B f(s_1(y_1, \cdots y_n), s_2, \cdots, s_n) |J|\;\d y_1\cdots \;\d y_n\\
  &= \P((Y_1, \cdots Y_n)\in B).
\end{align*}
So
\begin{prop}
  $(Y_1, \cdots, Y_n)$ has density
  \[
    g(y_1, \cdots, y_n) = f(s_1(y_1, \cdots, y_n), \cdots s_n(y_1, \cdots, y_n))|J|
  \]
  if $(y_1, \cdots, y_n)\in S$, $0$ otherwise.
\end{prop}

\begin{eg}
  Suppose $(X, Y)$ has density
  \[
    f(x, y) =
    \begin{cases}
      4xy & 0 \leq x \leq 1, 0\leq y \leq 1\\
      0 & \text{otherwise}
    \end{cases}
  \]
  We see that $X$ and $Y$ are independent, with each having a density $f(x) = 2x$.

  Define $U = X/Y$, $V = XY$. Then we have $X = \sqrt{UV}$ and $Y = \sqrt{V/U}$.

  The Jacobian is
  \[
    \det
    \begin{pmatrix}
      \partial x/\partial u & \partial x/\partial v\\
      \partial y/\partial u & \partial y/\partial v
    \end{pmatrix}
    =
    \det
    \begin{pmatrix}
      \frac{1}{2}\sqrt{v/u} & \frac{1}{2}\sqrt{u/v}\\
      -\frac{1}{2}\sqrt{v/u^3} & \frac{1}{2}\sqrt{1/uv}
    \end{pmatrix}
    = \frac{1}{2u}
  \]
  Alternatively, we can find this by considering
  \[
    \det
    \begin{pmatrix}
      \partial u/\partial x & \partial u/\partial y\\
      \partial v/\partial x & \partial u/\partial y
    \end{pmatrix} = 2u,
  \]
  and then inverting the matrix. So
  \[
    g(u, v) = 4\sqrt{uv}\sqrt{\frac{v}{u}}\frac{1}{2u} = \frac{2v}{u},
  \]
  if $(u, v)$ is in the image $S$, $0$ otherwise. So
  \[
    g(u, v) = \frac{2v}{u}I[(u, v)\in S].
  \]
  Since this is not separable, we know that $U$ and $V$ are not independent.
\end{eg}

In the linear case, life is easy. Suppose
\[
  \mathbf{Y} = \begin{pmatrix}
    Y_1\\
    \vdots\\
    Y_n
  \end{pmatrix} = A
  \begin{pmatrix}
    X_1\\
    \vdots\\
    X_n
  \end{pmatrix} = A\mathbf{X}
\]
Then $\mathbf{X} = A^{-1}\mathbf{Y}$. Then $\frac{\partial x_i}{\partial y_j} = (A^{-1})_{ij}$. So $|J| = |\det(A^{-1})| = |\det A|^{-1}$. So
\[
  g(y_1, \cdots, y_n) = \frac{1}{|\det A|}f(A^{-1}\mathbf{y}).
\]

\begin{eg}
  Suppose $X_1, X_2$ have joint pdf $f(x_1,x_2)$. Suppose we want to find the pdf of $Y = X_1 + X_2$. We let $Z = X_2$. Then $X_1 = Y - Z$ and $X_2 = Z$. Then
  \[
    \begin{pmatrix}
      Y\\
      Z
    \end{pmatrix} =
    \begin{pmatrix}
      1 & 1 \\
      0 & 1
    \end{pmatrix}
    \begin{pmatrix}
      X_1\\
      X_2
    \end{pmatrix}
    = A\mathbf{X}
  \]
  Then $|J| = 1/|\det A| = 1$. Then
  \[
    g(y, z) = f(y - z, z)
  \]
  So
  \[
    g_Y(y) = \int_{-\infty}^\infty f(y - z, z)\;\d z = \int_{-\infty}^\infty f(z, y - z)\;\d z.
  \]
  If $X_1$ and $X_2$ are independent, $f(x_1, x_2) = f_1(x_1) f_2(x_2)$. Then
  \[
    g(y) = \int_{-\infty}^\infty f_1(z) f_2(y - z)\;\d z.
  \]
\end{eg}

\subsubsection*{Non-injective transformations}
We previously discussed transformation of random variables by injective maps. What if the mapping is not? There is no simple formula for that, and we have to work out each case individually.

\begin{eg}
  Suppose $X$ has pdf $f$. What is the pdf of $Y=|X|$?

  We use our definition. We have
  \[
    \P(|X| \in x(a, b)) = \int_a^b f(x) + \int_{-b}^{-a} f(x)\;\d x = \int_a^b (f(x) + f(-x))\;\d x.
  \]
  So
  \[
    f_Y(x) = f(x) + f(-x),
  \]
  which makes sense, since getting $|X| = x$ is equivalent to getting $X = x$ or $X = -x$.
\end{eg}

\begin{eg}
  Suppose $X_1\sim \mathcal{E}(\lambda), X_2\sim \mathcal{E}(\mu)$ are independent random variables. Let $Y = \min(X_1, X_2)$. Then
  \begin{align*}
    \P(Y \geq t) &= \P(X_1 \geq t, X_2 \geq t)\\
    &= \P(X_1\geq t)\P(X_2 \geq t)\\
    &= e^{-\lambda t}e^{-\mu t}\\
    &= e^{-(\lambda +\mu)t}.
  \end{align*}
  So $Y\sim \mathcal{E}(\lambda + \mu)$.
\end{eg}

Given random variables, not only can we ask for the minimum of the variables, but also ask for, say, the second-smallest one. In general, we define the \emph{order statistics} as follows:
\begin{defi}[Order statistics]
  Suppose that $X_1, \cdots, X_n$ are some random variables, and $Y_1, \cdots, Y_n$ is $X_1, \cdots, X_n$ arranged in increasing order, i.e.\ $Y_1 \leq Y_2 \leq \cdots \leq Y_n$. This is the \emph{order statistics}.

  We sometimes write $Y_i = X_{(i)}$.
\end{defi}

Assume the $X_i$ are iid with cdf $F$ and pdf $f$. Then the cdf of $Y_n$ is
\[
  \P(Y_n \leq y) = \P(X_1 \leq y, \cdots, X_n \leq y) = \P(X_1 \leq y)\cdots\P(X_n \leq y) = F(y)^n.
\]
So the pdf of $Y_n$ is
\[
  \frac{\d }{\d y}F(y)^n = nf(y)F(y)^{n - 1}.
\]
Also,
\[
  \P(Y_1 \geq y) = \P(X_1\geq y, \cdots, X_n \geq y) = (1 - F(y))^n.
\]
What about the joint distribution of $Y_1, Y_n$?
\begin{align*}
  G(y_1, y_n) &= \P(Y_1\leq y_1, Y_n \leq y_n)\\
  &= \P(Y_n \leq y_n) - \P(Y_1 \geq y_1 , Y_n \leq y_n)\\
  &= F(y_n)^n - (F(y_n) - F(y_1))^n.
\end{align*}
Then the pdf is
\[
  \frac{\partial^2}{\partial y_1 \partial y_n}G(y_1, y_n) = n(n - 1)(F(y_n) - F(y_1))^{n - 2}f(y_1)f(y_n).
\]
We can think about this result in terms of the multinomial distribution. By definition, the probability that $Y_1\in [y_1, y_1 + \delta)$ and $Y_n\in (y_n - \delta, y_n]$ is approximately $g(y_1, y_n)$.

Suppose that $\delta$ is sufficiently small that all other $n - 2$ $X_i$'s are very unlikely to fall into $[y_1, y_1 + \delta)$ and $(y_n - \delta, y_n]$. Then to find the probability required, we can treat the sample space as three bins. We want exactly one $X_i$ to fall into the first and last bins, and $n - 2$ $X_i$'s to fall into the middle one. There are $\binom{n}{1, n-2, 1} = n(n - 1)$ ways of doing so.

The probability of each thing falling into the middle bin is $F(y_n) - F(y_1)$, and the probabilities of falling into the first and last bins are $f(y_1)\delta$ and $f(y_n)\delta$. Then the probability of $Y_1\in [y_1, y_1 + \delta)$ and $Y_n\in (y_n - \delta, y_n]$ is
\[
  n(n - 1)(F(y_n) - F(y_1))^{n - 2} f(y_1)f(y_n)\delta^2,
\]
and the result follows.

We can also find the joint distribution of the order statistics, say $g$, since it is just given by
\[
  g(y_1, \cdots y_n) = n! f(y_1)\cdots f(y_n),
\]
if $y_1 \leq y_2\leq \cdots \leq y_n$, 0 otherwise. We have this formula because there are $n!$ combinations of $x_1, \cdots, x_n$ that produces a given order statistics $y_1, \cdots, y_n$, and the pdf of each combination is $f(y_1)\cdots f(y_n)$.

In the case of iid exponential variables, we find a nice distribution for the order statistic.
\begin{eg}
  Let $X_1, \cdots, X_n$ be iid $\mathcal{E}(\lambda)$, and $Y_1, \cdots, Y_n$ be the order statistic. Let
  \begin{align*}
    Z_1 &= Y_1\\
    Z_2 &= Y_2 - Y_1\\
    &\vdots\\
    Z_n &= Y_n - Y_{n - 1}.
  \end{align*}
  These are the distances between the occurrences. We can write this as a $\mathbf{Z} = A\mathbf{Y}$, with
  \[
    A =
    \begin{pmatrix}
      1 & 0 & 0 & \cdots& 0\\
      -1 & 1 & 0 & \cdots & 0\\
      \vdots & \vdots & \vdots & \ddots & \vdots\\
      0 & 0 & 0 & \cdots & 1\\
    \end{pmatrix}
  \]
  Then $\det (A) = 1$ and hence $|J| = 1$. Suppose that the pdf of $Z_1, \cdots, Z_n$ is, say $h$. Then
  \begin{align*}
    h(z_1, \cdots, z_n) &= g(y_1, \cdots, y_n)\cdot 1\\
    &= n!f(y_1) \cdots f(y_n)\\
    &= n!\lambda^n e^{-\lambda (y_1 + \cdots + y_n)}\\
    &= n!\lambda^n e^{-\lambda (nz_1 + (n - 1)z_2 + \cdots + z_n)}\\
    &= \prod_{i = 1}^n (\lambda i)e^{-(\lambda i)z_{n + 1 - i}}
  \end{align*}
  Since $h$ is expressed as a product of $n$ density functions, we have
  \[
    Z_i \sim \mathcal{E}((n + 1 - i)\lambda).
  \]
  with all $Z_i$ independent.
\end{eg}


\subsection{Moment generating functions}
If $X$ is a continuous random variable, then the analogue of the probability generating function is the moment generating function:
\begin{defi}[Moment generating function]
  The \emph{moment generating function} of a random variable $X$ is
  \[
    m(\theta) = \E[e^{\theta X}].
  \]
  For those $\theta$ in which $m(\theta)$ is finite, we have
  \[
    m(\theta) = \int_{-\infty}^\infty e^{\theta x}f(x)\;\d x.
  \]
\end{defi}
We can prove results similar to that we had for probability generating functions.

We will assume the following without proof:
\begin{thm}
  The mgf determines the distribution of $X$ provided $m(\theta)$ is finite for all $\theta$ in some interval containing the origin.
\end{thm}

\begin{defi}[Moment]
  The $r$th \emph{moment} of $X$ is $\E[X^r]$.
\end{defi}

\begin{thm}
  The $r$th moment $X$ is the coefficient of $\frac{\theta^r}{r!}$ in the power series expansion of $m(\theta)$, and is
  \[
    \E[X^r] = \left.\frac{\d ^n}{\d \theta^n}m(\theta)\right|_{\theta = 0} = m^{(n)}(0).
  \]
\end{thm}

\begin{proof}
  We have
  \[
    e^{\theta X} = 1 + \theta X + \frac{\theta^2}{2!}X^2 + \cdots.
  \]
  So
  \[
    m(\theta) = \E[e^{\theta X}] = 1 + \theta \E[X] + \frac{\theta^2}{2!}\E [X^2] + \cdots\qedhere
  \]
\end{proof}

\begin{eg}
  Let $X\sim \mathcal{E}(\lambda)$. Then its mgf is
  \[
    \E[e^{\theta X}] = \int_0^\infty e^{\theta x}\lambda e^{-\lambda x}\;\d x = \lambda\int_0^\infty e^{-(\lambda - \theta )x}\;\d x = \frac{\lambda}{\lambda - \theta},
  \]
  where $0 < \theta < \lambda$. So
  \[
    \E[X] = m'(0) = \left.\frac{\lambda}{(\lambda - \theta)^2}\right|_{\theta = 0} = \frac{1}{\lambda}.
  \]
  Also,
  \[
    \E[X^2] = m''(0) = \left.\frac{2\lambda}{(\lambda - \theta)^3}\right|_{\theta = 0} = \frac{2}{\lambda^2}.
  \]
  So
  \[
    \var(X) = \E[X^2] - \E[X]^2 = \frac{2}{\lambda^2} - \frac{1}{\lambda^2} = \frac{1}{\lambda^2}.
  \]
\end{eg}

\begin{thm}
  If $X$ and $Y$ are independent random variables with moment generating functions $m_X(\theta), m_Y(\theta)$, then $X + Y$ has mgf $m_{X + Y}(\theta) = m_X(\theta)m_Y(\theta)$.
\end{thm}

\begin{proof}
  \[
    \E[e^{\theta(X + Y)}] = \E[e^{\theta X}e^{\theta Y}] = \E[e^{\theta X}]\E[e^{\theta Y}] = m_X(\theta)m_Y(\theta).\qedhere
  \]
\end{proof}
\section{More distributions}
\subsection{Cauchy distribution}

\begin{defi}[Cauchy distribution]
  The \emph{Cauchy distribution} has pdf
  \[
    f(x) = \frac{1}{\pi(1 + x^2)}
  \]
  for $-\infty < x < \infty$.
\end{defi}
We check that this is a genuine distribution:
\[
  \int_{-\infty}^\infty \frac{1}{\pi(1 + x^2)}\;\d x = \int_{-\pi/2}^{\pi/2}\frac{1}{\pi}\;\d \theta = 1
\]
with the substitution $x = \tan \theta$. The distribution is a bell-shaped curve.

\begin{prop}
  The mean of the Cauchy distribution is undefined, while $\E[X^2] = \infty$.
\end{prop}

\begin{proof}
  \[
    \E[X] = \int_0^\infty \frac{x}{\pi(1 + x^2)}\;\d x + \int_{-\infty}^0 \frac{x}{\pi(1 + x^2)}\;\d x = \infty - \infty
  \]
  which is undefined, but $\E[X^2] = \infty + \infty = \infty$.
\end{proof}

Suppose $X, Y$ are independent Cauchy distributions. Let $Z = X + Y$. Then
\begin{align*}
  f(z) &= \int_{-\infty}^\infty f_X(x)f_Y(z - x)\;\d x\\
  &= \int_{-\infty}^\infty \frac{1}{\pi^2} \frac{1}{(1 + x^2)(1 + (z - x)^2)}\;\d x\\
  &= \frac{1/2}{\pi(1 + (z/2)^2)}
\end{align*}
for all $-\infty < z < \infty$ (the integral can be evaluated using a tedious partial fraction expansion).

So $\frac{1}{2}Z$ has a Cauchy distribution. Alternatively the arithmetic mean of Cauchy random variables is a Cauchy random variable.

By induction, we can show that $\frac{1}{n}(X_1 + \cdots + X_n)$ follows Cauchy distribution. This becomes a ``counter-example'' to things like the weak law of large numbers and the central limit theorem. Of course, this is because those theorems require the random variable to have a mean, which the Cauchy distribution lacks.

\begin{eg}\leavevmode
  \begin{enumerate}
    \item If $\Theta\sim U[-\frac{\pi}{2}, \frac{\pi}{2}]$, then $X = \tan \theta$ has a Cauchy distribution. For example, if we fire a bullet at a wall 1 meter apart at a random random angle $\theta$, the vertical displacement follows a Cauchy distribution.
      \begin{center}
        \begin{tikzpicture}
          \node [circ] {};
          \draw [dashed] (0, 0) -- (3, 0) node [pos = 0.5, below] {$1$};
          \draw [dashed] (0, 0) -- (3, 2);
          \draw [->] (0, 0) -- (1.5, 1);
          \draw (0.5, 0) arc (0:33.69:0.5);
          \node at (0.7, 0.2) {$\theta$};
          \draw [->] (3.2, 0) -- (3.2, 2) node [pos = 0.5, right] {$X = \tan \theta$};
          \draw [->] (3.2, 2) -- (3.2, 0);
          \draw (3, -2.5) -- (3, 2.5);
        \end{tikzpicture}
      \end{center}
    \item If $X, Y\sim N(0, 1)$, then $X/Y$ has a Cauchy distribution.
  \end{enumerate}
\end{eg}

\subsection{Gamma distribution}
Suppose $X_1, \cdots, X_n$ are iid $\mathcal{E}(\lambda)$. Let $S_n = X_1 + \cdots + X_n$. Then the mgf of $S_n$ is
\[
  \E\left[e^{\theta(X_1 + \ldots + X_n)}\right] = \E\left[e^{\theta X_1}\right]\cdots \E\left[e^{\theta X_n}\right] = \left(\E\left[e^{\theta X}\right]\right)^n = \left(\frac{\lambda}{\lambda - \theta}\right)^n.
\]
We call this a gamma distribution.

We claim that this has a distribution given by the following formula:
\begin{defi}[Gamma distribution]
  The \emph{gamma distribution} $\Gamma(n, \lambda)$ has pdf
  \[
    f(x) = \frac{\lambda^n x^{n - 1}e^{-\lambda x}}{(n - 1)!}.
  \]
  We can show that this is a distribution by showing that it integrates to $1$.
\end{defi}
We now show that this is indeed the sum of $n$ iid $\mathcal{E}(\lambda)$:
\begin{align*}
  \E[e^{\theta X}] &= \int_0^\infty e^{\theta x} \frac{\lambda^n x^{n - 1}e^{-\lambda x}}{(n - 1)!}\;\d x\\
  &= \left(\frac{\lambda}{\lambda - \theta}\right)^n \int_0^\infty \frac{(\lambda - \theta)^n x^{n - 1}e^{-(\lambda - \theta)}}{(n - 1)!}\;\d x
\end{align*}
The integral is just integrating over $\Gamma(n, \lambda - \theta)$, which gives one. So we have
\[
  \E[e^{\theta X}]= \left(\frac{\lambda}{\lambda - \theta}\right)^n.
\]

\subsection{Beta distribution*}
Suppose $X_1, \cdots, X_n$ are iid $U[0, 1]$. Let $Y_1 \leq Y_2 \leq \cdots \leq Y_n$ be the order statistics. Then the pdf of $Y_i$ is
\[
  f(y) = \frac{n!}{(i - 1)!(n - i)!}y^{i - 1}(1 - y)^{n - i}.
\]
Note that the leading term is the multinomial coefficient $\binom{n}{i - 1, 1, n - 1}$. The formula is obtained using the same technique for finding the pdf of order statistics.

This is the \emph{beta distribution}: $Y_i \sim \beta(i, n - i + 1)$. In general
\begin{defi}[Beta distribution]
  The \emph{beta distribution} $\beta(a, b)$ has pdf
  \[
    f(x; a, b) = \frac{\Gamma(a + b)}{\Gamma(a)\Gamma(b)}x^{a - 1}(1 - x)^{b - 1}
  \]
  for $0 \leq x \leq 1$.

  This has mean $a/(a + b)$.
\end{defi}
Its moment generating function is
\[
  m(\theta) = 1 + \sum_{k = 1}^\infty \left(\prod_{r = 0}^{k - 1}\frac{a + r}{a + b + r}\right)\frac{\theta^k}{k!},
\]
which is horrendous!

\subsection{More on the normal distribution}
\begin{prop}
  The moment generating function of $N(\mu, \sigma^2)$ is
  \[
    \E[e^{\theta X}] = \exp\left(\theta\mu + \frac{1}{2}\theta^2 \sigma^2\right).
  \]
\end{prop}

\begin{proof}
 \[
  \E[e^{\theta X}] = \int_{-\infty}^\infty e^{\theta x} \frac{1}{\sqrt{2\pi} \sigma} e^{-\frac{1}{2}\sigma^2 (x - \mu)^2}\;\d x.
\]
Substitute $z = \frac{x - \mu}{\sigma}$. Then
\begin{align*}
  \E[e^{\theta X}] &= \int_{-\infty}^\infty \frac{1}{\sqrt{2\pi}}e^{\theta(\mu + \sigma z)} e^{-\frac{1}{2}z^2}\;\d z\\
  &= e^{\theta\mu + \frac{1}{2}\theta^2\sigma^2}\int_{-\infty}^\infty \underbrace{\frac{1}{\sqrt{2\pi}}e^{-\frac{1}{2}(z - \theta \sigma)^2}}_{\text{pdf of }N(\sigma\theta, 1)}\;\d z\\
  &= e^{\theta\mu + \frac{1}{2}\theta^2\sigma^2}.\qedhere
\end{align*}
\end{proof}

\begin{thm}
  Suppose $X, Y$ are independent random variables with $X\sim N(\mu_1, \sigma_1^2)$, and $Y\sim (\mu_2, \sigma_2^2)$. Then
  \begin{enumerate}
    \item $X + Y \sim N(\mu_1 + \mu_2 , \sigma_1^2 + \sigma_2^2)$.
    \item $aX \sim N(a\mu_1, a^2 \sigma_1^2)$.
  \end{enumerate}
\end{thm}

\begin{proof}\leavevmode
  \begin{enumerate}
    \item
      \begin{align*}
        \E[e^{\theta(X + Y)}] &= \E[e^{\theta X}] \cdot \E[e^{\theta Y}]\\
        &= e^{\mu_1 \theta + \frac{1}{2}\sigma_1^2\theta^2} \cdot e^{\mu_2 \theta + \frac{1}{2}\sigma_2^2 \theta}\\
        &= e^{(\mu_1 + \mu_2)\theta + \frac{1}{2}(\sigma_1^2 + \sigma_2^2)\theta^2}
      \end{align*}
      which is the mgf of $N(\mu_1 + \mu_2, \sigma_1^2 + \sigma_2^2)$.
    \item
      \begin{align*}
        \E[e^{\theta (aX)}] &= \E[e^{(\theta a)X}]\\
        &= e^{\mu(a\theta) + \frac{1}{2}\sigma^2 (a \theta)^2}\\
        &= e^{(a\mu)\theta + \frac{1}{2}(a^2\sigma^2)\theta^2}\qedhere
      \end{align*}%\qedhere
  \end{enumerate}
\end{proof}

Finally, suppose $X\sim N(0, 1)$. Write $\phi(x) = \frac{1}{\sqrt{2\pi}} e^{-x^2/2}$ for its pdf. It would be very difficult to find a closed form for its cumulative distribution function, but we can find an upper bound for it:
\begin{align*}
  \P(X \geq x) &= \int_x^\infty \phi(t)\;\d t\\
  &\leq \int_x^\infty \left(1 + \frac{1}{t^2}\right)\phi(t)\;\d t \\
  &= \frac{1}{x}\phi(x)
\end{align*}
To see the last step works, simply differentiate the result and see that you get $\left(1 + \frac{1}{x^2}\right)\phi(x)$.
So
\[
  \P(X \geq x) \leq \frac{1}{x}\frac{1}{\sqrt{2\pi}}e^{-\frac{1}{2}x^2}.
\]
Then
\[
  \log \P(X \geq x)\sim -\frac{1}{2}x^2.
\]

\subsection{Multivariate normal}
Let $X_1, \cdots, X_n$ be iid $N(0, 1)$. Then their joint density is
\begin{align*}
  g(x_1, \cdots, x_n) &= \prod_{i = 1}^n \phi(x_i) \\
  &= \prod_{1}^n\frac{1}{\sqrt{2\pi}}e^{-\frac{1}{2}x_i^2}\\
  &= \frac{1}{(2\pi)^{n/2}} e^{-\frac{1}{2}\sum_1^n x_i^2}\\
  &= \frac{1}{(2\pi)^{n/2}}e^{-\frac{1}{2}\mathbf{x}^T \mathbf{x}},
\end{align*}
where $\mathbf{x} = (x_1, \cdots, x_n)^T$.

This result works if $X_1, \cdots, X_n$ are iid $N(0, 1)$. Suppose we are interested in
\[
  \mathbf{Z} = \boldsymbol\mu + A\mathbf{X},
\]
where $A$ is an invertible $n\times n$ matrix. We can think of this as $n$ measurements $\mathbf{Z}$ that are affected by underlying standard-normal factors $\mathbf{X}$. Then
\[
  \mathbf{X} = A^{-1}(\mathbf{Z} - \boldsymbol\mu)
\]
and
\[
  |J| = |\det (A^{-1})| = \frac{1}{\det A}
\]
So
\begin{align*}
  f(z_1, \cdots, z_n) &= \frac{1}{(2\pi)^{n/2}}\frac{1}{\det A}\exp\left[-\frac{1}{2}\big((A^{-1}(\mathbf{z} - \boldsymbol\mu))^T(A^{-1}(\mathbf{z} - \boldsymbol\mu))\big)\right]\\
  &= \frac{1}{(2\pi)^{n/2}\det A}\exp\left[-\frac{1}{2}(\mathbf{z} - \boldsymbol\mu)^T \Sigma^{-1}(\mathbf{z} - \boldsymbol\mu)\right]\\
  &= \frac{1}{(2\pi)^{n/2}\sqrt{\det \Sigma}}\exp\left[-\frac{1}{2}(\mathbf{z} - \boldsymbol\mu)^T \Sigma^{-1}(\mathbf{z} - \boldsymbol\mu)\right].
\end{align*}
where $\Sigma = AA^T$ and $\Sigma^{-1} = (A^{-1})^TA^{-1}$. We say
\[
  \mathbf{Z} =
  \begin{pmatrix}
    Z_1\\
    \vdots\\
    Z_n
  \end{pmatrix}
  \sim MVN(\boldsymbol\mu, \Sigma)\text{ or }N(\boldsymbol\mu, \Sigma).
\]
This is the multivariate normal.

What is this matrix $\Sigma$? Recall that $\cov(Z_i, Z_j) = \E[(Z_i - \mu_i)(Z_j - \mu_j)]$. It turns out this covariance is the $i, j$th entry of $\Sigma$, since
\begin{align*}
  \E[(\mathbf{Z} - \boldsymbol\mu)(\mathbf{Z} - \boldsymbol\mu)^T] &= \E[AX(AX)^T]\\
  &= \E(AXX^TA^T) = A\E[XX^T]A^T\\
  &= AIA^T\\
  &= AA^T \\
  &= \Sigma
\end{align*}
So we also call $\Sigma$ the covariance matrix.

In the special case where $n = 1$, this is a normal distribution and $\Sigma = \sigma^2$.

Now suppose $Z_1, \cdots, Z_n$ have covariances $0$. Then $\Sigma = \diag(\sigma_1^2, \cdots, \sigma_n^2)$. Then
\[
  f(z_1, \cdots, z_n) = \prod_1^n \frac{1}{\sqrt{2\pi}\sigma_i} e^{-\frac{1}{2\sigma_i^2}(z_i - \mu_i)^2}.
\]
So $Z_1, \cdots, Z_n$ are independent, with $Z_i\sim N(\mu_i, \sigma_i^2)$.

Here we proved that if $\cov = 0$, then the variables are independent. However, this is only true when $Z_i$ are multivariate normal. It is generally not true for arbitrary distributions.

For these random variables that involve vectors, we will need to modify our definition of moment generating functions. We define it to be
\[
  m(\boldsymbol\theta) = \E[e^{\boldsymbol\theta^T \mathbf{X}}] = \E[e^{\theta_1 X_1 + \cdots + \theta_n X_n}].
\]
\subsubsection*{Bivariate normal}
This is the special case of the multivariate normal when $n = 2$. Since there aren't too many terms, we can actually write them out.

The \emph{bivariate normal} has
\[
  \Sigma=
  \begin{pmatrix}
    \sigma_1^2 & \rho\sigma_1\sigma_2\\
    \rho\sigma_1\sigma_2 & \sigma_2^2
  \end{pmatrix}.
\]
Then
\[
  \mathrm{corr}(X_1, X_2) = \frac{\cov(X_1, X_2)}{\sqrt{\var(X_1)\var(X_2)}} = \frac{\rho\sigma_1\sigma_2}{\sigma_1\sigma_2} = \rho.
\]
And
\[
  \Sigma^{-1} = \frac{1}{1 - \rho^2}
  \begin{pmatrix}
    \sigma_1^{-2} & -\rho\sigma_1^{-1}\sigma_2^{-1}\\
    -\rho\sigma_1^{-1}\sigma_2^{-1} & \sigma_2^{-2}
  \end{pmatrix}
\]
The joint pdf of the bivariate normal with zero mean is
\[
  f(x_1, x_2) = \frac{1}{2\pi \sigma_1 \sigma_2 \sqrt{1 - \rho^2}} \exp\left(-\frac{1}{2(1 - \rho^2)}\left(\frac{x_1^2}{\sigma_1^2} - \frac{2\rho x_1x_2}{\sigma_1\sigma_2} + \frac{x_2^2}{\sigma_2^2}\right)\right)
\]
If the mean is non-zero, replace $x_i$ with $x_i - \mu_i$.

The joint mgf of the bivariate normal is
\[
  m(\theta_1, \theta_2) = e^{\theta_1\mu_1 + \theta_2 \mu_2 + \frac{1}{2}(\theta_1^2\sigma_1^2 + 2\theta_1\theta_2\rho\sigma_1\sigma_2 + \theta_2^2\sigma_2^2)}.
\]
Nice and elegant.

\section{Central limit theorem}
Suppose $X_1, \cdots, X_n$ are iid random variables with mean $\mu$ and variance $\sigma^2$. Let $S_n = X_1 + \cdots + X_n$. Then we have previously shown that
\[
  \var(S_n/\sqrt{n}) = \var\left(\frac{S_n - n \mu}{\sqrt{n}}\right) = \sigma^2.
\]
\begin{thm}[Central limit theorem]
  Let $X_1, X_2, \cdots$ be iid random variables with $\E[X_i] = \mu$, $\var(X_i) = \sigma^2 < \infty$. Define
  \[
    S_n = X_1 + \cdots + X_n.
  \]
  Then for all finite intervals $(a, b)$,
  \[
    \lim_{n \to \infty}\P\left(a \leq \frac{S_{n} - n\mu}{\sigma\sqrt{n}} \leq b\right) = \int_a^b \frac{1}{\sqrt{2\pi}} e^{-\frac{1}{2}t^2}\;\d t.
  \]
  Note that the final term is the pdf of a standard normal. We say
  \[
    \frac{S_n - n\mu}{\sigma\sqrt{n}} \to_{D} N(0, 1).
  \]
\end{thm}

To show this, we will use the continuity theorem without proof:
\begin{thm}[Continuity theorem]
  If the random variables $X_1, X_2, \cdots$ have mgf's $m_1(\theta), m_2(\theta), \cdots$ and $m_n(\theta) \to m(\theta)$ as $n\to\infty$ for all $\theta$, then $X_n \to_D $ the random variable with mgf $m(\theta)$.
\end{thm}

We now provide a sketch-proof of the central limit theorem:
\begin{proof}
  wlog, assume $\mu = 0, \sigma^2 = 1$ (otherwise replace $X_i$ with $\frac{X_i - \mu}{\sigma}$).

  Then
  \begin{align*}
    m_{X_i}(\theta) &= \E[e^{\theta X_i}] = 1 + \theta \E[X_i] + \frac{\theta^2}{2!} \E[X_i^2] + \cdots\\
    &=1 + \frac{1}{2}\theta^2 + \frac{1}{3!}\theta^3 \E[X_i^3] + \cdots
  \end{align*}
  Now consider $S_n/\sqrt{n}$. Then
  \begin{align*}
    \E[e^{\theta S_n/\sqrt{n}}] &= \E[e^{\theta(X_1 + \ldots + X_n)/\sqrt{n}}]\\
    &= \E[e^{\theta X_1/\sqrt{n}}] \cdots \E[e^{\theta X_n/\sqrt{n}}]\\
    &= \left(\E[e^{\theta X_1/\sqrt{n}}]\right)^n\\
    &= \left(1 + \frac{1}{2}\theta^2 \frac{1}{n} + \frac{1}{3!}\theta^3 \E[X^3]\frac{1}{n^{3/2}} + \cdots\right)^{n}\\
    &\to e^{\frac{1}{2}\theta^2}
  \end{align*}
  as $n \to \infty$ since $(1 + a/n)^n \to e^a$. And this is the mgf of the standard normal. So the result follows from the continuity theorem.
\end{proof}
Note that this is not a very formal proof, since we have to require $\E[X^3]$ to be finite. Also, sometimes the moment generating function is not defined. But this will work for many ``nice'' distributions we will ever meet.

The proper proof uses the characteristic function
\[
  \chi_X(\theta) = E[e^{i\theta X}].
\]
An important application is to use the normal distribution to approximate a large binomial.

Let $X_i \sim B(1, p)$. Then $S_n \sim B(n, p)$. So $\E[S_n] = np$ and $\var (S_n) = p(1 - p)$. So
\[
  \frac{S_n - np}{\sqrt{np(1 - p)}} \to_D N(0, 1).
\]

\begin{eg}
  Suppose two planes fly a route. Each of $n$ passengers chooses a plane at random. The number of people choosing plane 1 is $S\sim B(n, \frac{1}{2})$. Suppose each has $s$ seats. What is
  \[
    F(s) = \P(S > s),
  \]
  i.e.\ the probability that plane 1 is over-booked? We have
  \[
    F(s) = \P(S > s) = \P\left(\frac{S - n/2}{\sqrt{n\cdot \frac{1}{2}\cdot \frac{1}{2}}} > \frac{s - n/2}{\sqrt{n}/2}\right).
  \]
  Since
  \[
    \frac{S - np}{\sqrt{n}/2}\sim N(0, 1),
  \]
  we have
  \[
    F(s) \approx 1 - \Phi\left(\frac{s - n/2}{\sqrt{n}/2}\right).
  \]
  For example, if $n = 1000$ and $s = 537$, then $\frac{S_n - n/2}{\sqrt{n}/2}\approx 2.34$, $\Phi(2.34)\approx 0.99$, and $F(s) \approx 0.01$. So with only $74$ seats as buffer between the two planes, the probability of overbooking is just $1/100$.
\end{eg}

\begin{eg}
  An unknown proportion $p$ of the electorate will vote Labour. It is desired to find $p$ without an error not exceeding $0.005$. How large should the sample be?

  We estimate by
  \[
    p' = \frac{S_n}{n},
  \]
  where $X_i\sim B(1, p)$. Then
  \begin{align*}
    \P(|p' - p| \leq 0.005) &= \P(|S_n - np| \leq 0.005n)\\
    &= \P\left(\underbrace{\frac{|S_n - np|}{\sqrt{np(1 - p)}}}_{\approx N(0, 1)} \leq \frac{0.005n}{\sqrt{np(1 - p)}}\right)
  \end{align*}
  We want $|p' - p| \leq 0.005$ with probability $\geq 0.95$. Then we want
  \[
    \frac{0.005n}{\sqrt{np(1 - p)}} \geq \Phi^{-1}(0.975) = 1.96.
  \]
  (we use 0.975 instead of 0.95 since we are doing a two-tailed test) Since the maximum possible value of $p(1 - p)$ is $1/4$, we have
  \[
    n \geq 38416.
  \]
  In practice, we don't have that many samples. Instead, we go by
  \[
    \P(|p' < p| \leq 0.03) \geq 0.95.
  \]
  This just requires $n \geq 1068$.
\end{eg}

\begin{eg}[Estimating $\pi$ with Buffon's needle]
  Recall that if we randomly toss a needle of length $\ell$ to a floor marked with parallel lines a distance $L$ apart, the probability that the needle hits the line is $p = \frac{2\ell}{\pi L}$.
  \begin{center}
    \begin{tikzpicture}
      \draw (-3, -1) -- (3, -1);
      \draw (-3, 1) -- (3, 1);
      \draw (-2, -0.1) -- (-0.5, 0.7) node [anchor = south east, pos = 0.5] {$\ell$};
      \draw [->] (-2, -0.1) -- (-2, 1) node [left, pos = 0.5] {$X$};
      \draw [->] (-2, 1) -- (-2, -0.1);
      \draw [dashed] (-2.5, -0.1) -- (-0.5, -0.1);
      \draw (-1.5, -0.1) arc (0:29:0.5);
      \node at (-1.2, 0.1) {$\theta$};
      \draw [->] (2, -1) -- (2, 1) node [right, pos = 0.5] {$L$};
      \draw [->] (2, 1) -- (2, -1);
    \end{tikzpicture}
  \end{center}
  Suppose we toss the pin $n$ times, and it hits the line $N$ times. Then
  \[
    N\approx N(np, np(1 - p))
  \]
  by the Central limit theorem. Write $p'$ for the actual proportion observed. Then
  \begin{align*}
    \hat{\pi} &= \frac{2\ell}{(N/n) L} \\
    &= \frac{\pi 2\ell/(\pi L)}{p'}\\
    &= \frac{\pi p}{p + (p' - p)}\\
    &= \pi \left(1 - \frac{p' - p}{p} + \cdots \right)
  \end{align*}
  Hence
  \[
    \hat\pi - \pi \approx \frac{p - p'}{p}.
  \]
  We know
  \[
    p' \sim N\left(p, \frac{p(1 - p)}{n}\right).
  \]
  So we can find
  \[
    \hat \pi - \pi \sim N\left(0, \frac{\pi^2 p(1 - p)}{np^2}\right) = N\left(0, \frac{\pi^2(1 - p)}{np}\right)
  \]
  We want a small variance, and that occurs when $p$ is the largest. Since $p = 2\ell/\pi L$, this is maximized with $\ell = L$. In this case,
  \[
    p = \frac{2}{\pi},
  \]
  and
  \[
    \hat \pi - \pi \approx N\left(0, \frac{(\pi - 2)\pi^2}{2n}\right).
  \]
  If we want to estimate $\pi$ to 3 decimal places, then we need
  \[
    \P(|\hat \pi - \pi| \leq 0.001) \geq 0.95.
  \]
  This is true if and only if
  \[
    0.001\sqrt{\frac{2n}{(\pi - 2)(\pi^2)}} \geq \Phi^{-1}(0.975) = 1.96
  \]
  So $n\geq 2.16 \times 10^7$. So we can obtain $\pi$ to 3 decimal places just by throwing a stick 20 million times! Isn't that exciting?
\end{eg}

\ifx \nhtml \undefined
\begin{landscape}
\section{Summary of distributions}
\subsection{Discrete distributions}
\begin{center}
  \begin{tabular}{ccccc}
    \toprule
    \textbf{Distribution} & \textbf{PMF} & \textbf{Mean} & \textbf{Variance} & \textbf{PGF} \\
    \midrule
    Bernoulli & $p^k(1 - p)^{1 - k}$ & $p$ & $p(1 - p)$ & $q + pz$\\
    Binomial & $\displaystyle\binom{n}{k}p^k(1 - p)^{n - k}$ & $np$ & $np(1 - p)$ & $(q + pz)^n$\\
    Geometric & $(1 - p)^k p$ & $\displaystyle\frac{1 - p}{p}$ & $\displaystyle\frac{1 - p}{p^2}$ & $\displaystyle\frac{1 - p}{1 - pz}$\\
    Poisson & $\displaystyle\frac{\lambda^k}{k!}e^{-\lambda}$ & $\lambda$ & $\lambda$ & $e^{\lambda(z - 1)}$\\
    \bottomrule
  \end{tabular}
\end{center}
\subsection{Continuous distributions}
\begin{center}
  \begin{tabular}{cccccc}
    \toprule
    \textbf{Distribution} & \textbf{PDF} & \textbf{CDF} & \textbf{Mean} & \textbf{Variance} & \textbf{MGF} \\
    \midrule
    Uniform & $\displaystyle\frac{1}{b - a}$ & $\displaystyle\frac{x - a}{b - a}$ & $\displaystyle\frac{a + b}{2}$ & $\displaystyle \frac{1}{12}(b - a)^2$ & $\displaystyle\frac{e^{\theta b} - e^{\theta a}}{\theta (b - a)}$ \\
    Normal & $\displaystyle \frac{1}{\sqrt{2\pi}\sigma}\exp\left(-\frac{(x - \mu)^2}{2\sigma^2}\right)$ & / & $\mu$ & $\sigma^2$ & $e^{\theta\mu + \frac{1}{2}\theta^2\sigma^2}$\\
    Exponential & $\lambda e^{-\lambda x}$ & $1 - e^{-\lambda x}$ & $\displaystyle \frac{1}{\lambda}$ & $\displaystyle\frac{1}{\lambda^2}$ & $\displaystyle\frac{\lambda}{\lambda - \theta}$ \\
    Cauchy & $\displaystyle \frac{1}{\pi(1 + x^2)}$ & / & undefined & undefined & undefined\\
    Gamma & $\displaystyle \frac{\lambda^n x^{n - 1}e^{-\lambda x}}{(n - 1)!}$ & / & $\displaystyle\frac{n}{\lambda}$ & $\displaystyle \frac{n}{\lambda^2}$ & $\displaystyle\left(\frac{\lambda}{\lambda - \theta}\right)^n$ \\
    Multivariate normal & $\displaystyle \frac{1}{(2\pi)^{n/2}\sqrt{\det \Sigma}} \exp\left[-\frac{1}{2}(\mathbf{z} - \boldsymbol\mu)^T\Sigma^{-1}(\mathbf{z} - \boldsymbol\mu)\right]$ & / & $\boldsymbol\mu$ & $\Sigma$ & /\\
    \bottomrule
  \end{tabular}
\end{center}
\end{landscape}
\fi
\end{document}


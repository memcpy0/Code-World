\documentclass[a4paper]{article}

\def\npart {III}
\def\nterm {Lent}
\def\nyear {2018}
\def\nlecturer {S.\ Rasmussen}
\def\ncourse {3-Manifolds}

\RequirePackage{etex}
\makeatletter
\ifx \nauthor\undefined
  \def\nauthor{Dexter Chua}
\else
\fi

\author{Based on lectures by \nlecturer \\\small Notes taken by \nauthor}
\date{\nterm\ \nyear}

\usepackage{alltt}
\usepackage{amsfonts}
\usepackage{amsmath}
\usepackage{amssymb}
\usepackage{amsthm}
\usepackage{booktabs}
\usepackage{caption}
\usepackage{enumitem}
\usepackage{fancyhdr}
\usepackage{graphicx}
\usepackage{mathdots}
\usepackage{mathtools}
\usepackage{microtype}
\usepackage{multirow}
\usepackage{pdflscape}
\usepackage{pgfplots}
\usepackage{siunitx}
\usepackage{textcomp}
\usepackage{slashed}
\usepackage{tabularx}
\usepackage{tikz}
\usepackage{tkz-euclide}
\usepackage[normalem]{ulem}
\usepackage[all]{xy}
\usepackage{imakeidx}

\makeindex[intoc, title=Index]
\indexsetup{othercode={\lhead{\emph{Index}}}}

\ifx \nextra \undefined
  \usepackage[hidelinks,
    pdfauthor={Dexter Chua},
    pdfsubject={Cambridge Maths Notes: Part \npart\ - \ncourse},
    pdftitle={Part \npart\ - \ncourse},
  pdfkeywords={Cambridge Mathematics Maths Math \npart\ \nterm\ \nyear\ \ncourse}]{hyperref}
  \title{Part \npart\ --- \ncourse}
\else
  \usepackage[hidelinks,
    pdfauthor={Dexter Chua},
    pdfsubject={Cambridge Maths Notes: Part \npart\ - \ncourse\ (\nextra)},
    pdftitle={Part \npart\ - \ncourse\ (\nextra)},
  pdfkeywords={Cambridge Mathematics Maths Math \npart\ \nterm\ \nyear\ \ncourse\ \nextra}]{hyperref}

  \title{Part \npart\ --- \ncourse \\ {\Large \nextra}}
  \renewcommand\printindex{}
\fi

\pgfplotsset{compat=1.12}

\pagestyle{fancyplain}
\ifx \ncoursehead \undefined
\def\ncoursehead{\ncourse}
\fi

\lhead{\emph{\nouppercase{\leftmark}}}
\ifx \nextra \undefined
  \rhead{
    \ifnum\thepage=1
    \else
      \npart\ \ncoursehead
    \fi}
\else
  \rhead{
    \ifnum\thepage=1
    \else
      \npart\ \ncoursehead \ (\nextra)
    \fi}
\fi
\usetikzlibrary{arrows.meta}
\usetikzlibrary{decorations.markings}
\usetikzlibrary{decorations.pathmorphing}
\usetikzlibrary{positioning}
\usetikzlibrary{fadings}
\usetikzlibrary{intersections}
\usetikzlibrary{cd}

\newcommand*{\Cdot}{{\raisebox{-0.25ex}{\scalebox{1.5}{$\cdot$}}}}
\newcommand {\pd}[2][ ]{
  \ifx #1 { }
    \frac{\partial}{\partial #2}
  \else
    \frac{\partial^{#1}}{\partial #2^{#1}}
  \fi
}
\ifx \nhtml \undefined
\else
  \renewcommand\printindex{}
  \DisableLigatures[f]{family = *}
  \let\Contentsline\contentsline
  \renewcommand\contentsline[3]{\Contentsline{#1}{#2}{}}
  \renewcommand{\@dotsep}{10000}
  \newlength\currentparindent
  \setlength\currentparindent\parindent

  \newcommand\@minipagerestore{\setlength{\parindent}{\currentparindent}}
  \usepackage[active,tightpage,pdftex]{preview}
  \renewcommand{\PreviewBorder}{0.1cm}

  \newenvironment{stretchpage}%
  {\begin{preview}\begin{minipage}{\hsize}}%
    {\end{minipage}\end{preview}}
  \AtBeginDocument{\begin{stretchpage}}
  \AtEndDocument{\end{stretchpage}}

  \newcommand{\@@newpage}{\end{stretchpage}\begin{stretchpage}}

  \let\@real@section\section
  \renewcommand{\section}{\@@newpage\@real@section}
  \let\@real@subsection\subsection
  \renewcommand{\subsection}{\@ifstar{\@real@subsection*}{\@@newpage\@real@subsection}}
\fi
\ifx \ntrim \undefined
\else
  \usepackage{geometry}
  \geometry{
    papersize={379pt, 699pt},
    textwidth=345pt,
    textheight=596pt,
    left=17pt,
    top=54pt,
    right=17pt
  }
\fi

\ifx \nisofficial \undefined
\let\@real@maketitle\maketitle
\renewcommand{\maketitle}{\@real@maketitle\begin{center}\begin{minipage}[c]{0.9\textwidth}\centering\footnotesize These notes are not endorsed by the lecturers, and I have modified them (often significantly) after lectures. They are nowhere near accurate representations of what was actually lectured, and in particular, all errors are almost surely mine.\end{minipage}\end{center}}
\else
\fi

% Theorems
\theoremstyle{definition}
\newtheorem*{aim}{Aim}
\newtheorem*{axiom}{Axiom}
\newtheorem*{claim}{Claim}
\newtheorem*{cor}{Corollary}
\newtheorem*{conjecture}{Conjecture}
\newtheorem*{defi}{Definition}
\newtheorem*{eg}{Example}
\newtheorem*{ex}{Exercise}
\newtheorem*{fact}{Fact}
\newtheorem*{law}{Law}
\newtheorem*{lemma}{Lemma}
\newtheorem*{notation}{Notation}
\newtheorem*{prop}{Proposition}
\newtheorem*{question}{Question}
\newtheorem*{problem}{Problem}
\newtheorem*{rrule}{Rule}
\newtheorem*{thm}{Theorem}
\newtheorem*{assumption}{Assumption}

\newtheorem*{remark}{Remark}
\newtheorem*{warning}{Warning}
\newtheorem*{exercise}{Exercise}

\newtheorem{nthm}{Theorem}[section]
\newtheorem{nlemma}[nthm]{Lemma}
\newtheorem{nprop}[nthm]{Proposition}
\newtheorem{ncor}[nthm]{Corollary}


\renewcommand{\labelitemi}{--}
\renewcommand{\labelitemii}{$\circ$}
\renewcommand{\labelenumi}{(\roman{*})}

\let\stdsection\section
\renewcommand\section{\newpage\stdsection}

% Strike through
\def\st{\bgroup \ULdepth=-.55ex \ULset}


%%%%%%%%%%%%%%%%%%%%%%%%%
%%%%% Maths Symbols %%%%%
%%%%%%%%%%%%%%%%%%%%%%%%%

\let\U\relax
\let\C\relax
\let\G\relax

% Matrix groups
\newcommand{\GL}{\mathrm{GL}}
\newcommand{\Or}{\mathrm{O}}
\newcommand{\PGL}{\mathrm{PGL}}
\newcommand{\PSL}{\mathrm{PSL}}
\newcommand{\PSO}{\mathrm{PSO}}
\newcommand{\PSU}{\mathrm{PSU}}
\newcommand{\SL}{\mathrm{SL}}
\newcommand{\SO}{\mathrm{SO}}
\newcommand{\Spin}{\mathrm{Spin}}
\newcommand{\Sp}{\mathrm{Sp}}
\newcommand{\SU}{\mathrm{SU}}
\newcommand{\U}{\mathrm{U}}
\newcommand{\Mat}{\mathrm{Mat}}

% Matrix algebras
\newcommand{\gl}{\mathfrak{gl}}
\newcommand{\ort}{\mathfrak{o}}
\newcommand{\so}{\mathfrak{so}}
\newcommand{\su}{\mathfrak{su}}
\newcommand{\uu}{\mathfrak{u}}
\renewcommand{\sl}{\mathfrak{sl}}

% Special sets
\newcommand{\C}{\mathbb{C}}
\newcommand{\CP}{\mathbb{CP}}
\newcommand{\GG}{\mathbb{G}}
\newcommand{\N}{\mathbb{N}}
\newcommand{\Q}{\mathbb{Q}}
\newcommand{\R}{\mathbb{R}}
\newcommand{\RP}{\mathbb{RP}}
\newcommand{\T}{\mathbb{T}}
\newcommand{\Z}{\mathbb{Z}}
\renewcommand{\H}{\mathbb{H}}

% Brackets
\newcommand{\abs}[1]{\left\lvert #1\right\rvert}
\newcommand{\bket}[1]{\left\lvert #1\right\rangle}
\newcommand{\brak}[1]{\left\langle #1 \right\rvert}
\newcommand{\braket}[2]{\left\langle #1\middle\vert #2 \right\rangle}
\newcommand{\bra}{\langle}
\newcommand{\ket}{\rangle}
\newcommand{\norm}[1]{\left\lVert #1\right\rVert}
\newcommand{\normalorder}[1]{\mathop{:}\nolimits\!#1\!\mathop{:}\nolimits}
\newcommand{\tv}[1]{|#1|}
\renewcommand{\vec}[1]{\boldsymbol{\mathbf{#1}}}

% not-math
\newcommand{\bolds}[1]{{\bfseries #1}}
\newcommand{\cat}[1]{\mathsf{#1}}
\newcommand{\ph}{\,\cdot\,}
\newcommand{\term}[1]{\emph{#1}\index{#1}}
\newcommand{\phantomeq}{\hphantom{{}={}}}
% Probability
\DeclareMathOperator{\Bernoulli}{Bernoulli}
\DeclareMathOperator{\betaD}{beta}
\DeclareMathOperator{\bias}{bias}
\DeclareMathOperator{\binomial}{binomial}
\DeclareMathOperator{\corr}{corr}
\DeclareMathOperator{\cov}{cov}
\DeclareMathOperator{\gammaD}{gamma}
\DeclareMathOperator{\mse}{mse}
\DeclareMathOperator{\multinomial}{multinomial}
\DeclareMathOperator{\Poisson}{Poisson}
\DeclareMathOperator{\var}{var}
\newcommand{\E}{\mathbb{E}}
\newcommand{\Prob}{\mathbb{P}}

% Algebra
\DeclareMathOperator{\adj}{adj}
\DeclareMathOperator{\Ann}{Ann}
\DeclareMathOperator{\Aut}{Aut}
\DeclareMathOperator{\Char}{char}
\DeclareMathOperator{\disc}{disc}
\DeclareMathOperator{\dom}{dom}
\DeclareMathOperator{\fix}{fix}
\DeclareMathOperator{\Hom}{Hom}
\DeclareMathOperator{\id}{id}
\DeclareMathOperator{\image}{image}
\DeclareMathOperator{\im}{im}
\DeclareMathOperator{\re}{re}
\DeclareMathOperator{\tr}{tr}
\DeclareMathOperator{\Tr}{Tr}
\newcommand{\Bilin}{\mathrm{Bilin}}
\newcommand{\Frob}{\mathrm{Frob}}

% Others
\newcommand\ad{\mathrm{ad}}
\newcommand\Art{\mathrm{Art}}
\newcommand{\B}{\mathcal{B}}
\newcommand{\cU}{\mathcal{U}}
\newcommand{\Der}{\mathrm{Der}}
\newcommand{\D}{\mathrm{D}}
\newcommand{\dR}{\mathrm{dR}}
\newcommand{\exterior}{\mathchoice{{\textstyle\bigwedge}}{{\bigwedge}}{{\textstyle\wedge}}{{\scriptstyle\wedge}}}
\newcommand{\F}{\mathbb{F}}
\newcommand{\G}{\mathcal{G}}
\newcommand{\Gr}{\mathrm{Gr}}
\newcommand{\haut}{\mathrm{ht}}
\newcommand{\Hol}{\mathrm{Hol}}
\newcommand{\hol}{\mathfrak{hol}}
\newcommand{\Id}{\mathrm{Id}}
\newcommand{\lie}[1]{\mathfrak{#1}}
\newcommand{\op}{\mathrm{op}}
\newcommand{\Oc}{\mathcal{O}}
\newcommand{\pr}{\mathrm{pr}}
\newcommand{\Ps}{\mathcal{P}}
\newcommand{\pt}{\mathrm{pt}}
\newcommand{\qeq}{\mathrel{``{=}"}}
\newcommand{\Rs}{\mathcal{R}}
\newcommand{\Vect}{\mathrm{Vect}}
\newcommand{\wsto}{\stackrel{\mathrm{w}^*}{\to}}
\newcommand{\wt}{\mathrm{wt}}
\newcommand{\wto}{\stackrel{\mathrm{w}}{\to}}
\renewcommand{\d}{\mathrm{d}}
\renewcommand{\P}{\mathbb{P}}
%\renewcommand{\F}{\mathcal{F}}


\let\Im\relax
\let\Re\relax

\DeclareMathOperator{\area}{area}
\DeclareMathOperator{\card}{card}
\DeclareMathOperator{\ccl}{ccl}
\DeclareMathOperator{\ch}{ch}
\DeclareMathOperator{\cl}{cl}
\DeclareMathOperator{\cls}{\overline{\mathrm{span}}}
\DeclareMathOperator{\coker}{coker}
\DeclareMathOperator{\conv}{conv}
\DeclareMathOperator{\cosec}{cosec}
\DeclareMathOperator{\cosech}{cosech}
\DeclareMathOperator{\covol}{covol}
\DeclareMathOperator{\diag}{diag}
\DeclareMathOperator{\diam}{diam}
\DeclareMathOperator{\Diff}{Diff}
\DeclareMathOperator{\End}{End}
\DeclareMathOperator{\energy}{energy}
\DeclareMathOperator{\erfc}{erfc}
\DeclareMathOperator{\erf}{erf}
\DeclareMathOperator*{\esssup}{ess\,sup}
\DeclareMathOperator{\ev}{ev}
\DeclareMathOperator{\Ext}{Ext}
\DeclareMathOperator{\fst}{fst}
\DeclareMathOperator{\Fit}{Fit}
\DeclareMathOperator{\Frac}{Frac}
\DeclareMathOperator{\Gal}{Gal}
\DeclareMathOperator{\gr}{gr}
\DeclareMathOperator{\hcf}{hcf}
\DeclareMathOperator{\Im}{Im}
\DeclareMathOperator{\Ind}{Ind}
\DeclareMathOperator{\Int}{Int}
\DeclareMathOperator{\Isom}{Isom}
\DeclareMathOperator{\lcm}{lcm}
\DeclareMathOperator{\length}{length}
\DeclareMathOperator{\Lie}{Lie}
\DeclareMathOperator{\like}{like}
\DeclareMathOperator{\Lk}{Lk}
\DeclareMathOperator{\Maps}{Maps}
\DeclareMathOperator{\orb}{orb}
\DeclareMathOperator{\ord}{ord}
\DeclareMathOperator{\otp}{otp}
\DeclareMathOperator{\poly}{poly}
\DeclareMathOperator{\rank}{rank}
\DeclareMathOperator{\rel}{rel}
\DeclareMathOperator{\Rad}{Rad}
\DeclareMathOperator{\Re}{Re}
\DeclareMathOperator*{\res}{res}
\DeclareMathOperator{\Res}{Res}
\DeclareMathOperator{\Ric}{Ric}
\DeclareMathOperator{\rk}{rk}
\DeclareMathOperator{\Rees}{Rees}
\DeclareMathOperator{\Root}{Root}
\DeclareMathOperator{\sech}{sech}
\DeclareMathOperator{\sgn}{sgn}
\DeclareMathOperator{\snd}{snd}
\DeclareMathOperator{\Spec}{Spec}
\DeclareMathOperator{\spn}{span}
\DeclareMathOperator{\stab}{stab}
\DeclareMathOperator{\St}{St}
\DeclareMathOperator{\supp}{supp}
\DeclareMathOperator{\Syl}{Syl}
\DeclareMathOperator{\Sym}{Sym}
\DeclareMathOperator{\vol}{vol}

\pgfarrowsdeclarecombine{twolatex'}{twolatex'}{latex'}{latex'}{latex'}{latex'}
\tikzset{->/.style = {decoration={markings,
                                  mark=at position 1 with {\arrow[scale=2]{latex'}}},
                      postaction={decorate}}}
\tikzset{<-/.style = {decoration={markings,
                                  mark=at position 0 with {\arrowreversed[scale=2]{latex'}}},
                      postaction={decorate}}}
\tikzset{<->/.style = {decoration={markings,
                                   mark=at position 0 with {\arrowreversed[scale=2]{latex'}},
                                   mark=at position 1 with {\arrow[scale=2]{latex'}}},
                       postaction={decorate}}}
\tikzset{->-/.style = {decoration={markings,
                                   mark=at position #1 with {\arrow[scale=2]{latex'}}},
                       postaction={decorate}}}
\tikzset{-<-/.style = {decoration={markings,
                                   mark=at position #1 with {\arrowreversed[scale=2]{latex'}}},
                       postaction={decorate}}}
\tikzset{->>/.style = {decoration={markings,
                                  mark=at position 1 with {\arrow[scale=2]{latex'}}},
                      postaction={decorate}}}
\tikzset{<<-/.style = {decoration={markings,
                                  mark=at position 0 with {\arrowreversed[scale=2]{twolatex'}}},
                      postaction={decorate}}}
\tikzset{<<->>/.style = {decoration={markings,
                                   mark=at position 0 with {\arrowreversed[scale=2]{twolatex'}},
                                   mark=at position 1 with {\arrow[scale=2]{twolatex'}}},
                       postaction={decorate}}}
\tikzset{->>-/.style = {decoration={markings,
                                   mark=at position #1 with {\arrow[scale=2]{twolatex'}}},
                       postaction={decorate}}}
\tikzset{-<<-/.style = {decoration={markings,
                                   mark=at position #1 with {\arrowreversed[scale=2]{twolatex'}}},
                       postaction={decorate}}}

\tikzset{circ/.style = {fill, circle, inner sep = 0, minimum size = 3}}
\tikzset{scirc/.style = {fill, circle, inner sep = 0, minimum size = 1.5}}
\tikzset{mstate/.style={circle, draw, blue, text=black, minimum width=0.7cm}}

\tikzset{eqpic/.style={baseline={([yshift=-.5ex]current bounding box.center)}}}
\tikzset{commutative diagrams/.cd,cdmap/.style={/tikz/column 1/.append style={anchor=base east},/tikz/column 2/.append style={anchor=base west},row sep=tiny}}

\definecolor{mblue}{rgb}{0.2, 0.3, 0.8}
\definecolor{morange}{rgb}{1, 0.5, 0}
\definecolor{mgreen}{rgb}{0.1, 0.4, 0.2}
\definecolor{mred}{rgb}{0.5, 0, 0}

\def\drawcirculararc(#1,#2)(#3,#4)(#5,#6){%
    \pgfmathsetmacro\cA{(#1*#1+#2*#2-#3*#3-#4*#4)/2}%
    \pgfmathsetmacro\cB{(#1*#1+#2*#2-#5*#5-#6*#6)/2}%
    \pgfmathsetmacro\cy{(\cB*(#1-#3)-\cA*(#1-#5))/%
                        ((#2-#6)*(#1-#3)-(#2-#4)*(#1-#5))}%
    \pgfmathsetmacro\cx{(\cA-\cy*(#2-#4))/(#1-#3)}%
    \pgfmathsetmacro\cr{sqrt((#1-\cx)*(#1-\cx)+(#2-\cy)*(#2-\cy))}%
    \pgfmathsetmacro\cA{atan2(#2-\cy,#1-\cx)}%
    \pgfmathsetmacro\cB{atan2(#6-\cy,#5-\cx)}%
    \pgfmathparse{\cB<\cA}%
    \ifnum\pgfmathresult=1
        \pgfmathsetmacro\cB{\cB+360}%
    \fi
    \draw (#1,#2) arc (\cA:\cB:\cr);%
}
\newcommand\getCoord[3]{\newdimen{#1}\newdimen{#2}\pgfextractx{#1}{\pgfpointanchor{#3}{center}}\pgfextracty{#2}{\pgfpointanchor{#3}{center}}}

\newcommand\qedshift{\vspace{-17pt}}
\newcommand\fakeqed{\pushQED{\qed}\qedhere}

\def\Xint#1{\mathchoice
   {\XXint\displaystyle\textstyle{#1}}%
   {\XXint\textstyle\scriptstyle{#1}}%
   {\XXint\scriptstyle\scriptscriptstyle{#1}}%
   {\XXint\scriptscriptstyle\scriptscriptstyle{#1}}%
   \!\int}
\def\XXint#1#2#3{{\setbox0=\hbox{$#1{#2#3}{\int}$}
     \vcenter{\hbox{$#2#3$}}\kern-.5\wd0}}
\def\ddashint{\Xint=}
\def\dashint{\Xint-}

\newcommand\separator{{\centering\rule{2cm}{0.2pt}\vspace{2pt}\par}}

\newenvironment{own}{\color{gray!70!black}}{}

\newcommand\makecenter[1]{\raisebox{-0.5\height}{#1}}

\mathchardef\mdash="2D

\newenvironment{significant}{\begin{center}\begin{minipage}{0.9\textwidth}\centering\em}{\end{minipage}\end{center}}
\DeclareRobustCommand{\rvdots}{%
  \vbox{
    \baselineskip4\p@\lineskiplimit\z@
    \kern-\p@
    \hbox{.}\hbox{.}\hbox{.}
  }}
\DeclareRobustCommand\tph[3]{{\texorpdfstring{#1}{#2}}}
\makeatother

\DeclareMathOperator\Tor{Tor}

\begin{document}
\maketitle
{\small
\setlength{\parindent}{0em}
\setlength{\parskip}{1em}
This course aims to provide a survey of topics relevant to research in 3-manifold topology and geometry.
\begin{itemize}
 \item \textit{Knots and links.} Invariants of knots and links, including the Jones and Alexander polynomials. Categorification of invariants. Dehn filling and Dehn surgery.
 \item \textit{Geometrization and Hyperbolic geometry.} A survey, mostly without proof, of primary notions from geometrization --- which characterizes 3-manifold geometry and topology in terms of the fundamental group --- and hyperbolic geometry.
 \item \textit{3-manifold constructions.} Mapping tori, handle decompositions, Heegaard splittings, triangulations.
 \item \textit{Foliations.} Singular, Reeb, and taut foliations, with connections to fundamental group actions, topology, and geometry. Transverse foliations on Seifert fibred spaces.
\end{itemize}
\subsubsection*{Pre-requisites}
Part III Algebraic Topology and Part III Differential Geometry.
}
\tableofcontents

\section{Introduction}
In the 1900's, Poincar\'e asked the following question:
\begin{center}
  Does homology distinguish $S^3$ from other closed $3$-manifolds?
\end{center}
In general, for a closed orientable manifold, by the universal coefficients theorem for cohomology, we know
\[
  H^i(M) \cong \Hom(H_i(M), \Z) \oplus \Ext(H_{i - 1}(M), \Z),
\]
and $\Ext(H_{i - 1}(M), \Z)$ is just the torsion subgroup of $H_{i - 1}(M)$. In particular, we have
\[
  H^3(M) = H_3(M) \oplus \Tor(H_2(M)).
\]
But we know both $H^3(M)$ and $H_3(M)$ are $\Z$. So we know that $\Tor(H_2(M)) = 0$.

On the other hand, we have Poincar\'e duality
\[
  H_1(M) \cong H^2(M) \cong H_2(M) \oplus \Tor(H_1(M)).
\]
Thus, $H_2(M)$ is just the free bit of $H_1(M)$. The conclusion is that, for a $3$-manifold, the homology is completely determined by $H_1$.

The answer to Poincar\'e's original question is no! Poincar\'e constructed a ``\term{Poincar\'e sphere}'' $P_S$ with $H_1(P_S)$ but $P_S \not\cong S^3$.

Thus, we need a stronger invariant. Thus, he came up with the notion of the fundamental group $\pi_1(M)$. The next question is:
\begin{center}
  Does $\pi_1$ distinguish $S^3$?
\end{center}
We can test, and check that $|\pi_1(P_S)| = 120$, but $\pi_1(S^3)$ is trivial.

The answer of this question didn't come until 2002, when Perelman proved
\begin{thm}[Poincar\'e conjecture]\index{Poincar\'e conjecture}
  If $M$ is a $3$-manifold, $\pi_1(M)$ is finite, and $M$ is \emph{prime}, then the universal cover of $M$ is $S^3$.
\end{thm}

\begin{cor}
  If $\pi_1(M)$ is trivial, then $M \cong S^3$.
\end{cor}

%Instead of the Poincar\'e conjecture, we can consider the following simplification: We say the Poincar\'e conjecture is \emph{topologically true} if whenever $M$ is homotopy equivalent to $S^n$, then $M$ is homeomorphic to $S^n$.

It turns out in high dimensions, the Poincar\'e conjecture is easy. Via the $h$-cobordism theorem, Smale and Stallings proved that as long as $n \geq 5$, any simply-connected $n$-manifold with the homology of $S^n$ is in fact homeomorphic to $S^n$.
%For $n \geq 5$, Smale and Stallings proved that the Poincar\'e conjecture is topologically true, using the $h$-cobordism theorem. % check this

\begin{thm}[Freedman]
  The Poincar\'e conjecture is topologically true for $S^4$.
\end{thm}

This answers the topological question, namely whether manifolds are \emph{homeomorphic} to $S^n$. How about the differential structure?
\begin{thm}[Milnor]
  There exists exotic smooth structures on $S^7$, i.e.\ there exists different smooth structures on $S^7$ that are not diffeomorphic.
\end{thm}

On the other hand, there are no exotic smooth structures on $S^5$ and $S^6$, but there are for ``most'' $n \geq 7$.

It is also not hard to see that there are no exotic smooth structures on $S^1$ and $S^2$. In dimensions 3, we have
\begin{thm}[Morse]
  Two $3$-manifolds are homeomorphic iff they are diffeomorphic.
\end{thm}

For $S^4$, it is not known!

So how do we distinguish smooth structures on $4$-manifolds? The idea of Donaldson was to use an invariant. The idea was to pick some differential equation and count how many solutions there are on each manifold. If two manifolds are diffeomorphic, then they should have the same number of solutions! In particular, he took the equations of gauge theory from classical field theory in physics. Donaldson used $\SU(2)$ gauge theory, while Seiberg and Witten used $\U(1)$ gauge theory with spinor structures.

But why are we talking about $4$-manifolds? The point is that $4 = 3 + 1$. In particular, if we have a cobordism between $3$-manifolds, then we get a $4$-manifold! Often we can reduce gauge invariants on the total $4$-manifold into ``dimensionally reduced'' gauge invariants on the bounding $3$-manifolds.

In research today, there is a huge industry of studying gauge-theoretic invariants on $3$-manifolds. It is still unclear what these invariants can tell us about the $3$-manifolds themselves.

Now let's think about the following question --- what $3$-manifolds are there? For example, which groups are fundamental groups of $3$-manifolds? It turns out this question is uninteresting in higher dimensions, since any group can be the fundamental group of a $4$-manifold, or above.

Let's look at some examples.
\begin{defi}[Lens space]\index{lens space}
  A \emph{lens space} $L(p, q)$ is the quotient of $S^3 \subseteq \C^2$ by the relation
  \[
    (z_1, z_2) \sim (e^{2\pi i/p} z_1, e^{2\pi i q/p} z_2).
  \]
\end{defi}

\begin{prop}
  We have $\pi_1(L(p, q)) \cong \Z/p\Z$.
\end{prop}

On the other hand,
\begin{thm}[Reidemeister]
  $L(p, q) \cong L(p, q')$ iff $q \equiv \pm q'^{\pm 1} \pmod p$.
\end{thm}

So the fundamental group does not distinguish between these. But it turns out this is all that can go wrong.
\begin{thm}[Perelman]
  If no ``pieces'' of $M$ are lens spaces, then $\pi_1(M)$ completely determines $M$ up to homeomorphism.
\end{thm}

\subsection{Structures on 3-manifolds}
One can put knots and links on $3$-manifolds.
\begin{defi}[Knot]\index{knot}
  Let $M$ be a $3$-manifold. A \emph{knot} on $M$ is an embedding of $S^1$ into $M$.
\end{defi}
If $M$ is not specified, it is taken to be $S^3$.
\begin{defi}[Link]\index{link}
  Let $M$ be a $3$-manifold. A \emph{link} on $M$ is an embedding of some disjoint union of $S^1$'s into $M$.
\end{defi}
It turns out knots are useful for building $3$-manifolds:
\begin{thm}[Dehn]
  A $3$-manifold can be presented as surgery on a link in $S^3$.
\end{thm}

We can also put contact structures on $3$-manifolds, which we can think of as boundary conditions for symplectic structures, and there are also taut foliations.

\subsection{Prime decomposition}
Suppose we have a 3-manifold, and we want to try to simplify it. One way to do so is to try to write it as a \emph{connected sum} of simpler manifolds.

% insert definition of connected sum.

\begin{defi}[Prime manifold]\index{prime manifold}
  An $n$-dimensional manifold $M$ is \emph{prime} if whenever $M = A_1 \# A_2$, then one of the $A_i$ is $S^n$.
\end{defi}

One useful technique is the following:
\begin{thm}[Alexander's trick]
  If $\phi: S^2 \to S^2$ is a homeomorphism, then it extends to a homeomorphism $\phi: B^3 \to B^3$.
\end{thm}
This says there is a unique way to glue in a $B^3$.
\begin{proof}
  % fill this in
\end{proof}
This is a motivation to show that the direct sum gluing map along $S^2$ is unique up to homeomorphism.

\begin{thm}[Papakyriakopoulos $S^2$ theorem]\index{Papakyriakopoulos $S^2$ theorem}
  % insert theorem
\end{thm}

\begin{thm}[Knesser--Milnor theorem]\index{Knesser--Milnor theorem}

\end{thm}
To prove existence, the idea is that every time we break our manifold up into connected sums, the $\pi_2$ gets ``simpler'', and group theory helps us ensure the process will eventually stop.

\begin{thm}
  Any orientation-preserving diffeomorphism $\phi: S^2 \to S^2$ is \term{isotopic} to the identity $S^2 \to S^2$ via homeomorphisms.
\end{thm}

\begin{proof}
  We want to think of $S^2$ as the one-point compactification of $\R^2$. By composing with a rotation (which isotopic to the identity), we may assume $\phi$ fixes the point $\infty$. Thus, we may prove the same result with $S^2$ replaced by $\R^2$.

%  Pick some point $p \in S^2$. By composing $\phi$ with a rotation (which is isotopic to the identity), we may assume $\phi(p) = p$. By some rotation and stretching, we may further assume that $\phi$ is differentiable at $p$ and $\d_p \phi = \id$, using that $\phi$ is orientation-preserving. Then we use the fact that any homeomorphism $\psi: \R^n \to \R^n$ with $\psi(0) = 0$ and $\d_0 \psi = \id$ is isotopic to the identity on $\R^n$.
\end{proof}

% unique decomposition theorem

% more things

\begin{defi}[Splitting]\index{splitting}
  Given a closed surface $\Sigma$ and an embedding $\Sigma \hookrightarrow M$, the \emph{splitting} of $M$ along $\Sigma$ is $M | \Sigma = M \setminus \nu(\Sigma)$, where $\nu(\Sigma)$ is a tubular neighbourhood of $\Sigma$.
\end{defi}
The idea is that we just want to remove $\Sigma$, but it is more well-behaved if we remove an $3$-manifold from a $3$-manifold, as opposed to removing a $2$-manifold from a $3$-manifold.

\begin{defi}[Separating submanifold]\index{separating submanifold}
  A closed subsurface $\Sigma \hookrightarrow M$ of a connected manifold $M$ is \emph{separating} if $M|\Sigma$ has two components.
\end{defi}

Thus, a manifold $M$ is prime if it has no separating $S^2$-embeddings that does not bound a ball.

\begin{eg}
  $\{1\} \times S^2$ is not separating in $S^1 \times S^2$, but the equator is separating in $S^3$.
\end{eg}

\begin{defi}[Irreducible manifold]\index{irreducible manifold}
  A manifold $M$ is irreducible if it contains no $S^2$ that does not bounds a $B^3$.
\end{defi}
In fact, there is only one $3$-manifold that is prime but not irreducible.

\begin{prop}
  $S^2 \times S^1$ is the unique $3$-manifold that is prime but not irreducible.
\end{prop}

A hard theorem by Alexander's theorem says:
\begin{thm}[Alexander's theorem]
  Any embedded $S^2$ in $\R^3$ bounds a ball.
\end{thm}

\begin{cor}
  $S^3$ is prime and irreducible.
\end{cor}

What does decomposition along $S^2$'s do to the fundamental group? We will need the following basic properties of the fundamental group:
\begin{thm}[Product theorem]\index{product theorem}
  Let $X, Y$ be spaces. Then
  \[
    \pi_1(X, Y) \simeq \pi_1(X) \times \pi_1(Y).
  \]
\end{thm}

\begin{thm}[Seifert--van Kampen theorem]\index{Seifert--van Kampen theorem}
  Let $X, Y$ be $3$-manifolds. Then
  \[
    \pi_1(X_1 \# X_2) \cong \pi_1(X_1) * \pi_1(X_2).
  \]
\end{thm}
The main point here is that $S^2$ is simply connected.

\begin{prop}[Homotopy lifting property of covering spaces]\index{homotopy lifting property}
  If $X, Y$ are manifolds and $p: \tilde{Y} \to Y$ is a covering map, then for any map $\tilde{f}: X \to \tilde{Y}$ and any homotopy $f_t: X \to Y$ with $f_0 = f = p \circ \tilde{f}$, there is a unique lift $\tilde{f}_t: X \to Y$ of $f_t$ such that $p \circ \tilde{f}_t = f_t$.
\end{prop}

% maybe say something about the lifting criterion.

\begin{thm}[Grushko's theorem]\index{Grushko's theorem}
  If $G_1$ and $G_2$ are finitely-generated groups, then $\rank(G_1 * G_2) = \rank(G_1) + \rank(G_2)$, where the \term{rank} of $G$ is the minimal number of generators in a presentation of $G$.
\end{thm}

\begin{thm}[Poincar\'e conjecture]
  If $X$ is a closed, oriented $3$-manifold with $\pi_1(X) = 1$, then $X \cong S^3$.
\end{thm}

Together, these tell us if we keep decomposing a $3$-manifolds into direct sums, then it eventually terminates.

\begin{thm}[Kneser's conjecture/Kneser--Stalling theorem]\index{Kneser's conjecture}\index{Kneser--Stalling theorem}
  If $M$ is a compact, oriented, connected $3$-manifold, and $\pi_1(M)$ decomposes as $G_1 * G_2$, then there exists $M_i$ (compact, oriented, connected) such that $\pi_1(M_i) = G_i$ and $M \cong M_1 \# M_2$.
\end{thm}

\section{Surface decompositions}
\begin{thm}[Loop theorem]\index{Loop theorem}
  Let $M$ be a $3$-manifold, and $B$ a compact surface with an inclusion $i: B \hookrightarrow \partial M$. Suppose $N \lhd \pi_1(B)$ is a normal subgroup, and $N \not\supseteq \ker (i_*: \pi_1(\partial B) \to \pi_1(N))$. Then there is an embedding $f: (D^2, S^1) \to (M, B)$ such that $f: S^1 \to B$ represents a conjugacy class in $\pi_1(B)$ which is not in $N$.
\end{thm}

\begin{cor}[Dehn's lemma]\index{Dehn's lemma}
  Let $M$ be a $3$-manifold, and $f: (D^2, S^1) \to (M, \partial M)$ be a map that is an embedding on some collar neighbourhood $A$ of $\partial D$. Then there is an embedding $f': D^2 \hookrightarrow M$ such that $f|_A = f'|_A$.
\end{cor}
Dehn's lemma was stated before the loop theorem was proved. The loop theorem was proven by Papakyriakopoulos, and Dehn's lemma was later proved by Papakyriakopoulos and Stallings.

\begin{defi}[Compressing disk]
  Given a compact oriented surface $\Sigma \hookrightarrow M$ properly embedded, a \emph{compressing disk} is a proper embedding $\iota: (D^2, S^1) \hookrightarrow (M, \Sigma)$ such that $\iota(S^1)$ is essential in $\Sigma$.
\end{defi}

\begin{defi}[Incompressible $3$-manifold]\index{incompressible $3$-manifold}
  A $3$-manifold $M$ with boundary $\Sigma = \partial M$ is \emph{incompressible} if there are no compressing disks $D^2$ in $M$ with $\partial D \hookrightarrow \partial M$.
\end{defi}

\begin{defi}[Essential submanifold]\index{essential submanifold}
  An embedded surface of genus not $0$, $\Sigma \hookrightarrow M$ is \emph{essential} if it is $\pi_1$-injective. An embedding of $S^2$ is essential if it is $\pi_2$ injective. A curve is essential if it does not bound a disk.
\end{defi}

A corollary of the loop theorem and Dehn's lemma is
\begin{thm}
  If an embedded surface is essential, then it is incompressible.
\end{thm}
Under the appropriate hypothesis, this is an if-and-only-if.

\begin{thm}[Jaco--Shalen, Johannson] % check this
  If $M$ is a closed, oriented, irreducible, toroidal $3$-manifold, then there is a collection of disjoint embedded essential tori $T_1, \ldots, T_k \hookrightarrow M$ such that the splitting of $M$ along the tori consists of pieces that are Seifert fibered or ``simple'' (which Thurston showed is the same as hyperbolic).
\end{thm}
This is called a \term{JSJ decomposition}.

\begin{thm}[Morton Biowny Kirby] % fix this
  A codimension $1$ or $2$ topological embedding has a collar iff it is locally flat.
\end{thm}

\begin{defi}[Locally flat embedding]\index{locally flat embedding}
  An embedding $\iota: X \hookrightarrow Y$ is \emph{locally flat} if for every $x \in X$, there is a chart about $Y$ containing $x$ such that the embedding looks like $\R^k \hookrightarrow \R^n$.
\end{defi}

In this course, an embedding is always smooth or locally flat.

\begin{thm}
  If $X$ and $Y$ are compact, then $f: X \to Y$ is a proper embedding iff $f|_{\int(X)} \to \int(Y)$ is a proper map, and $f$ is an embedding.
\end{thm}

\section{Decomposition of 3-manifolds}
Suppose $X_1, X_2$ are compact, connected, oriented $n$-manifolds. There is a natural way of joining them together to get a new manifold.
\begin{defi}[Connected sum]\index{connected sum}
  Let $X_1, X_2$ are compact, connected, oriented $n$-manifolds. Pick two embeddings $\iota_i; B^n \hookrightarrow X_i$. We define the connected sum $X_1 \# X_2$ to be the union
  \[
    X_1 \# X_2 = (X_1 \setminus \iota_i(B^n)) \cup_\varphi(X_2 \setminus \iota(B^n)),
  \]
  where $\varphi$ is an orientation \emph{reversing} map identifying the two boundaries of $X_1$ and $X_2$.
\end{defi}

It is a hard theorem that this is well-defined, and we can perform this in both the topological category and the smooth category.

We now want to try to decompose our manifolds under this operation.
\begin{defi}[Prime manifold]\index{prime manifold}
  A manifold $M$ is prime if whenever $M \cong M_1 \# M_2$, one of $M_1$ and $M_2$ is $S^n$.
\end{defi}

Our objective is to decompose every manifold into a sum of prime manifolds. To do so, a reasonable strategy is to embed spheres $S^{n - 1}$ into $M$, and try to cut $M$ out along this $S^{n - 1}$. If this $S^{n - 1}$ bounds a ball, then we would have cut out a decomposition where one piece is $S^n$. This leads to the definition:
\begin{defi}[Irreducible manifold]\index{irreducible manifold}
  A manifold $M$ is irreducible if every embedding of $S^{n - 1}$ bounds a ball.
\end{defi}

Note that being irreducible is not the same as being prime! For example, $S^2 \times S^1$ is prime, but is not irreducible, as the obvious embedding of $S^2$ does not bound a ball. The problem is that removing the copy of $S^2$ does not necessarily break your manifold into two pieces.

We now focus on the case $n = 3$. We then have
\begin{thm}[Kneser--Milnor theorem]\index{Kneser--Milnor theorem}
  Every $3$-manifold admits a decomposition into a connected sum of prime manifolds, none of which is $S^3$, and the decomposition is unique up to permutation.
\end{thm}

This is not easy to prove, and falls out of the scope of the course, but we shall indicate how this follows from the Poincar\'e conjecture as well as some group-theoretic results.

First of all, by Seifert--van Kampen, we have
\begin{prop}
  Let $M = M_1 \# M_2$. Then
  \[
    \pi_1(M) \cong \pi_1(M_1) * \pi_2(M_2).
  \]
\end{prop}
The idea is that when this happens, $\pi_1(M_i)$ are simpler than $\pi_1(M)$. So if we keep decomposing, it must stop. To make this precise, recall that the \term{rank} of a group is the minimal number of generators. We then have

\begin{thm}[Grushko's theorem]\index{Grushko's theorem}
  If $G_1$ and $G_2$ are finitely-generated groups, then $\rank(G_1 * G_2) = \rank(G_1) + \rank(G_2)$, where the \term{rank} of $G$ is the minimal number of generators in a presentation of $G$.
\end{thm}

Again, we are not going to prove this. To conclude the Kneser--Milnor theorem, we use
\begin{thm}[Poincar\'e conjecture]\index{Poincar\'e conjecture}
  Every $3$-manifold with trivial $\pi_1$ is homeomorphic to $S^3$.
\end{thm}

From these, the Kneser--Milnor theorem follows.

\section{The mapping class group}
If we want to glue $3$-manifolds along a fixed surface $\Sigma$, then we need to pick a homeomorphism that identifies the boundaries of the manifolds. This corresponds to understanding the homeomorphisms from $\Sigma$ to itself. This suggests studying
\begin{defi}[Mapping class group]\index{mapping class group}
  Let $\Sigma$ be a surface. The mapping class group $MCG(\Sigma)$ of $\Sigma$ is the group of homeomorphisms $\Sigma \to \Sigma$ quotiented by isotopy.
\end{defi}

In general, isotopy is not very amenable to the techniques of algebraic topology. Instead, we want to work with homotopy. A perhaps surprising theorem of Baer tells us this doesn't matter.
\begin{thm}[Baer's theorem]\index{Baer's theorem}
  Two homeomorphisms $f_1, f_2: \Sigma \to \Sigma$ of surfaces are isotopic iff they are homeomorphic.
\end{thm}
Thus, we can define $MCG(\Sigma)$ to be the $\pi_0$ of the topological group of self-homeomorphisms of $\Sigma$.

We can further strengthen the theorem, to say
\begin{thm}
  $f: \Sigma \to \Sigma$ are classified by their induced map on $\pi_1$. Thus, we have an embedding $MCG(\Sigma) \hookrightarrow \Aut(\pi_1(\Sigma))$. % quotient by conjugation?
\end{thm}

\begin{thm}[Dehn] % check this
  This map is also surjective if $\Sigma$ is closed.
\end{thm}

This has a very explicit descriptions in the case where $\Sigma$ is a torus, which is probably an exercise in Part II Algebraic Topology. In this case, $\Aut(\Z \oplus \Z) \cong \SL(2, \Z)$ is generated by $\begin{pmatrix}1 & -1\\ 0 & 1\end{pmatrix}$ and $\begin{pmatrix}1 & 0\\1 & 1\end{pmatrix}$, the \term{Dehn twist}. % fix this


In general, if we have any surface, it has many holes, and we can do a Dehn twist on any embedded $S^1$.
\begin{thm}[Dehn, Lickorish, Humphries, Johnson]
  The mapping class groups are generated by $2g + 1$ Dehn twists. % insert picture
\end{thm}

We can focus on the case of a torus, where the mapping class group is $\SL(2, \Z)$. Let $M$ be an element in $\SL(2, \Z)$. Then it has eigenvalues $\lambda, \lambda^{-1}$. There are a few possible cases:
\begin{itemize}
  \item Elliptic: If $\lambda, \lambda^{-1}$ are complex, then $\lambda^{-1} = \lambda^*$. One example is $\begin{pmatrix}0 & 1\\ -1 & 1\end{pmatrix}$.
  \item Parabolic: If $\lambda = \lambda^{-1} = 1$, then this is elliptic. These include $\begin{pmatrix}1 & n\\0 & 1\end{pmatrix}$.
  \item Hyperbolic: These are the other cases, e.g.\ $\begin{pmatrix}2 & 1\\ 1 & 1\end{pmatrix}$.
\end{itemize}

Given a homeomorphism $M: T^2 \to T^2$, we can form the \term{mapping torus} given by taking $\Sigma \times [0, 1]$ and then quotienting $(M(x), 0) \sim (x, 1)$. % Seifert fibration

\subsection{JSJ decomposition}
\begin{defi}[Boundary parallel]\index{boundary parallel}
  An embedded surface $\Sigma \subseteq M$ is \emph{boundary parallel} if it is isotopic to a boundary component of $M$.
\end{defi}

\begin{defi}[Peripheral subgroup]\index{peripheral subgroup}
  A \emph{peripheral subgroup} of $\pi(X)$ is a subgroup of $\pi_1(X)$ that is conjugate to the inclusion of a boundary component of $X$.
\end{defi}

\begin{defi}[Geometrically toroidal]\index{geometrically toroidal}
  A $3$-manifold $M$ is geometrically toroidal if $M$has an embedded incompressible torus that is not boundary parallel.
\end{defi}

\begin{defi}[Group-theoretically atoroidal]\index{group-theoretically atoroidal}
  A manifold $M$ is \emph{group-theoretically atoroidal} if $\pi_1(M)$ has no non-peripheral $\Z \oplus \Z$ subgroup.
\end{defi}

\begin{thm}[Apanasov]
  If $M$ is compact, irreducible, oriented, boundary incompressible (i.e.\ the boundary does not have any compressing disks), then it is geometrically toroidal iff it is geometrically toroidal iff it is group-theoretically toroidal.
\end{thm}
In these cases, we just say $M$ is \term{toroidal}.

\begin{thm}[JSJ decomposition]\index{JSJ decomposition}
  If $M$ is compact, irreducible, oriented with \term{toroidal boundary} (i.e.\ the boundary is a disjoint union of tori), then there is a (possibly empty) disjoint union $\{T_1^2, \ldots, T_n^2\}$ of embedding incompressible tori, such that $M \setminus \{T_1^2 \cup \cdots \cup T_n^2\}$ such that all the pieces are either Seifert fibered or atoroidal (it could be both).
\end{thm}

\begin{defi}[Haken manifold]\index{Haken manifold}
  A connected, compact, oriented $3$-manifold $M$ is \emph{Haken} if it has a properly embedded, incompressible surface that is not boundary parallel.
\end{defi}

\begin{thm}[Seifert]
  Any compact oriented $3$-manifold with toroidal boundary has a Seifert surface.
\end{thm}

\begin{cor}
  Such a manifold is Haken.
\end{cor}

Why do we care about Haken things?
\begin{thm}[Thurston's hyperbolization theorem]
  % insert correct statement
\end{thm}

\begin{cor}
  In the case of non-empty splitting, the non-Seifert-fibered things are hyperbolic.
\end{cor}

How about the closed cased? These are either Seifert fibered or atoroidal closed. The closed atoroidal case is hard.
\begin{thm}[Perelman, 2002--2003]
  If I have an irreducible, oriented, atoroidal closed manifold, then I can Ricci flow with surgery to a constant curvature metric. If the curvature is finite, then the space has finite topological subgroup, which is a rigid subgroup of $\SO(4)$, acting on $S^3 \subseteq \R^4$, and the universal cover is $S^3$.

  In the case of negative curvature, we are hyperbolic.
\end{thm}


\printindex
\end{document}

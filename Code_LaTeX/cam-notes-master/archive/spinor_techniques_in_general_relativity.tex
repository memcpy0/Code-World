\documentclass[a4paper]{article}

\def\npart {IV}
\def\nterm {Lent}
\def\nyear {2018}
\def\nlecturer {I.\ M.\ M.\ Borzym and P.\ O'Donnell}
\def\ncourse {Spinor Techniques in General Relativity}
\def\ncoursehead {Spinor Techniques in GR}

\RequirePackage{etex}
\makeatletter
\ifx \nauthor\undefined
  \def\nauthor{Dexter Chua}
\else
\fi

\author{Based on lectures by \nlecturer \\\small Notes taken by \nauthor}
\date{\nterm\ \nyear}

\usepackage{alltt}
\usepackage{amsfonts}
\usepackage{amsmath}
\usepackage{amssymb}
\usepackage{amsthm}
\usepackage{booktabs}
\usepackage{caption}
\usepackage{enumitem}
\usepackage{fancyhdr}
\usepackage{graphicx}
\usepackage{mathdots}
\usepackage{mathtools}
\usepackage{microtype}
\usepackage{multirow}
\usepackage{pdflscape}
\usepackage{pgfplots}
\usepackage{siunitx}
\usepackage{textcomp}
\usepackage{slashed}
\usepackage{tabularx}
\usepackage{tikz}
\usepackage{tkz-euclide}
\usepackage[normalem]{ulem}
\usepackage[all]{xy}
\usepackage{imakeidx}

\makeindex[intoc, title=Index]
\indexsetup{othercode={\lhead{\emph{Index}}}}

\ifx \nextra \undefined
  \usepackage[hidelinks,
    pdfauthor={Dexter Chua},
    pdfsubject={Cambridge Maths Notes: Part \npart\ - \ncourse},
    pdftitle={Part \npart\ - \ncourse},
  pdfkeywords={Cambridge Mathematics Maths Math \npart\ \nterm\ \nyear\ \ncourse}]{hyperref}
  \title{Part \npart\ --- \ncourse}
\else
  \usepackage[hidelinks,
    pdfauthor={Dexter Chua},
    pdfsubject={Cambridge Maths Notes: Part \npart\ - \ncourse\ (\nextra)},
    pdftitle={Part \npart\ - \ncourse\ (\nextra)},
  pdfkeywords={Cambridge Mathematics Maths Math \npart\ \nterm\ \nyear\ \ncourse\ \nextra}]{hyperref}

  \title{Part \npart\ --- \ncourse \\ {\Large \nextra}}
  \renewcommand\printindex{}
\fi

\pgfplotsset{compat=1.12}

\pagestyle{fancyplain}
\ifx \ncoursehead \undefined
\def\ncoursehead{\ncourse}
\fi

\lhead{\emph{\nouppercase{\leftmark}}}
\ifx \nextra \undefined
  \rhead{
    \ifnum\thepage=1
    \else
      \npart\ \ncoursehead
    \fi}
\else
  \rhead{
    \ifnum\thepage=1
    \else
      \npart\ \ncoursehead \ (\nextra)
    \fi}
\fi
\usetikzlibrary{arrows.meta}
\usetikzlibrary{decorations.markings}
\usetikzlibrary{decorations.pathmorphing}
\usetikzlibrary{positioning}
\usetikzlibrary{fadings}
\usetikzlibrary{intersections}
\usetikzlibrary{cd}

\newcommand*{\Cdot}{{\raisebox{-0.25ex}{\scalebox{1.5}{$\cdot$}}}}
\newcommand {\pd}[2][ ]{
  \ifx #1 { }
    \frac{\partial}{\partial #2}
  \else
    \frac{\partial^{#1}}{\partial #2^{#1}}
  \fi
}
\ifx \nhtml \undefined
\else
  \renewcommand\printindex{}
  \DisableLigatures[f]{family = *}
  \let\Contentsline\contentsline
  \renewcommand\contentsline[3]{\Contentsline{#1}{#2}{}}
  \renewcommand{\@dotsep}{10000}
  \newlength\currentparindent
  \setlength\currentparindent\parindent

  \newcommand\@minipagerestore{\setlength{\parindent}{\currentparindent}}
  \usepackage[active,tightpage,pdftex]{preview}
  \renewcommand{\PreviewBorder}{0.1cm}

  \newenvironment{stretchpage}%
  {\begin{preview}\begin{minipage}{\hsize}}%
    {\end{minipage}\end{preview}}
  \AtBeginDocument{\begin{stretchpage}}
  \AtEndDocument{\end{stretchpage}}

  \newcommand{\@@newpage}{\end{stretchpage}\begin{stretchpage}}

  \let\@real@section\section
  \renewcommand{\section}{\@@newpage\@real@section}
  \let\@real@subsection\subsection
  \renewcommand{\subsection}{\@ifstar{\@real@subsection*}{\@@newpage\@real@subsection}}
\fi
\ifx \ntrim \undefined
\else
  \usepackage{geometry}
  \geometry{
    papersize={379pt, 699pt},
    textwidth=345pt,
    textheight=596pt,
    left=17pt,
    top=54pt,
    right=17pt
  }
\fi

\ifx \nisofficial \undefined
\let\@real@maketitle\maketitle
\renewcommand{\maketitle}{\@real@maketitle\begin{center}\begin{minipage}[c]{0.9\textwidth}\centering\footnotesize These notes are not endorsed by the lecturers, and I have modified them (often significantly) after lectures. They are nowhere near accurate representations of what was actually lectured, and in particular, all errors are almost surely mine.\end{minipage}\end{center}}
\else
\fi

% Theorems
\theoremstyle{definition}
\newtheorem*{aim}{Aim}
\newtheorem*{axiom}{Axiom}
\newtheorem*{claim}{Claim}
\newtheorem*{cor}{Corollary}
\newtheorem*{conjecture}{Conjecture}
\newtheorem*{defi}{Definition}
\newtheorem*{eg}{Example}
\newtheorem*{ex}{Exercise}
\newtheorem*{fact}{Fact}
\newtheorem*{law}{Law}
\newtheorem*{lemma}{Lemma}
\newtheorem*{notation}{Notation}
\newtheorem*{prop}{Proposition}
\newtheorem*{question}{Question}
\newtheorem*{problem}{Problem}
\newtheorem*{rrule}{Rule}
\newtheorem*{thm}{Theorem}
\newtheorem*{assumption}{Assumption}

\newtheorem*{remark}{Remark}
\newtheorem*{warning}{Warning}
\newtheorem*{exercise}{Exercise}

\newtheorem{nthm}{Theorem}[section]
\newtheorem{nlemma}[nthm]{Lemma}
\newtheorem{nprop}[nthm]{Proposition}
\newtheorem{ncor}[nthm]{Corollary}


\renewcommand{\labelitemi}{--}
\renewcommand{\labelitemii}{$\circ$}
\renewcommand{\labelenumi}{(\roman{*})}

\let\stdsection\section
\renewcommand\section{\newpage\stdsection}

% Strike through
\def\st{\bgroup \ULdepth=-.55ex \ULset}


%%%%%%%%%%%%%%%%%%%%%%%%%
%%%%% Maths Symbols %%%%%
%%%%%%%%%%%%%%%%%%%%%%%%%

\let\U\relax
\let\C\relax
\let\G\relax

% Matrix groups
\newcommand{\GL}{\mathrm{GL}}
\newcommand{\Or}{\mathrm{O}}
\newcommand{\PGL}{\mathrm{PGL}}
\newcommand{\PSL}{\mathrm{PSL}}
\newcommand{\PSO}{\mathrm{PSO}}
\newcommand{\PSU}{\mathrm{PSU}}
\newcommand{\SL}{\mathrm{SL}}
\newcommand{\SO}{\mathrm{SO}}
\newcommand{\Spin}{\mathrm{Spin}}
\newcommand{\Sp}{\mathrm{Sp}}
\newcommand{\SU}{\mathrm{SU}}
\newcommand{\U}{\mathrm{U}}
\newcommand{\Mat}{\mathrm{Mat}}

% Matrix algebras
\newcommand{\gl}{\mathfrak{gl}}
\newcommand{\ort}{\mathfrak{o}}
\newcommand{\so}{\mathfrak{so}}
\newcommand{\su}{\mathfrak{su}}
\newcommand{\uu}{\mathfrak{u}}
\renewcommand{\sl}{\mathfrak{sl}}

% Special sets
\newcommand{\C}{\mathbb{C}}
\newcommand{\CP}{\mathbb{CP}}
\newcommand{\GG}{\mathbb{G}}
\newcommand{\N}{\mathbb{N}}
\newcommand{\Q}{\mathbb{Q}}
\newcommand{\R}{\mathbb{R}}
\newcommand{\RP}{\mathbb{RP}}
\newcommand{\T}{\mathbb{T}}
\newcommand{\Z}{\mathbb{Z}}
\renewcommand{\H}{\mathbb{H}}

% Brackets
\newcommand{\abs}[1]{\left\lvert #1\right\rvert}
\newcommand{\bket}[1]{\left\lvert #1\right\rangle}
\newcommand{\brak}[1]{\left\langle #1 \right\rvert}
\newcommand{\braket}[2]{\left\langle #1\middle\vert #2 \right\rangle}
\newcommand{\bra}{\langle}
\newcommand{\ket}{\rangle}
\newcommand{\norm}[1]{\left\lVert #1\right\rVert}
\newcommand{\normalorder}[1]{\mathop{:}\nolimits\!#1\!\mathop{:}\nolimits}
\newcommand{\tv}[1]{|#1|}
\renewcommand{\vec}[1]{\boldsymbol{\mathbf{#1}}}

% not-math
\newcommand{\bolds}[1]{{\bfseries #1}}
\newcommand{\cat}[1]{\mathsf{#1}}
\newcommand{\ph}{\,\cdot\,}
\newcommand{\term}[1]{\emph{#1}\index{#1}}
\newcommand{\phantomeq}{\hphantom{{}={}}}
% Probability
\DeclareMathOperator{\Bernoulli}{Bernoulli}
\DeclareMathOperator{\betaD}{beta}
\DeclareMathOperator{\bias}{bias}
\DeclareMathOperator{\binomial}{binomial}
\DeclareMathOperator{\corr}{corr}
\DeclareMathOperator{\cov}{cov}
\DeclareMathOperator{\gammaD}{gamma}
\DeclareMathOperator{\mse}{mse}
\DeclareMathOperator{\multinomial}{multinomial}
\DeclareMathOperator{\Poisson}{Poisson}
\DeclareMathOperator{\var}{var}
\newcommand{\E}{\mathbb{E}}
\newcommand{\Prob}{\mathbb{P}}

% Algebra
\DeclareMathOperator{\adj}{adj}
\DeclareMathOperator{\Ann}{Ann}
\DeclareMathOperator{\Aut}{Aut}
\DeclareMathOperator{\Char}{char}
\DeclareMathOperator{\disc}{disc}
\DeclareMathOperator{\dom}{dom}
\DeclareMathOperator{\fix}{fix}
\DeclareMathOperator{\Hom}{Hom}
\DeclareMathOperator{\id}{id}
\DeclareMathOperator{\image}{image}
\DeclareMathOperator{\im}{im}
\DeclareMathOperator{\re}{re}
\DeclareMathOperator{\tr}{tr}
\DeclareMathOperator{\Tr}{Tr}
\newcommand{\Bilin}{\mathrm{Bilin}}
\newcommand{\Frob}{\mathrm{Frob}}

% Others
\newcommand\ad{\mathrm{ad}}
\newcommand\Art{\mathrm{Art}}
\newcommand{\B}{\mathcal{B}}
\newcommand{\cU}{\mathcal{U}}
\newcommand{\Der}{\mathrm{Der}}
\newcommand{\D}{\mathrm{D}}
\newcommand{\dR}{\mathrm{dR}}
\newcommand{\exterior}{\mathchoice{{\textstyle\bigwedge}}{{\bigwedge}}{{\textstyle\wedge}}{{\scriptstyle\wedge}}}
\newcommand{\F}{\mathbb{F}}
\newcommand{\G}{\mathcal{G}}
\newcommand{\Gr}{\mathrm{Gr}}
\newcommand{\haut}{\mathrm{ht}}
\newcommand{\Hol}{\mathrm{Hol}}
\newcommand{\hol}{\mathfrak{hol}}
\newcommand{\Id}{\mathrm{Id}}
\newcommand{\lie}[1]{\mathfrak{#1}}
\newcommand{\op}{\mathrm{op}}
\newcommand{\Oc}{\mathcal{O}}
\newcommand{\pr}{\mathrm{pr}}
\newcommand{\Ps}{\mathcal{P}}
\newcommand{\pt}{\mathrm{pt}}
\newcommand{\qeq}{\mathrel{``{=}"}}
\newcommand{\Rs}{\mathcal{R}}
\newcommand{\Vect}{\mathrm{Vect}}
\newcommand{\wsto}{\stackrel{\mathrm{w}^*}{\to}}
\newcommand{\wt}{\mathrm{wt}}
\newcommand{\wto}{\stackrel{\mathrm{w}}{\to}}
\renewcommand{\d}{\mathrm{d}}
\renewcommand{\P}{\mathbb{P}}
%\renewcommand{\F}{\mathcal{F}}


\let\Im\relax
\let\Re\relax

\DeclareMathOperator{\area}{area}
\DeclareMathOperator{\card}{card}
\DeclareMathOperator{\ccl}{ccl}
\DeclareMathOperator{\ch}{ch}
\DeclareMathOperator{\cl}{cl}
\DeclareMathOperator{\cls}{\overline{\mathrm{span}}}
\DeclareMathOperator{\coker}{coker}
\DeclareMathOperator{\conv}{conv}
\DeclareMathOperator{\cosec}{cosec}
\DeclareMathOperator{\cosech}{cosech}
\DeclareMathOperator{\covol}{covol}
\DeclareMathOperator{\diag}{diag}
\DeclareMathOperator{\diam}{diam}
\DeclareMathOperator{\Diff}{Diff}
\DeclareMathOperator{\End}{End}
\DeclareMathOperator{\energy}{energy}
\DeclareMathOperator{\erfc}{erfc}
\DeclareMathOperator{\erf}{erf}
\DeclareMathOperator*{\esssup}{ess\,sup}
\DeclareMathOperator{\ev}{ev}
\DeclareMathOperator{\Ext}{Ext}
\DeclareMathOperator{\fst}{fst}
\DeclareMathOperator{\Fit}{Fit}
\DeclareMathOperator{\Frac}{Frac}
\DeclareMathOperator{\Gal}{Gal}
\DeclareMathOperator{\gr}{gr}
\DeclareMathOperator{\hcf}{hcf}
\DeclareMathOperator{\Im}{Im}
\DeclareMathOperator{\Ind}{Ind}
\DeclareMathOperator{\Int}{Int}
\DeclareMathOperator{\Isom}{Isom}
\DeclareMathOperator{\lcm}{lcm}
\DeclareMathOperator{\length}{length}
\DeclareMathOperator{\Lie}{Lie}
\DeclareMathOperator{\like}{like}
\DeclareMathOperator{\Lk}{Lk}
\DeclareMathOperator{\Maps}{Maps}
\DeclareMathOperator{\orb}{orb}
\DeclareMathOperator{\ord}{ord}
\DeclareMathOperator{\otp}{otp}
\DeclareMathOperator{\poly}{poly}
\DeclareMathOperator{\rank}{rank}
\DeclareMathOperator{\rel}{rel}
\DeclareMathOperator{\Rad}{Rad}
\DeclareMathOperator{\Re}{Re}
\DeclareMathOperator*{\res}{res}
\DeclareMathOperator{\Res}{Res}
\DeclareMathOperator{\Ric}{Ric}
\DeclareMathOperator{\rk}{rk}
\DeclareMathOperator{\Rees}{Rees}
\DeclareMathOperator{\Root}{Root}
\DeclareMathOperator{\sech}{sech}
\DeclareMathOperator{\sgn}{sgn}
\DeclareMathOperator{\snd}{snd}
\DeclareMathOperator{\Spec}{Spec}
\DeclareMathOperator{\spn}{span}
\DeclareMathOperator{\stab}{stab}
\DeclareMathOperator{\St}{St}
\DeclareMathOperator{\supp}{supp}
\DeclareMathOperator{\Syl}{Syl}
\DeclareMathOperator{\Sym}{Sym}
\DeclareMathOperator{\vol}{vol}

\pgfarrowsdeclarecombine{twolatex'}{twolatex'}{latex'}{latex'}{latex'}{latex'}
\tikzset{->/.style = {decoration={markings,
                                  mark=at position 1 with {\arrow[scale=2]{latex'}}},
                      postaction={decorate}}}
\tikzset{<-/.style = {decoration={markings,
                                  mark=at position 0 with {\arrowreversed[scale=2]{latex'}}},
                      postaction={decorate}}}
\tikzset{<->/.style = {decoration={markings,
                                   mark=at position 0 with {\arrowreversed[scale=2]{latex'}},
                                   mark=at position 1 with {\arrow[scale=2]{latex'}}},
                       postaction={decorate}}}
\tikzset{->-/.style = {decoration={markings,
                                   mark=at position #1 with {\arrow[scale=2]{latex'}}},
                       postaction={decorate}}}
\tikzset{-<-/.style = {decoration={markings,
                                   mark=at position #1 with {\arrowreversed[scale=2]{latex'}}},
                       postaction={decorate}}}
\tikzset{->>/.style = {decoration={markings,
                                  mark=at position 1 with {\arrow[scale=2]{latex'}}},
                      postaction={decorate}}}
\tikzset{<<-/.style = {decoration={markings,
                                  mark=at position 0 with {\arrowreversed[scale=2]{twolatex'}}},
                      postaction={decorate}}}
\tikzset{<<->>/.style = {decoration={markings,
                                   mark=at position 0 with {\arrowreversed[scale=2]{twolatex'}},
                                   mark=at position 1 with {\arrow[scale=2]{twolatex'}}},
                       postaction={decorate}}}
\tikzset{->>-/.style = {decoration={markings,
                                   mark=at position #1 with {\arrow[scale=2]{twolatex'}}},
                       postaction={decorate}}}
\tikzset{-<<-/.style = {decoration={markings,
                                   mark=at position #1 with {\arrowreversed[scale=2]{twolatex'}}},
                       postaction={decorate}}}

\tikzset{circ/.style = {fill, circle, inner sep = 0, minimum size = 3}}
\tikzset{scirc/.style = {fill, circle, inner sep = 0, minimum size = 1.5}}
\tikzset{mstate/.style={circle, draw, blue, text=black, minimum width=0.7cm}}

\tikzset{eqpic/.style={baseline={([yshift=-.5ex]current bounding box.center)}}}
\tikzset{commutative diagrams/.cd,cdmap/.style={/tikz/column 1/.append style={anchor=base east},/tikz/column 2/.append style={anchor=base west},row sep=tiny}}

\definecolor{mblue}{rgb}{0.2, 0.3, 0.8}
\definecolor{morange}{rgb}{1, 0.5, 0}
\definecolor{mgreen}{rgb}{0.1, 0.4, 0.2}
\definecolor{mred}{rgb}{0.5, 0, 0}

\def\drawcirculararc(#1,#2)(#3,#4)(#5,#6){%
    \pgfmathsetmacro\cA{(#1*#1+#2*#2-#3*#3-#4*#4)/2}%
    \pgfmathsetmacro\cB{(#1*#1+#2*#2-#5*#5-#6*#6)/2}%
    \pgfmathsetmacro\cy{(\cB*(#1-#3)-\cA*(#1-#5))/%
                        ((#2-#6)*(#1-#3)-(#2-#4)*(#1-#5))}%
    \pgfmathsetmacro\cx{(\cA-\cy*(#2-#4))/(#1-#3)}%
    \pgfmathsetmacro\cr{sqrt((#1-\cx)*(#1-\cx)+(#2-\cy)*(#2-\cy))}%
    \pgfmathsetmacro\cA{atan2(#2-\cy,#1-\cx)}%
    \pgfmathsetmacro\cB{atan2(#6-\cy,#5-\cx)}%
    \pgfmathparse{\cB<\cA}%
    \ifnum\pgfmathresult=1
        \pgfmathsetmacro\cB{\cB+360}%
    \fi
    \draw (#1,#2) arc (\cA:\cB:\cr);%
}
\newcommand\getCoord[3]{\newdimen{#1}\newdimen{#2}\pgfextractx{#1}{\pgfpointanchor{#3}{center}}\pgfextracty{#2}{\pgfpointanchor{#3}{center}}}

\newcommand\qedshift{\vspace{-17pt}}
\newcommand\fakeqed{\pushQED{\qed}\qedhere}

\def\Xint#1{\mathchoice
   {\XXint\displaystyle\textstyle{#1}}%
   {\XXint\textstyle\scriptstyle{#1}}%
   {\XXint\scriptstyle\scriptscriptstyle{#1}}%
   {\XXint\scriptscriptstyle\scriptscriptstyle{#1}}%
   \!\int}
\def\XXint#1#2#3{{\setbox0=\hbox{$#1{#2#3}{\int}$}
     \vcenter{\hbox{$#2#3$}}\kern-.5\wd0}}
\def\ddashint{\Xint=}
\def\dashint{\Xint-}

\newcommand\separator{{\centering\rule{2cm}{0.2pt}\vspace{2pt}\par}}

\newenvironment{own}{\color{gray!70!black}}{}

\newcommand\makecenter[1]{\raisebox{-0.5\height}{#1}}

\mathchardef\mdash="2D

\newenvironment{significant}{\begin{center}\begin{minipage}{0.9\textwidth}\centering\em}{\end{minipage}\end{center}}
\DeclareRobustCommand{\rvdots}{%
  \vbox{
    \baselineskip4\p@\lineskiplimit\z@
    \kern-\p@
    \hbox{.}\hbox{.}\hbox{.}
  }}
\DeclareRobustCommand\tph[3]{{\texorpdfstring{#1}{#2}}}
\makeatother


\newcommand\Cl{\mathrm{Cl}}
\newcommand\omicron{o}
\begin{document}
\maketitle
{\small
\setlength{\parindent}{0em}
\setlength{\parskip}{1em}
Spinor structures and techniques are an essential part of modern mathematical physics. This course provides a gentle introduction to spinor methods which are illustrated with reference to a simple 2-spinor formalism in four dimensions. Apart from their role in the description of fermions, spinors also often provide useful geometric insights and consequent algebraic simplifications of some calculations which are cumbersome in terms of spacetime tensors.

The first half of the course will include an introduction to spinors illustrated by 2-spinors. Topics covered will include the conformal group on Minkowski space and a discussion of conformal compactifications, geometry of scri, other simple simple geometric applications of spinor techniques, zero rest mass field equations, Petrov classification, the Plucker embedding and a comparison with Euclidean spacetime. More specific references will be provided during the course and there will be worked examples and handouts provided during the lectures.

The second half of the course will include: Newman--Penrose (NP) spin coefficient formalism, NP field equations, NP quantities under Lorentz transformations, Geroch--Held--Penrose (GHP) formalism, modified GHP formalism, Goldberg-Sachs theorem, Lanczos potential theory, Introduction to twistors. There will be no problem sets.
\subsubsection*{Pre-requisites}
The Part 3 general relativity course is a prerequisite.

No prior knowledge of spinors will be assumed.
}
\tableofcontents

\section{Spin structures}
%Spinors are important in physics and mathematics:
%\begin{enumerate}
%  \item In quantum field theory, spinors let us work with fermions.
%  \item Dirac equation and dirac spinors let us describe quantum states of the relativistic electrons.
%  \item They are useful in applications to symplectic geometry, gauge theory, complex algebraic geometry, K-theory, orientation theory, etc. For example, they provide a proof of the Atiyah--Singer index theorem. This is useful in conformal field theories, for example.
%\end{enumerate}
%It is useful because they give the universal covers of $\SO(p, q)$.

Consider the special orthogonal group $\SO(n)$. We can naturally think of this as a subspace of \emph{all} endomorphisms $\R^n$, $\End(\R^n)$. Under this embedding, we can read off the Lie algebra of $\SO(n)$ as a certain linear subspace of $\End(\R^n)$. For $n > 2$, it is often useful to consider the universal cover $\Spin(n)$ of $\SO(n)$. It turns out this also naturally sits as a multiplicative subgroup of a certain algebra, called a \emph{Clifford algebra}. This lets us understand the Lie algebra of $\Spin(n)$ concretely as a certain linear subspace of the Clifford algebra.

\begin{defi}[Clifford algebra]\index{Clifford algebra}
  Let $K$ be a field, $V$ be a vector space over $K$, and $Q$ a non-degenerate quadratic form on $V$. We define the Clifford algebra $\Cl(V, Q)$ to be the quotient of the tensor power $\bigoplus_{i = 0}^\infty V^{\otimes i}$ by the relation $x^2 = Q(x)$ for all $x \in V$.
\end{defi}
Since $V$ includes into $\Cl(V, Q)$, we can often confuse $V$ with the image in $\Cl(V, Q)$ (as we already did), but we have to be more careful when we use suffix notation later on.

This enjoys the following universal property:
\begin{prop}
  Let $A$ be an associative algebra over $K$ and suppose $j: V \to A$ is a linear map such that $j(x)^2 = Q(x)$. Then there is a unique extension of the map to $\Cl(V, Q)$.
  \[
    \begin{tikzcd}
      V \ar[r, "j"] \ar[rd] & A\\
      & \Cl(V, Q) \ar[u, dashed]
    \end{tikzcd}
  \]
\end{prop}

In general, the Clifford algebra is not a commutative algebra. Instead, if $Q$ comes from a bilinear form $\bra \ph, \ph \ket$ (which is always the case in characteristic not $2$), then we have the formula
\[
  uv + vu = Q(u + v) - Q(u) - Q(v) = 2 \bra u, v\ket.
\]
It is often convenient to pick a basis for our Clifford algebra. To do so, we begin with an orthonormal basis $e_1, \ldots, e_n$ of $V$. Then
\[
  \{e_{j(1)} \cdots e_{j(k)} : 1 \leq j(1) < \cdots < j(k) \leq n\}
\]
is a basis for $\Cl(V, Q)$ and satisfies $e_i e_j = - e_j e_i$. Thus,
\[
  \dim \Cl(V, Q) = \sum_{i = 0}^n \binom{n}{k} = 2^n.
\]
Note that we can extend the quadratic form on $V$ to a quadratic form on all of $\Cl(V, Q)$, requiring that the distinct elements $e_{j(1)} \cdots e_{j(k)}$ to be orthogonal and $Q(e_{j(1)} \cdots e_{j(k)}) = Q(e_{j(1)}) \cdots Q(e_{j(k)})$. We have $Q(\lambda) = \lambda^2$. This is in fact independent of the basis chosen.

\subsection{Real Clifford Algebras}
Let's first try to understand real Clifford algebras, which are a bit more complicated than their complex cousins. By Sylvester's law of inertia, we may assume our quadratic form on $\R^n$ is given by
\[
  Q(v) = v_1^2 + \cdots + v_p^2 - v_{p + 1}^2 - \cdots - v_{p + q}^2,
\]
where $p + q = n$. We write the corresponding Clifford algebra as $\Cl_{p, q}(\R)$. It follows easily from definition that we have isomorphisms
\[
  \Cl_{0, 0}(\R) \cong \R, \quad \Cl_{0, 1}(\R) \cong \C,\quad \Cl_{0, 2}(\R) \cong \H.
\]
With a bit more work, one sees that
\[
  \Cl_{0, 3}(\R) \cong \H \oplus \H,\quad \Cl_{0, 4} \cong M_2(\H).
\]
In general, the following can be said about these Clifford algebras:
\begin{itemize}
  \item Every $\Cl_{p, q}(\R)$ is of the form $M_k(D)$ or $M_k(D) \oplus M_k(D)$, where $k$ is a positive integer (possibly $1$) and $D = \R, \C$ or $\H$.
  \item The type of $\Cl_{p, q}(\R)$ (i.e.\ the choice of $D$ and whether it is of the first or second form) depends only on $p - q \pmod 8$, and once we know the type, we can determine $k$ using the fact that $\dim \Cl_{p, q}(\R) = 2^{p + q}$. In particular, all these Clifford algebras are semi-simple.
  \item In fact, we have
    \begin{center}
      \begin{tabular}{ccccc}
        \toprule
        $p - q$ & $\Cl_{p, q}(\R)$ &\hphantom{asd} & $p - q$ & $\Cl_{p, q}(\R)$\\
        \midrule
        0 & $M_k(\R)$ & & 4 & $M_k(\H)$ \\
        1 & $M_k(\R) \oplus M_k(\R)$ & & $5$ & $M_k(\H) \oplus M_k(\H)$\\
        2 & $M_k(\R)$ & & 6 & $M_k(\H)$\\
        3 & $M_k(\C)$ & & 7 & $M_k(\C)$\\
        \bottomrule
      \end{tabular}
    \end{center}
\end{itemize}
The fact that the structure of $\Cl_{p, q}(\R)$ depends only on the value of $p - q \pmod 8$ is known as \term{Bott periodicity}.

\subsection{Complex Clifford Algebras}
In the complex case, there is no longer distinction between signatures, and $\Cl(V, Q)$ depends only on $\dim V$. We write these as $\Cl_n(\C)$. The structure is considerably simpler in the complex case, and the ``type'' of $\Cl_n(\C)$ depends only on $n \pmod 2$. Observe that for any real vector space $V$ and quadratic form $Q$, we always have
\[
  \Cl(V, Q) \otimes \C = \Cl(V \otimes \C, Q \otimes \C).
\]
Thus, we have
\[
  \Cl_{p + q}(\C) = \Cl_{p, q}(\R) \otimes \C.
\]
From the above, we can read off
\begin{center}
  \begin{tabular}{cc}
    \toprule
    $n$ & $\Cl_{n}(\C)$ \\
    \midrule
    0 & $M_k(\C)$\\
    1 & $M_k(\C) \oplus M_k(\C)$\\
    \bottomrule
  \end{tabular}
\end{center}

%\begin{center}
%  \begin{tabular}{cccccccccc}
%    \toprule
%    $p\backslash q$ & 0 & 1 & 2 & 3 & 4 & 5 & 6 & 7 & 8 \\
%    \midrule
%    0 & $\R$ & $\C$ & $\H$ & $\H^2$ & $M_2(\H)$ & $M_4(\C)$ & $M_8(\R)$ & $M_8(\R)^2$ & $M_{16}(\R)$ \\
%    1 & $\R^2$ & $M_2(\R)$ & $M_2(\C)$ & $M_2(\H)$ & $M_2(\H)^2$ & $M_4(\H)$ & $M_8(\C)$ & $M_{16}(\R)$\\
%    2 & $M_2(\R)$\\
%    3 \\
%    4 \\
%    5 \\
%    6 \\
%    7 \\
%    8 \\
%    \bottomrule
%  \end{tabular}
%\end{center}
%\begin{itemize}
%  \item $C_{0, 0}(\R) \cong \R$.
%  \item $C_{0, 1}(\R) \cong \C$.
%  \item $C_{0, 2}(\R) \cong \H$.
%  \item $C_{0, 3}(\R) \cong \H \oplus \H$.
%\end{itemize}

%\begin{defi}[Conformal structure]\index{conformal structure}
%  A \emph{conformal structure} on a real manifold is a smoothly varying family of light cones in the tangent spaces, such that at each point the null cone is defined as the zero set of a non-degenerate real quadratic form.
%\end{defi}

\section{The Lorentz group}
\subsection{The universal cover of the Lorentz group}
The group $\SO(1, 3)$ is not simply connected, which is not good. In particular, representations of the Lie algebra $\so(1, 3)$ need not lift to representations of $\SO(1, 3)$, which is undesirable. Thus, we would like to find a universal cover of $\SO(1, 3)$. Of course, it exists by general theory, but we would like an explicit identification of this universal cover.

First consider the case of $\SO(3)$, which is a natural subgroup of $\SO(1, 3)$. It is ``well-known'' that its universal cover is $\SU(2)$, and this has a Clifford algebra interpretation.

Recall that $C_{3, 0} \cong M_2(\C)$. This identification can be obtained explicitly by the inclusion of $\R^3$ into $M_2(\C)$ by the map
\begin{align*}
  x &\mapsto \tilde{x}\\
  \begin{pmatrix}
    x^1\\x^2\\x^3
  \end{pmatrix}&\mapsto
  \frac{1}{\sqrt{2}}(x^1 \sigma_1 + x^2 \sigma_2 + x^3 \sigma_3) \\
  &= \frac{1}{\sqrt{2}}
  \begin{pmatrix}
    x^3 & x^1 - ix^2\\
    x^1 + i x^2 & -x^3
  \end{pmatrix}.
\end{align*}
The presence of the $\frac{1}{\sqrt{2}}$ is purely by convention. This gives an isomorphism (of vector spaces) between $\R^3$ and the traceless self-adjoint matrices, and this ``remembers'' the norm in the sense that $\|x\|^2 = - 2 \det \tilde{x}$.

The group of matrices in $\GL_2(\C)$ that preserve this subspace under conjugation is exactly $\U(2)$, and the spin group $\Spin(3)$ is $\SU(2)$, and acts on $M_2(\C)$ by conjugation. Moreover, the only elements that act trivially are $\pm I$, giving a short exact sequence
\[
  0 \to \Z_2 \to \SU(2) \to \SO(3) \to 0.
\]
The surjectivity follows from the fact that $\SO(3)$ is connected, and the derivative of the map at the identity is non-singular.

Moreover, since $\SU(2) \cong S^3$ as spaces, we know $\SU(2)$ is simply connected, and this exhibits $\SU(2)$ as a double cover.

Instead of trying to explicitly understand the Clifford algebra interpretation for $\SO(1, 3)$, it is easier to directly extend the above map to work with $\SO(1, 3)$.

We define a map $\R^4 \to M_2(\C)$ by sending
\begin{align*}
  x &\mapsto \tilde{x}\\
  \begin{pmatrix}
    x^0\\x^1\\x^2\\x^3
  \end{pmatrix}&\mapsto
  \frac{1}{\sqrt{2}}(x^0 I + x^1 \sigma_1 + x^2 \sigma_2 + x^3 \sigma_3) \\
  &= \frac{1}{\sqrt{2}}
  \begin{pmatrix}
    x^0 + x^3 & x^1 - ix^2\\
    x^1 + i x^2 & x^0-x^3
  \end{pmatrix}.
\end{align*}
Thus, we now map $\R^4$ into the space of all self-adjoint matrices. As before, we have
\[
  \|x\|^2 = (x^0)^2 - (x^1)^2 - (x^2)^2 - (x^3)^2 = 2\det \tilde{x}.
\]
Now in fact conjugating by $\SL(2, \C)$\index{$\SL(2, \C)$} preserves the space of Hermitian matrices, and as before $\pm I$ are the only matrices that act trivially. So we have a short exact sequence
\[
  0 \to \Z_2 \to \SL(2, \C) \to \SO(1, 3) \to 0
\]
that exhibits $\SL(2, \C)$ as a double cover of $\SO(1, 3)$.

Since $\SL(2, \C)$ deformation retracts to $\SU(2)$ by the Gram-Schmidt process, we know that
\begin{prop}
  $\SL(2, \C)$ is simply connected.
\end{prop}

\begin{cor}
  $\SL(2, \C)$ is the universal cover of $\SO(1, 3)$.
\end{cor}

It turns out the identification of $\R^4$ with the self-adjoint matrices brings us some unexpected benefits --- rotations and boosts now take a particularly simple form (whose proof is a simple verification):
\begin{prop}\leavevmode
  \begin{itemize}
    \item Boosts in the 0-3 plane (with the 1-2 plane fixed) are given by conjugation by
      \[
        \begin{pmatrix}
          e^{\psi/2} & 0\\
          0 & e^{-\psi/2}
        \end{pmatrix} \in \SL(2, \C)
      \]
    \item Rotation in the 1-2 plane (with the 0-3 plane fixed) by $\varphi$ is given by conjugation by
      \[
        \begin{pmatrix}
          e^{i\varphi/2} & 0\\
          0 & e^{-i\varphi/2}
        \end{pmatrix} \in \SL(2, \C)
      \]
  \end{itemize}
\end{prop}
Note that in the case of rotation, while rotating by $\varphi$ and rotating by $\varphi + 2\pi$ are the same rotations, they are represented by different elements of $\SL(2, \C)$. This is, of course, reflecting the fact that $\SL(2, \C)$ is a non-trivial double cover of $\SO(1, 3)$.

We might ask --- where did this construction come from? It doesn't, \emph{a priori}, seem like something of Clifford algebra origin, but it turns out we can provide some Clifford algebra understanding of this.

First observe $\dim_\R C_{1, 3} = 2^4$, while $\dim_\R M_{2^k}(\C) = 2^{2k + 1}$, so there is no hope in identifying $C_{1, 3}$ as some complex matrix algebra (of course, we can also look up our table of identifications above, and see that $C_{1, 3}$ is something else). However, we can look one level higher, and see that $C_{2, 3} \cong M_4(\C)$. Thus, we are lead to consider the following composition
\[
  \R^4 \hookrightarrow \R^5 \hookrightarrow C_{2, 3} \cong M_4(\C).
\]
By convention, we label the basis elements of $\R^5$ as $\gamma^0, \gamma^1, \gamma^2, \gamma^3, \gamma^5$, where $\gamma^0$ and $\gamma^5$ are the elements that square to $1$. This are represented as elements in $M_4(\C)$ by
\[
  \gamma^0 =
  \begin{pmatrix}
    \mathbf{0} & \mathbf{1}\\
    \mathbf{1} & \mathbf{0}
  \end{pmatrix},\quad
  \gamma^i =
  \begin{pmatrix}
    \mathbf{0} & \sigma^i\\
    -\sigma^i & \mathbf{0}
  \end{pmatrix},\quad
  \gamma^5 =
  \begin{pmatrix}
    \mathbf{1} & \mathbf{0}\\
    \mathbf{0} & -\mathbf{1}
  \end{pmatrix}
\]
for $i = 1, 2, 3$. We now immediately see that our inclusion of $\R^4$ to $M_2(\C)$ is given by picking out the top right-hand corner of their image in $M_4(\C)$.

\subsection{More about \tph{$\SL(2, \C)$}{SL(2, C)}{SL(2, &\#x2102;)}}
Let us think a bit more about $\SL(2, \C)$. This acts naturally on the space $\C^2$, and the elements of $\SL(2, \C)$ are picked out as those that have determinant $1$. Recall that given a linear map $T: V \to V$ with $\dim V = n$, the determinant of $T$ is defined to be the induced map $T^{\wedge n}: \bigwedge^n V \to \bigwedge^n V$, which is given by multiplication by a scalar. In particular, $\SL(2, \C)$ consists of the matrices $T$ such that $Tx \wedge Ty = x \wedge y$ for all $x$ and $y$.

It is convenient to think of $\wedge$ as a bilinear map. To do so, we \emph{pick} an identification $\bigwedge^2 \C^2 \cong \C$. Since $\wedge$ is naturally anti-symmetric and non-degenerate, this defines a \term{symplectic form} on $\C^2$, and we can thus identify $\SL(2, \C) = \Sp(2, \C)$.

The bilinear form is denoted $\varepsilon$, with
\[
  \varepsilon_{ab} =
  \begin{pmatrix}
    0 & 1\\
    -1 & 0
  \end{pmatrix}
\]
in an appropriate basis. Any basis for which $\varepsilon_{ab}$ looks like this is known as a \term{spinor basis}. Thus, $\SL(2, \C)$ is equivalently the matrices that send a spinor basis to another spinor basis.

Since $\varepsilon$ is not symmetric, we have to be careful about how we order things. The matrix $\varepsilon_{ab}$ is \emph{defined} to be so that
\[
  \xi^a \chi^b \varepsilon_{ab} = \xi \wedge \chi.
\]
The bilinear form $\varepsilon$ induces an isomorphism between $\C^2$ and its dual, and we can use this to raise and lower indices. Again, we have to fix an order, and we set the dual of $\xi$ to be the map $\xi \wedge -$. Thus, we must define
\[
  \xi_a = \xi^b \varepsilon_{ba}
\]
One checks that the condition $\varepsilon^{ab} \varepsilon_{ac} \varepsilon_{bd} = \varepsilon_{cd}$ forces
\[
  \varepsilon^{ab} = \varepsilon_{ab}.
\]
Thus, it follows that
\[
  \varepsilon^{ab} \varepsilon_{cb} \equiv \varepsilon_c\!^a = \delta_c\!^a
\]
Therefore, we have
\[
  \xi^b = \varepsilon^{ba} \xi_a.
\]
Observe that for consistency, we must set
\[
  \varepsilon^a\!_c = \varepsilon_b\!^d \varepsilon^{ab} \varepsilon_{dc} = - \varepsilon_a\!^c.
\]
\subsection{Conjugate spinors}
As a complex Lie group, $\SL(2, \C)$ acts naturally on $\C^2$, as we have previously discussed. We call this the \term{fundamental representation}. However, for us, $\SL(2, \C)$ really acts as a \emph{real} Lie group $\Spin(1, 3)$. Given this, one can consider the conjugate representation of $\SL(2, \C)$ on $\C^2$, given by
\[
  N \cdot \bar{\xi} = N^* \bar{\xi}.
\]
Here $N^*$ denotes the result of taking the complex conjugate of every entry in $N$, \emph{without} taking a transpose. Elements in the conjugate representation are not compatible with elements in the fundamental representation, and in particular we cannot contract them. Thus, it is common to use primes to label their indices, e.g.\ $\bar{\xi}^{a'}$. It is also common to write some bars on top of their names, as we have done so far.

As before, we can fix a antisymmetric form $\bar{\varepsilon}^{a'b'}$, which has the same entries as $\varepsilon^{a'b'}$. By fixing appropriate bases, complex conjugation defines an anti-linear map between the fundamental and conjugate representations.

 % obtain by complexifying C^2
\subsection{Spinors in Minkowski spacetime}
Let $M$ be the real Minkowski spacetime, and the trivial spin structure gives rise to a spinor bundle $M \times T(S)$. We can complexify Minkowski space as well as a metric, so that
\[
  \d s^2 = g_{ab} \;\d z^a \;\d z^b.
\]
Note that this is not the usual Hermitian complexification.

\section{\tph{$\SL(2, \C)$}{SL(2, C)}{SL(2, &\#x2102;)}}
\subsection{\tph{$\SL(2, \C)$}{SL(2, C)}{SL(2, &\#x2102;)} and \texorpdfstring{$\SO(1, 3)$}{SO(1, 3)}}
Fast-forwarding to the end of the chapter, the conclusion of the section is
\begin{thm}\index{$\SL(2, \C)$}
  $\SL(2, \C)$ is simply connected, and there is a non-trivial double cover $\SL(2, \C) \to \SO(1, 3)$. Hence $\Spin(1, 3) \cong \SL(2, \C)$.
\end{thm}

Establishing this theorem will involve thinking about the representations of $\SL(2, \C)$, and it will take a while before $\R^{1 + 3}$ and $\SO(1, 3)$ resurface from the discussion. The formalism developed in this identification will be very useful for later purposes.

First of all, we prove that $\SL(2, \C)$ is simply connected.
\begin{prop}
  $\SL(2, \C)$ is simply connected.
\end{prop}

\begin{proof}
  By the Gram--Schmidt process, $\SL(2, \C)$ deformation retracts onto $\SU(2)$, and $\SU(2) \cong S^3$.
\end{proof}

We begin with the defining representation of $\SL(2, \C)$, which we shall call $S$. Thus, $S \cong \C^2$, and $\SL(2, \C)$ acts on $S$ in the usual way, as we learnt in IA Vectors and Matrices. It is important to make the point that despite everything here seems complex, we always think of $\SL(2, \C)$ as a \emph{real} Lie group. In particular, we shall happily take complex conjugations without worrying about holomorphicity. $S$ is both a complex and a real representation of $\SL(2, \C)$ (which, I shall reiterate, is thought of as a real Lie group, like $\U(1)$), and they will take different roles at different time.

There are many representations we can produce out of $S$. The first obvious operation is taking the dual, and this gives $S^*$. The next operation is formed by taking the complex conjugate. This is given by the action of $\SL(2, \C)$ on $\C^2$ by
\begin{align*}
  \SL(2, \C) \times \C^2 &\to \C^2\\
  (A, \kappa) &\mapsto A^* \kappa,
\end{align*}
where $A^*$ is the matrix obtained by taking the complex conjugate of every entry in $A^*$, \emph{without} taking a transpose. We call this representation $S'$. Finally, we can perform both of these actions, and produce $S'^*$. We shall write $T(S)$ for the (commutative) algebra freely generated by tensor products of the different versions of $S$, and this is called the \term{spinor algebra}

Observe that as \emph{real} representations, $S$ is isomorphic to $S'$ --- after fixing a basis, there is a natural conjugation map $S \to S'$ given by complex conjugating each component. We usually write this as $\kappa \mapsto \bar{\kappa}$. This is not a $\C$-linear map, so does not give an isomorphism of \emph{complex} representations. We shall, from now on, fix such an isomorphism, without necessarily picking a basis. However, once we have fixed this isomorphism, picking a basis of one of $S$ and $S"$ gives a basis of the other.

In index notation, we shall write elements of $S$ with upper, capital indices, e.g.\ $\kappa^A$, and elements of $S'$ with upper, capital indices followed by a prime, e.g.\ $\bar{\kappa}^{A'}$. The elements in the duals are written with lower indices.

A fun fact: If we take $S$ and treat it as a real representation, then complexifying it makes it split as $S\otimes_\R \C \cong S \oplus S'$. Similarly, if we take the real Lie algebra $\sl(2, \C)$ and complexify, it splits as $\sl(2, \C) \oplus \sl(2, \C)$, and $S$ and $S'$ are the fundamental representations of each copy of $\sl(2, \C)$.

To understand $\SL(2, \C)$ better, recall that $\SL(2, \C)$ is defined to be the two-by-two matrices of determinant $1$. Recall also the definition of a determinant: given a linear map $T: V \to V$, where $\dim V = n$, the determinant of $T$ is defined to be the map $T^{\wedge n}: V^{\wedge n} \to V^{\wedge n}$, which is multiplication by a single complex number. In the particular case $V = S = \C^2$, we can fix an isomorphism $S^{\wedge 2} \cong \C$. Then $\wedge: S \otimes S \to S^{\wedge 2} \cong \C$ defines a non-degenerate anti-symmetric form, i.e.\ a symplectic form. Thus, $\SL(2, \C)$ is the space of all matrices that preserve the symplectic form, i.e.\ $\SL(2, \C) \cong \Sp(2, \C)$.

Since this symplectic form is quite important, we shall give it a name, $\varepsilon$, and establish the index conventions for it. Since $\varepsilon$ is not symmetric, we have to be a bit careful. We define $\varepsilon_{AB}$ to be such that
\[
  \kappa^A \tau^B \varepsilon_{AB} = \kappa \wedge \tau = \varepsilon(\kappa, \tau).
\]
The non-degeneracy of $\varepsilon$ induces an isomorphism between $S$ and $S^*$, sending $\kappa$ to $\kappa \wedge -$. In index notation, this means
\[
  \kappa_A = \kappa^B \varepsilon_{BA}.
\]
The dual form $\varepsilon^{AB}$ is required to satisfy $\varepsilon^{AB} \varepsilon_{AC} \varepsilon_{BD} = \varepsilon_{CD}$, and this forces
\[
  \varepsilon^{AB} = \varepsilon_{AB}
\]
in a symplectic/spinor basis\index{symplectic basis}\index{spinor basis}, i.e.\ a basis where
\[
  \varepsilon_{AB} =
  \begin{pmatrix}
    0 & 1\\
    -1 & 0
  \end{pmatrix}.
\]
From this, it follows that
\[
  \varepsilon^{AB} \varepsilon_{CB} \equiv \varepsilon_C\!^A = \delta_C\!^a,
\]
and therefore we must set
\[
  \kappa^B = \varepsilon^{BA} \kappa_A.
\]
Finally, observe that for consistency, we have
\[
  \varepsilon^A\!_C = \varepsilon_B\!^D \varepsilon^{AB} \varepsilon_{DC} = - \varepsilon_A\!^C.
\]
We can similarly form $\bar{\varepsilon}_{A'B'} = \varepsilon_{AB}$, and will usually write it without the bar.

Recall that we previously had a conjugation map $S \to S'$, that somehow remembers our complex structure. Usually, we can recover ``real'' things by looking at the fixed point of a conjugation map, but here it doesn't map to itself. The next best thing we can do is to consider the tensor product $S \otimes S'$. By performing complex conjugation on both entries, and then composing with a swap, we obtain a conjugation map
\[
  \begin{tikzcd}[row sep=tiny]
    S \otimes S' \ar[r] & S' \otimes S \ar[r] & S \otimes S'\\
    \kappa \otimes \bar{\tau} \ar[r, maps to] & \bar{\kappa} \otimes \tau \ar[r, maps to] & \tau \otimes \bar{\kappa}
  \end{tikzcd}
\]
We can take the tensor product of $\varepsilon$ and $\bar{\varepsilon}$, which now becomes a \emph{symmetric} bilinear form $\varepsilon \otimes \bar{\varepsilon}$ from $S \otimes S'$ to $\C$, that is equivariant under conjugation. Restricting to the real parts, i.e.\ the parts fixed by conjugation, this descents to a symmetric bilinear form $g: \R^4 \otimes \R^4 \to \R$.

The claim is, of course, that this signature of this bilinear form is $(1, 3)$, and thus gives an embedding of $\R^{1 + 3}$ into $S \otimes S'$. Since the conjugation is $\SL(2, \C)$ equivariant, this restricts the $\SL(2, \C)$ to one on $\R^{1 + 3}$, and the since the bilinear form is also $\SL(2, \C)$ equivariant by construction, this induces the desired map $\SL(2, \C) \to \SO(1, 3)$.

To verify this claim, it suffices to write everything out and look at it. We can write an element $V^{AA'} \in S \otimes S'$ as a matrix
\[
  V^{AA'} =
  \begin{pmatrix}
    V^{00'} & V^{01'}\\
    V^{10'} & V^{11'}
  \end{pmatrix},
\]
and then our conjugation action corresponds to taking the Hermitian conjugate. Moreover, one checks that $(\varepsilon \otimes \bar{\varepsilon})(V, V) = \det V$. The elements of the ``real'' part then correspond to the Hermitian matrices, which are in general of the form
\[
  V^{AA'} =
  \frac{1}{\sqrt{2}}
  \begin{pmatrix}
    V^0 + V^3 & V^1 - i V^2\\
    V^1 + i V^2 & V^0 V^3
  \end{pmatrix}.
\]
Identifying this with $V = (V^0, V^1, V^2, V^3) \in \R^4$, we easily read off the norm squared of $V$ as
\[
  \|V\|^2 = \frac{1}{2} \Big((V^0)^2 - (V^1)^2 - (V^2)^2 - (V^3)^2\Big),
\]
which, up to a scaling, is just the usual $(1, 3)$-norm on $\R^{1 + 3}$. Thus, we have a canonical copy of $\R^{1 + 3}$ in our spinor algebra.

Under this presentation, $\SL(2, \C)$ acts by conjugation on the Hermitian matrices, and $\pm I$ are the only matrices that act trivially. Thus, we have a short exact sequence
\[
  0 \to \Z_2 \to \SL(2, \C) \to \SO(1, 3) \to 0
\]
that exhibits $\SL(2, \C)$ as a double cover of $\SO(1, 3)$.

It turns out the identification of $\R^4$ with the Hermitian matrices brings us some unexpected benefits --- rotations and boosts now take a particularly simple form (whose proof is a simple verification):
\begin{prop}\leavevmode
  \begin{itemize}
    \item Boosts in the 0-3 plane (with the 1-2 plane fixed) are given by conjugation by
      \[
        \begin{pmatrix}
          e^{\psi/2} & 0\\
          0 & e^{-\psi/2}
        \end{pmatrix} \in \SL(2, \C)
      \]
    \item Rotation in the 1-2 plane (with the 0-3 plane fixed) by $\varphi$ is given by conjugation by
      \[
        \begin{pmatrix}
          e^{i\varphi/2} & 0\\
          0 & e^{-i\varphi/2}
        \end{pmatrix} \in \SL(2, \C)
      \]
  \end{itemize}
\end{prop}
Note that in the case of rotation, while rotating by $\varphi$ and rotating by $\varphi + 2\pi$ are the same rotations, they are represented by different elements of $\SL(2, \C)$. This is, of course, reflecting the fact that $\SL(2, \C)$ is a non-trivial double cover of $\SO(1, 3)$.

\subsection{Vectors from spinors}
So we have now identified our ``vectors'', i.e.\ elements in $\R^{1 + 3}$ as certain special spinors, namely the Hermitian elements in $S \otimes S'$. Thus, given a spinor, there is an easy way to produce a vector.

To be concrete, if $\kappa^A \in S$, then $\bar{\kappa}^{A'} \in S'$. Thus, we can form
\[
  k^a = k^{AA'} = \kappa^A \bar{\kappa}^{A'} = \kappa \otimes \bar{\kappa}.
\]
We write it as $k^a$ to signify the fact that we are thinking of it as living in $\R^{1 + 3}$. Can we obtain all vectors this way? No! We can easily compute
\[
  k^a k_a = \kappa^A \bar{\kappa}^{A'} \kappa_A \bar{\kappa}_{A'} = 0,
\]
using that the symplectic form is anti-symmetric. Thus, this procedure only produces null vectors.

An obvious next question is, does this produce all null vectors? Let us look at dimensions. The space of null vectors has real dimension $3$, while the space of all $\kappa^A$ have dimension $4$. However, we observe that replacing $\kappa^A$ with $e^{i\theta} \kappa^A$ does not change $k^a$. So we have (at least) one dimension of degeneracy. So the answer might be yes. It turns out this is close, but not quite it.

\begin{lemma}
  A vector is of the form $\kappa^A \bar{\kappa}^{A'}$ iff it is a future-pointing null vector. Moreover, $\kappa^A \bar{\kappa}^{A'}$ is unique up to a phase.
\end{lemma}

\begin{proof}
  We write $k^a = k^{AA'}$ as a matrix
  \[
    k^{AA'} =
    \begin{pmatrix}
      \xi & \bar{\varsigma}\\
      \varsigma & \eta
    \end{pmatrix}.
  \]
  Then the null condition dictates that this has determinant $0$, and the future-pointing condition requires $\xi$ and $\eta$ to be positive (they must have the same sign since $\det k = 0$ and $\varsigma \bar\varsigma$ is always non-negative).

  From this, we see that $\kappa^A \bar{\kappa}^{A'}$ is always null and future-pointing. Conversely, if $k^{AA'}$ is null and future-pointing, then the vanishing of the determinant implies the rows are proportional to each other. We may wlog assume $\varsigma = z \xi$. Then
  \[
    \eta = z \bar{\varsigma} = |z|^2 \bar{\xi} = |z|^2 \xi,
  \]
  using that $\xi$ is real. Thus, we can write
  \[
    k^{AA'} = \xi
    \begin{pmatrix}
      1 & \bar{z}\\
      z & |z|^2
    \end{pmatrix} =
    \begin{pmatrix}
      \sqrt{\xi}\\ \sqrt{\xi} z
    \end{pmatrix}
    \begin{pmatrix}
      \sqrt{\xi} & \sqrt{\xi}\bar{z}
    \end{pmatrix}.
  \]
  Finally, to show the uniqueness of $\kappa$, note that except in degenerate cases, we can always fix the first component of $\kappa$ to be positive and real, and we see that picking different options for the second component give rise to different $k^a$'s.
\end{proof}

Using this, we now have a way to represent spinors as elements in ``real space'' $\R^{1 + 3}$, by mapping $\kappa^A$ to $k^a = \kappa^A \bar{\kappa}^{A'}$. By doing so, we have lost information about the phase of $\kappa^A$, and we might hope to find something that knows about this phase as well. It turns out we can quite do that, but we can recover the phase up to a sign. This is perhaps expected, reflecting the fact that $\SL(2, \C) \to \SO(1, 3)$ is a double cover, and not an isomorphism.

To describe this construction, suppose $\kappa^A$ is a non-zero spinor. Suppose $\tau^A$ is another spinor such that $(\kappa^A, \tau^A)$ is a spinor basis. This $\tau^A$ always exists, and is well-defined up to a multiple of $\kappa^A$. Consider
\[
  L^a = \kappa^A \bar{\tau}^{A'} + \bar{\kappa}^{A'} \tau^A.
\]
This is then well-defined up to a multiple of $k^a$. We call $k^a$ the \term{flag pole}, and the space of all possible $L^a$ the \term{flag plane}.

Observe that if we multiply $\kappa_A$ by a phase $e^{i\theta}$, then $\kappa^A \bar{\tau}^{A'}$ gets multiplied by $e^{2i\theta}$, while $\bar{\kappa}^{A'} \tau^A$ gets multiplied by $e^{-2i\theta}$. In either case, $L^a$ doesn't change when we multiply by $-1$. This is all that is lost.

\begin{lemma}
  The pair $(k^a, L^a)$ determines $\kappa^A$ up to a sign. % prove this
\end{lemma}

What do we get out of this? Observe that by picking out $\R^{1 + 3}$ as a subspace of $S \otimes S'$, it comes with a \emph{canonical} time orientation, namely that given by the requirement that future null vectors are those of the form $\kappa^A \kappa^{A'}$.

\section{Newman--Penrose formalism}
\subsection{Spin space}
Spin space is the space spinors live in. We write $\boldsymbol\zeta, \boldsymbol\eta$ for ``spin vectors''. These are elements that belong to \emph{spin space} $S$. As a vector space, $S$ is a two-dimensional complex vector space. Further, there exists an non-degenerate anti-symmetric bilinear form $[\ph, \ph]: S \otimes S \to \C$.

A \term{spin basis} is a basis $\{\boldsymbol\omicron, \boldsymbol\iota\}$ such that
\[
  [\boldsymbol\omicron, \boldsymbol\iota] = 1 = - [\boldsymbol\iota, \boldsymbol\omicron].
\]
Then we define the components of some spin vector $\boldsymbol\zeta \in S$ by
\[
  \boldsymbol\zeta = \zeta^0 \boldsymbol\omicron + \zeta^1 \boldsymbol\iota.
\]
We can compute these coefficients by
\[
  \zeta_0 = [\boldsymbol\zeta, \boldsymbol\iota], \zeta^1 = [\boldsymbol\zeta, \boldsymbol\omicron].
\]
So
\[
  \boldsymbol\omicron = (1, 0),\quad \boldsymbol\iota = (0, 1).
\]
By direct calculation,
\[
  [\boldsymbol\zeta, \boldsymbol\eta] = \zeta^0 \eta^1 - \zeta^1 \eta^0.
\]
We introduce the \term{Levi--Civita spinor} $\varepsilon_{AB}$ of $S_{AB} = (S^*)^{\otimes 2}$ defined by
\[
  [\zeta, \eta] = \varepsilon_{AB} \zeta^A \zeta^B.
\]
Then antisymmetry entails $\varepsilon_{AB} = - \varepsilon_{BA}$.
\begin{prop}
  We have
  \begin{enumerate}
    \item $\varepsilon_{AB} = \varepsilon_{BA}$.
    \item $\zeta_B = \varepsilon_{AB} \zeta^A = - \varepsilon_{BA} \zeta^A$. This is defined so that $[\zeta, \eta] = \zeta_B \eta^B$
    \item $\varepsilon^{AB} \varepsilon_{AC} = \delta^B_C = \varepsilon_C\!^B = - \varepsilon^B\!_C$.
  \end{enumerate}
\end{prop}
In general, if we have multi-spinors $\chi^{M\ldots NA}$, then
\[
  \chi^{M\ldots NA} = \varepsilon^{AB} \chi^{M\ldots N}\!_B = - \chi^{M\ldots N}\!_B \varepsilon^{BA} = \chi^{M\ldots NB} \varepsilon_B\!^A
\]
We also have
\[
  \varepsilon_{A[B} \varepsilon_{CD]} = 0.
\]
Observe that this is the same as saying
\[
  \varepsilon_{AB} \varepsilon_{CD} + \varepsilon_{AC} \varepsilon_{DB} + \varepsilon_{AD} \varepsilon_{BC} = 0.
\]
Raising $C$ and $D$ and rearranging, we get
\[
  \varepsilon_A\!^C \varepsilon_B\!^D - \varepsilon_B\!^C \varepsilon_A\!^D = \varepsilon_{AB} \varepsilon^{CD}.
\]
Transvecting (i.e.\ contracting) with $\chi_{\cdots CD \cdots}$, we get
\[
  2\chi_{\cdots [AB] \cdots} = \chi_{\cdots AB \cdots} - \chi_{\cdots BA \cdots } = \chi_{\cdots C}\!^C \!_{\cdots \varepsilon_{AB}}.
\]
In particular, if $\chi_{\cdots [AB]\cdots} = \chi_{AB \cdots}$, then we get
\[
  \chi_{\cdots AB \cdots} = \frac{1}{2} \chi_{\cdots C}\!^C_{\cdots} \varepsilon_{AB}.
\]
This is just the statement that every anti-symmetric matrix is a multiple of $\varepsilon$.

For a general $\chi$, we can always write it as a sum of symmetric part and a sum of antisymmetric part,
\[
  \chi_{\cdots AB \cdots} = \chi_{\cdots (AB) \cdots} + \chi_{\cdots [AB] \cdots},
\]
it follows that
\begin{prop}
  \[
    \chi_{\cdots AB\cdots} = \chi_{\cdots (AB) \cdots} + \frac{1}{2} \chi_{\cdots C}\!^C_{\cdots} \varepsilon_{AB}.
  \]
\end{prop}
Thus, in some sense, only symmetric tensors matter.

\subsection{Infeld--van der Waerden symbols}
The \term{Infeld--van der Waerden symbols} are connecting quantities which allow transference between world tensors and $2$-spinors. They are depicted as $\sigma^{\mathbf{a}}_{AB'}$, given by
\[
  \sigma^{\mathbf{0}}_{AB'} = \frac{1}{\sqrt{2}}
  \begin{pmatrix}
    1 & 0\\
    0 & 1
  \end{pmatrix},\quad \sigma^{\mathbf{1}}_{AB'} = \frac{-1}{\sqrt{2}}
  \begin{pmatrix}
    0 & 1\\
    1 & 0
  \end{pmatrix},\quad \sigma^{\mathbf{2}}_{AB'} = \frac{-1}{\sqrt{2}}
  \begin{pmatrix}
    0 & -i\\
    i & 0
  \end{pmatrix},\quad \sigma^{\mathbf{3}}_{AB'} = \frac{-1}{\sqrt{2}}
  \begin{pmatrix}
    1 & 0 \\
    0 & -1
  \end{pmatrix}.
\]
These are Hermitian, so that $\overline{\sigma^{\mathbf{a}}_{AB'}} = \sigma^\mathbf{a}_{AB'}$, where by definition,
\[
  \overline{\sigma^{\mathbf{a}}_{AB'}} = \bar{\sigma}^{\mathbf{a}}_{A'B}.
\]
For a general spinors to be real, we need
\[
  \bar{\chi}_{AA' \cdots}\!^{BB'\cdots} = \chi_{AA'\cdots }\!^{BB'\cdots}
\]
The relationship between $\sigma^{\mathbf{a}}_{AB'}$ and metric tensors is
\[
  g_{\mathbf{a}\mathbf{b}} = \varepsilon_{AB} \varepsilon_{A'B'} \sigma_a^{AA'} \sigma_\mathbf{b}^{BB'}.
\]
Introduce
\[
  \sigma_{\mathbf{a}}\!^{AA'} \sigma^{\mathbf{b}}\!_{AA'} = \delta_{\mathbf{a}}^{\mathbf{b}},\quad \sigma^{\mathbf{a}}_{AA'} \sigma_{\mathbf{a}}\!^{BB'} = \varepsilon_A\!^B \varepsilon_{A'}\!^{B'}
\]
we can now show that this is equivalent to
\[
  g_{\mathbf{a}\mathbf{b}} \sigma^{\mathbf{a}}\!_{AA'} \sigma^\mathbf{b}\!_{BB'} = \varepsilon_{AB} \varepsilon_{A'B'}.
\]
\subsection{Null vectors}

\printindex
\end{document}
